\chapter{Advanced Plotting}
\label{chap:advancedplots}


Python will display functions of the type $F(x,y)$, either by
making a contour plot (like a topographic map) or by displaying the
function as height above the $xy$ plane like a perspective drawing.
You can also display functions like $\vec{F}(x,y)$, where $\vec{F}$
is a vector-valued function, using vector field plots.

\medskip

\section{Making 2-D Grids}
\index{Meshgrid}

Before we can display 2-dimensional data, we need to define arrays
{\tt X} and {\tt Y} that span the region that you want to plot, then create
the function $F(x,y)$ over the plane.  First, you need to understand
how Python goes from
one-dimensional arrays \texttt{x} and \texttt{y} to two-dimensional
matrices \texttt{X} and \texttt{Y} using the commands {\tt
meshgrid} and {\tt mgrid}. Begin by executing the following
example.
\begin{codeexample}
\begin{VerbatimOut}{\listingFile}
from numpy import arange,meshgrid,cos,exp,pi,mgrid
from matplotlib import pyplot
from mpl_toolkits.mplot3d import Axes3D  #<- You need this to make 3D plots

# Define the arrays x and y
# Don't make the step size too small or you will kill the
# system (you have a large, but finite amount of memory)
x=arange(-1,1,0.1)
y=arange(0,1.5,0.1)

# Use meshgrid to convert these 1-d arrays into 2-d matrices
# of x and y values over the plane
X,Y=meshgrid(x,y)
#X,Y=mgrid[-1:1:0.1,0:1.5:0.1]  # You could also do this in place of meshgrid
# Get F(x,y) by using F(X,Y). Note the use of .* with X and Y
# rather than with x and y
F=(2-cos(pi*X))*exp(Y)

# Plot the function
fig = pyplot.figure()
ax=fig.gca(projection='3d')
ax.plot_surface(X,Y,F)
pyplot.xlabel('x')
pyplot.ylabel('y')
pyplot.show()
\end{VerbatimOut}
\end{codeexample}
The picture should convince you that Python did indeed make
things two-dimensional, and that this way of plotting could be very
useful. But exactly how Python did it is tricky, so pay close
attention.  First notice that anytime you want a 3 dimensional plot,
you will need the following lines:

\begin{Verbatim}
from mpl_toolkits.mplot3d import Axes3D  #<- You need this to make 3D plots
fig = pyplot.figure()   # Create the figure object
ax=fig.gca(projection='3d')  # Set the axes to be 3 dimensional
\end{Verbatim}

Next, we need to understand how \code{meshgrid} turns one-dimensional arrays
\code{x} and \code{y} into two-dimensional arrays \code{X}
and \code{Y}, consider the following simple example. Suppose that
the arrays \code{x} and \code{y} are given by
\begin{Verbatim}
x=[1,2,3]
y=[4,5]
\end{Verbatim}
The command {\tt X,Y=meshgrid(x,y)} produces the following
results:\marginfig{Figures/figMeshGrid}{\label{figMeshGrid}}
\begin{equation}\label{eq:MeshgridMatrices}
X = \left[ \begin{array}{lll}
1 & 2 & 3 \\
1 & 2 & 3 \\
\end{array}
\right]~~~~~~~~ Y = \left[ \begin{array}{lll}
4 & 4 & 4 \\
5 & 5 & 5 \\
\end{array}
\right] .
\end{equation}
Compare these matrices to Fig.~\ref{figMeshGrid} to see how they
correspond to the small area of $x$-$y$ space that we are
representing.  Note that the $y$ dimension entries are flipped from
what you might expect because of the difference in conventions for
writing matrices (left-to-right and top-to-bottom) versus plotting
functions (left-to-right and bottom-to-top).

\index{Ndgrid} With \code{meshgrid}, the first index of both \code{X}
and \code{Y} is a $y$-index, since matrix elements are indexed using the
\code{X(row,column)} convention. For example, the elements \code{X(1,1)} and
\code{X(2,1)} both have value 1 because as the $y$-index changes, $x$ stays the
same. Similarly, \code{Y(1,1)=4} and \code{Y(2,1)=5} because as the
$y$-index changes, $y$ does change. This means that if we think of $x$
having an array index $i$ and $y$ having index $j$, then the two-dimensional
matrices have indices
\begin{equation}
    X(j,i)~~~~Y(j,i)  .
\end{equation}
But in physics we often think of two-dimensional functions of $x$ and $y$ in
the form $F(x,y)$, i.e., the first position is for $x$ and the second one is
for $y$. Because we think this way it would be nice if the matrix indices
worked this way too, meaning that if $x=x(i)$ and $y=y(j)$, then the
two-dimensional matrices would be indexed as
\begin{equation}
X(i,j)~~~~Y(i,j)
\end{equation}
instead of in the backwards order made by {\tt meshgrid}.

Numpy has a command called {\tt mgrid} which is similar to
\code{meshgrid} but does the conversion to two dimensions the other
way round. For instance, with the example arrays for $x$ and $y$ used
above {\tt X,Y=mgrid[1:4,4:6]} would produce the results
\begin{equation}
X = \left[ \begin{array}{ll}
1 & 1 \\
2 & 2 \\
3 & 3 \\
\end{array}
\right]~~~~~~~~ Y = \left[ \begin{array}{ll}
4 & 5 \\
4 & 5 \\
4 & 5 \\
\end{array}
\right]
\end{equation}
These matrices have the indexes in the $X(i,j)$, $Y(i,j)$ order,
but lose the spatial correlation that \code{meshgrid} gives
between Eq.~\eqref{eq:MeshgridMatrices} and Fig.~\ref{figMeshGrid}.

A third argument can be added to the \code{mgrid} command to indicate
the stepsize.  For example, try the following:
\begin{Verbatim}
from numpy import mgrid
X,Y=mgrid[1:4:0.5,4:6:0.5]
\end{Verbatim}
The $0.5$ that was added indicates that I want a mesh from one to four
in increments of $0.5$.  If the third argument is imaginary, like
this:
\begin{Verbatim}
from numpy import mgrid
X,Y=mgrid[1:4:10j,4:6:10j]
\end{Verbatim}
then the meaning changes to indicating the number of elements you'd
like in the range.  This is very similar to the difference between
\code{arange} and \code{linspace}.


Plots made either with \code{plot_surface(X,Y,F)} or \code{contour(X,Y,F)} (discussed below) will look
the same with either grid.  However, {\tt streamline} plots require your
data to be in the format provided by {\tt meshgrid}.  You will need to be
familiar with both methods of creating a grid. You will have need to do
it both ways, depending on the circumstance.



\section{Surface Plots}
\index{Plotting, contour and surface} \index{Surface plots}
\index{Contour plots}

Now that we understand how to make two-dimensional grids, let's make
some plots. Surface plots can be made with either the
\code{pyplot.plot_surface} command or the \code{pyplot.plot_wireframe}
command.  Run the following example to see what a wireframe plot looks
like. Make sure you read through the comments and watch what it does.

\begin{codeexample}
\begin{VerbatimOut}{\listingFile}
from numpy import arange,meshgrid,cos,exp,pi,mgrid
from matplotlib import pyplot
from mpl_toolkits.mplot3d import Axes3D  #<- You need this to make 3D plots

# Define the arrays x and y
# Don't make the step size too small or you will kill the
# system (you have a large, but finite amount of memory)
x=arange(-1,1,0.1)
y=arange(0,1.5,0.1)

# Use meshgrid to convert these 1-d arrays into 2-d matrices
# of x and y values over the plane
#X,Y=meshgrid(x,y)
X,Y=mgrid[-1:1:0.1,0:1.5:0.1]  # You could also do this in place of meshgrid
# Get F(x,y) by using F(X,Y). Note the use of .* with X and Y
# rather than with x and y
F=(2-cos(pi*X))*exp(Y)

# Plot the function
fig = pyplot.figure()
ax=fig.gca(projection='3d')
ax.plot_wireframe(X,Y,F)
pyplot.xlabel('x')
pyplot.ylabel('y')
pyplot.show()
\end{VerbatimOut}
\end{codeexample}


%To make this surface
%oscillate up and down like a manta ray you could do this.
%
%\begin{codeexample}
%\begin{VerbatimOut}{\listingFile}
%clear; close all;
%
%x=-1:.1:1;
%y=0:.1:1.5;
%[X,Y]=ndgrid(x,y);
%F=(2-cos(pi*X)).*exp(Y);
%
%dt=.1;
%
%for m=1:100
%   t=m*dt;
%   g=F*cos(t);
%
%   surf(X,Y,g);
%   AZ=30;EL=45;
%   view(AZ,EL);
%   title('Surface Plot')
%   xlabel('x')
%   ylabel('y')
%   axis([-1 1 -1 1 min(min(F)) max(max(F))])
%   pause(.1)
%end
%\end{VerbatimOut}
%\end{codeexample}

%\subsection*{Fourier Series}
%\index{Fourier series} \index{Physics 318}
%
%Python will make graphical displays of infinite series quickly and
%easily. Consider this solution for the electrostatic potential in a
%long tube of rectangular cross-section bounded by long metal strips
%held at different potentials. The tube has width $2b$ in $x$ and
%width $a$ in $y$. The two strips at $x=-b$ and $x=+b$ are held at
%potential $V_0$ while the strips at $y=0$ and $y=a$ are
%grounded.\footnote{See {\it Introduction to Electrodynamics, Third
%Edition} by David Griffiths, pages 132-134.} The electrostatic
%potential is given by
%\begin{equation}
%V(x,y) = \frac{4 V_0}{\pi} \sum_{n=0}^{\infty} \frac{1}{(2n+1)}
%\frac{\cosh{[(2n+1) \pi x/a]} }{ \cosh{[(2n+1) \pi b/a]} } \sin{[
%(2n+1) \pi y/a]} .
%\end{equation}
%Here is a piece of Python code that evaluates this function on an
%$xy$ grid and displays it as a surface plot.
%
%\begin{codeexample}
%\begin{VerbatimOut}{\listingFile}
%clear;close all;
%
%a=2;b=1;Vo=1;  % set some constants
%
%% build the x and y grids
%Nx=80;Ny=40;
%dx=2*b/Nx;dy=a/Ny;
%x=-b:dx:b;y=0:dy:a;
%
%[X,Y]=ndgrid(x,y); % build the 2-d grids for plotting
%
%Nterms=20; % set the number of terms to include
%
%V=zeros(Nx+1,Ny+1); % zero V out so we can add into it
%
%% add the terms of the sum into V
%for m=0:Nterms
%   V=V+cosh((2*m+1)*pi*X/a)/cosh((2*m+1)*pi*b/a).*sin((2*m+1)*pi*Y/a)/(2*m+1);
%end
%
%V=4*Vo/pi*V; % put on the multiplicative constant
%
%surf(X,Y,V) % surface plot the result
%xlabel('x');
%ylabel('y');
%zlabel('V(x,y)')
%\end{VerbatimOut}
%\end{codeexample}

\section{Contour Plots}

Contour plots can be generated in a very similar way as for surface
plots.  Instead of using \code{plot_surface} or \code{plot_wireframe},
the command \code{contour} is used.  Look at the example below.


\begin{codeexample}
\begin{VerbatimOut}{\listingFile}
from numpy import arange,meshgrid,cos,exp,pi,mgrid
from matplotlib import pyplot

# Define the arrays x and y
# Don't make the step size too small or you will kill the
# system (you have a large, but finite amount of memory)
x=arange(-1,1,0.05)
y=arange(0,1.5,0.05)

# Use meshgrid to convert these 1-d arrays into 2-d matrices
# of x and y values over the plane
X,Y=meshgrid(x,y)
#X,Y=mgrid[-1:1:0.1,0:1.5:0.1]  # You could also do this in place of meshgrid
# Get F(x,y) by using F(X,Y). Note the use of .* with X and Y
# rather than with x and y
F=(2-cos(pi*X))*exp(Y)

# Plot the function
pyplot.contour(X,Y,F)
pyplot.xlabel('x')
pyplot.ylabel('y')
pyplot.show()
\end{VerbatimOut}
\end{codeexample}

\section{Vector Field Plots}
\index{Vector Field Plots} \index{Quiver plots}

Python will plot vector fields for you with arrows. This is a good
way to visualize flows, electric fields, magnetic fields, etc. The
command that makes these plots is {\tt quiver} and the code below
illustrates its use in displaying the electric field of a line
charge and the magnetic field of a long wire.  Note that the vector
field components must be computed in Cartesian geometry.
\begin{codeexample}
\begin{VerbatimOut}{\listingFile}
from numpy import arange, meshgrid,sqrt,log
from matplotlib import pyplot
x = arange(-5.25,5.25,0.5) # define the x and y grids (avoid (0,0))
y = arange(-5.25,5.25,0.5) # define the x and y grids (avoid (0,0))

X,Y=meshgrid(x,y)

# Electric field of a long charged wire
Ex=X/(X**2+Y**2)
Ey=Y/(X**2+Y**2)

pyplot.quiver(X,Y,Ex,Ey) # make the field arrow plot
pyplot.title('E of a long charged wire')
pyplot.axis('equal')  # make the x and y axes be equally scaled

# Magnetic field of a long current-carrying wire
Bx=-Y/(X**2+Y**2)
By=X/(X**2+Y**2)

# make the field arrow plot
pyplot.figure()
pyplot.quiver(X,Y,Bx,By)
pyplot.axis('equal')  # make the x and y axes be equally scaled
pyplot.title('B of a long current-carrying wire')

# The big magnitude difference across the region makes most arrows too small
# to see.  This can be fixed by plotting unit vectors instead
# (losing all magnitude information, but keeping direction)
B=sqrt(Bx**2+By**2)
Ux=Bx/B
Uy=By/B

pyplot.figure()
pyplot.quiver(X,Y,Ux,Uy);
pyplot.axis('equal')  # make the x and y axes be equally scaled
pyplot.title('B(wire): unit vectors')

# Or, you can still see qualitative size information without such a big
# variation in arrow size by having the arrow length be logarithmic.
Bmin=B.min()
Bmax=B.max()
# s is the desired ratio between the longest arrow and the shortest one
s=2  # choose an arrow length ratio
k=(Bmax/Bmin)**(1/(s-1))
logsize=log(k*B/Bmin)
Lx=Ux*logsize
Ly=Uy*logsize

pyplot.figure()
pyplot.quiver(X,Y,Lx,Ly)
pyplot.axis('equal')  # make the x and y axes be equally scaled
pyplot.title('B(wire): logarithmic arrows')
pyplot.show()
\end{VerbatimOut}
\end{codeexample}
\index{Zoom in and out} There may be too much detail to really see
what's going on in some field plots. You can work around this
problem by clicking on the zoom icon on the tool bar and then using
the mouse to define the region you want to look at. Clicking on the
zoom-out icon, then clicking on the figure will take you back where
you came from. Or double-click on the figure will also take you
back.

\section{Streamlines}

In addition to plotting little arrows at each point in the vector field, you
can plot ``streamlines.'' For fluid dynamics, streamlines show the path that
a particle would follow if the arrows at each point represent the fluid
velocity at that point.  In electricity and magnetism, flow lines for the
electric and magnetic fields are the field lines that you learned about in
your introductory electricity and magnetism course.

As mentioned above, the streamline plotting routine assumes that the grid data
was created in the {\tt meshgrid} format rather than the {\tt mgrid} format.

\begin{codeexample}
\begin{VerbatimOut}{\listingFile}
from numpy import arange, mgrid,ones,array
from matplotlib import pyplot
x = arange(0,1,0.1)
y = arange(0,1,0.1)

#Define the position arrays x and y
X,Y = meshgrid(x,y)
#X,Y = mgrid[0:1:0.1,0:1:0.1]  #This won't work for streamplots.!!!

#Define the flow velocity arrays u and v
u = X
v = -Y

#Create a quiver plot of the flow velocity
pyplot.figure()
#pyplot.quiver(X,Y,u,v)

#Plot streamlines that start at different points along the line y=1.
startx = arange(0.05,0.95,0.05)
starty = ones(len(startx)) -0.2
sp = array([startx,starty])
pyplot.streamplot(x,y,u,v,start_points=sp.T,color='crimson',linewidth = 2)
pyplot.show()
\end{VerbatimOut}
\end{codeexample}

\subsection*{3-D Line Plots}
\index{Space curves} \index{Plot3} \index{Curves, 3-D} Python will
draw three-dimensional curves in space with the \code{plot} command.
However, there are a few additional lines of code that must be added
to make the plot 3 dimensional.  Here is how you would do a spiral on
the surface of a sphere using spherical coordinates.  I've
hightlighted the additions needed for the 3D plot

\begin{codeexample}
\begin{VerbatimOut}{\listingFile}
from matplotlib import pyplot
from mpl_toolkits.mplot3d import Axes3D  # <- Important
from numpy import pi, arange, sin, cos

dphi=pi/100 # set the spacing in azimuthal angle

N=30 # set the number of azimuthal trips
phi=arange(0,N * 2 * pi, dphi)

theta=phi/N/2 # go from north to south once

r=1  # sphere of radius 1

# convert spherical to Cartesian
x=r*sin(theta)*cos(phi)
y=r*sin(theta)*sin(phi)
z=r*cos(theta)

# plot the spiral
fig = pyplot.figure()
fig.gca(projection='3d')  # Create 3D axis
pyplot.plot(x,y,z)
pyplot.axis('equal')
pyplot.show()
\end{VerbatimOut}
\end{codeexample}


\section{Animations}
Often you will want to see how a plot evolves over time.  Maybe you
are plotting the waveform on a string and you want to see how it
changes in time.  Animations can be built by repeatedly constructing a
new plot and then waiting before you clear the canvas and plot again.
Look at the example below to see how to do this:


\begin{codeexample}
\begin{VerbatimOut}{\listingFile}
from numpy import arange,sin,cos
from matplotlib import pyplot

t = 0
x = arange(0,5,0.01)   #Domain over which I want to plot the function.
 while t < 10:
    y = sin(5 * x - 3 * t) * cos(2 * x)  #Construct array of function
                                         # values at current time.
    pyplot.clf()  # Clear the canvas, otherwise all plots end up on
                  # top of each other
    pyplot.plot(x,y,'r-',linewidth=3)
    pyplot.xlabel('x')
    pyplot.ylabel('f(x)')
    pyplot.title('t = {:.4f}'.format(t))
    pyplot.draw()           # draw the plot, but don't wait for
                            # someone to close the window.
    pyplot.pause(0.01)      # Wait before plotting the next one.
    t = t + .1              # Advance time.
\end{VerbatimOut}
\end{codeexample}

Here I am plotting snapshots of the function $\sin(5 x - 3 t) \cos(2
x)$ from $x = 0$ to $x=5$.  The animation runs from $t=0$ to $t=10$ s.
You'll want to make sure that you use \code{pyplot.draw()} instead of
\code{pyplot.show()} like you've done before.  If you use
\code{pyplot.show()} the animation will stop at each snapshot and
you'll have to close the figure window to allow the animation to
proceed.  Also notice the use of the \code{pyplot.clf()} command which
clears out any previous plots.  If you forget to do this, all of the
animation's plots will get put on top of each other.  And finally, the
\code{pyplot.pause(0.01)} command does exactly what it says: waits
before proceeding to the next plot. Changing the pause time will affect the speed of
the animation.
