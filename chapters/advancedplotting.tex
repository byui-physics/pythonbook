\chapter{Basic Plotting}
\label{chap:1DPlots}

Creating plots is an important task in science and engineering.  The
old addage ``A picture is worth a thousand words!'' is wrong.... it's
worth way more than that if you do it right.
\medskip

\section{Linear Plots}

\subsection*{Making a Grid}
\index{Plotting, xy} Simple plots of $y$ vs. $x$ are done using a
library called \texttt{matplotlib}.  First, arrays of $x$ and $y$
values must be constructed.  To build an array $x$ of $x$-values
starting at $x=0$, ending at $x=10$, and having a step size of
$.01$, you'll need numpy's \texttt{arange} function:
\begin{Verbatim}
from numpy import arange
x=arange(0,10,0.01)
\end{Verbatim}
To make a corresponding array of $y$ values according to the
function $y(x)=\sin(5x)$ simply type this
\begin{Verbatim}
from numpy import sin
y=sin(5*x)
\end{Verbatim}
\reminder{\lefthand}{Remember that you must use \ul{numpy's} \texttt{sin}
  function if you want to be able to pass an array of values to it.
  Using math's version will produce an error.}
Both of these arrays are the same length, as you can check with the
{\tt len} command
\begin{Verbatim}
len(x)
len(y)
\end{Verbatim}

\subsection*{Plotting the Function}

Once you have two arrays of the same size, you plot \texttt{y} vs.
\texttt{x} like this
\begin{Verbatim}
from matplotlib import pyplot
pyplot.figure()
pyplot.plot(x,y)
pyplot.show()
\end{Verbatim}
This will create a plot where the points are connected.  If you omit
the \texttt{pyplot.show()} command, the plot will not appear on your
screen.  You can save the plot to a file by replacing the
\texttt{pyplot.show()}  command with
\begin{Verbatim}
pyplot.savefig('filename.png')
\end{Verbatim}
If you want to see the actual data points being plotted, do this
instead.
\begin{Verbatim}
from matplotlib import pyplot
pyplot.scatter(x,y)
\end{Verbatim}
\index{Array, first or second half} And what if you want to plot part
of the \texttt{x} and \texttt{y} arrays? The {\tt colon} operator can
help.  Try the following code to plot the first and second half
separately:
\begin{Verbatim}
from matplotlib import pyplot
nhalf = len(x)/2
pyplot.plot(x(1:nhalf),y(1:nhalf))
pyplot.plot(x(nhalf:),y(nhalf:))
pyplot.show()
\end{Verbatim}

\subsection*{Controlling the Axes}
\label{sec:Axes}

\index{Axis Command} Python chooses the axes to fit the functions
that you are plotting. You can override this choice by specifying
your own axes, like this:
\begin{Verbatim}
pyplot.axis([0, 10, -1.3, 1.3])
\end{Verbatim}
\index{Xlim} \index{Ylim} Or, if you want to specify just the
$x$-range or the $y$-range, you can use {\tt xlim}:
\begin{Verbatim}
pyplotplot(x,y)
pyplot.xlim([0, 10])
\end{Verbatim}
or {\tt ylim}:
\begin{Verbatim}
pyplot.plot(x,y)
pyplot.ylim([-1.3, 1.3])
\end{Verbatim}
And if you want equally scaled axes, so that plots of circles are
perfectly round instead of elliptical, use
\begin{Verbatim}
pyplot.axis('equal')
\end{Verbatim}
\index{Axis equal} \index{Plot: equally scaled axes}


\subsection*{Logarithmic Plots}
\index{Log plots} \index{Plots, logarithmic} To make log and
semi-log plots use the matplotlib's commands {\tt semilogx}, {\tt semilogy}, and
{\tt loglog}. They work like this:
\begin{Verbatim}
from matplotlib import pyplot
from numpy import arange,exp
pyplot.clf()  # Clear the figure
x=arange(0,8,0.1)
y=exp(x)

pyplot.semilogx(x,y)  # Replace with semilogx or loglog
pyplot.title('semilogx')

pyplot.show()
\end{Verbatim}

\subsection*{3-D Line Plots}
\index{Space curves} \index{Plot3} \index{Curves, 3-D} Python will
draw three-dimensional curves in space with the {\tt plot} command.
However, there are a few additional lines of code that must be added
to make the plot 3 dimensional.  Here is how you would do a spiral on
the surface of a sphere using spherical coordinates.  I've
hightlighted the additions needed for the 3D plot
\marginpar{\captionsetup{type=table}
\begin{tabular}{ll}
'-'   &    solid line style     \\
'--'  &    dashed line style    \\
'-.'  &    dash-dot line style  \\
':'   &    dotted line style    \\
'.'   &    point marker         \\
','   &    pixel marker         \\
'o'   &    circle marker        \\
'v'   &    triangle down marker \\
'\^'   &    triangle up marker   \\
'<'   &    triangle left marker \\
'>'   &    triangle right marker\\
'1'   &    tri down marker      \\
'2'   &    tri up marker        \\
'3'   &    tri left marker      \\
'4'   &    tri right marker     \\
's'   &    square marker        \\
'p'   &    pentagon marker      \\
'*'   &    star marker          \\
'h'   &    hexagon1 marker      \\
'H'   &    hexagon2 marker      \\
'+'   &    plus marker          \\
'x'   &    x marker             \\
'D'   &    diamond marker       \\
'd'   &    thin diamond marker  \\
'|'   &    vline marker         \\
'\_'   &    hline marker         \\
\end{tabular}
\captionof{table}{Line styles and plot markers.\label{tab:linestyles}}
}
\begin{codeexample}
\begin{VerbatimOut}{\listingFile}
from matplotlib import pyplot
from mpl_toolkits.mplot3d import Axes3D  # <- Important
from numpy import pi, arange, sin, cos

dphi=pi/100 # set the spacing in azimuthal angle

N=30 # set the number of azimuthal trips
phi=arange(0,N * 2 * pi, dphi)

theta=phi/N/2 # go from north to south once

r=1  # sphere of radius 1

# convert spherical to Cartesian
x=r*sin(theta)*cos(phi)
y=r*sin(theta)*sin(phi)
z=r*cos(theta)

# plot the spiral
fig = pyplot.figure()
fig.gca(projection='3d')  # Create 3D axis
pyplot.plot(x,y,z)
pyplot.axis('equal')
pyplot.show()
\end{VerbatimOut}
\end{codeexample}

\section{Plot Appearance}

\subsection*{Line Color and Style}

To specify the color and line style of your plot, use the following syntax
\begin{Verbatim}
pyplot.plot(x,y,'r-')
\end{Verbatim}
The \verb|'r-'| option string tells the plot command to plot the
curve in red connecting the dots with a continuous line. Many other
colors and line styles are possible, and instead of connecting the
dots you can plot symbols at the points with various line styles
between the points. For instance, if you type
\begin{Verbatim}
pyplot.plot(x,y,'g.')
\end{Verbatim}
you get green dots at the data points with no connecting line.  Table
\ref{tab:linestyles} has a list of the possiblilities

%For on-screen viewing these standard plot appearance commands will
%usually be fine.  If you want to make publication quality figures
%you will have to work harder. See Chapter~\ref{chap:Plots} for more
%information.

\subsection*{Labeling your plots}
To label the $x$ and $y$ axes, do this after the {\tt plot} command:
\begin{Verbatim}
pyplot.xlabel('Distance (m)')
pyplot.ylabel('Amplitude (mm)')
\end{Verbatim}
And to put a title on you can do this:
\begin{Verbatim}
pyplot.title('Oscillations on a String')
\end{Verbatim}
\index{Lettering plots} \index{Text, on plots} To write on your
plot, you can use Python's {\tt text} command in the format:
\begin{Verbatim}
pyplot.text(10,.5,'Hi');
\end{Verbatim}
which will place the text ``Hi'' at position $x=10$ and $y=0.5$ on
your plot.

You can also build labels and titles that contain numbers you have
generated; just construct a formatted string and then use this string
variable as the argument of the commands {\tt xlabel}, {\tt ylabel},
and {\tt title}, like this:
\begin{Verbatim}
s = 'Oscillations with k={:.d}'.format(5)
pyplot.title(s)
\end{Verbatim}
In this example we hard-coded the number 5, but you can do the same
thing with variables.


\subsection*{Greek Letters, Subscripts, and Superscripts}

\marginpar{\captionsetup{type=table}
\begin{tabular}{ll}
$\alpha$   & $\backslash$alpha   \\
$\beta$    & $\backslash$beta    \\
$\gamma$   & $\backslash$gamma   \\
$\delta$   & $\backslash$delta   \\
$\epsilon$ & $\backslash$epsilon \\
$\phi$     & $\backslash$phi     \\
$\theta$   & $\backslash$theta   \\
$\kappa$   & $\backslash$kappa   \\
$\lambda$  & $\backslash$lambda  \\
$\mu$      & $\backslash$mu      \\
$\nu$      & $\backslash$nu      \\
$\pi$      & $\backslash$pi      \\
$\rho$     & $\backslash$rho     \\
$\sigma$   & $\backslash$sigma   \\
$\tau$     & $\backslash$tau     \\
$\xi$      & $\backslash$xi      \\
$\zeta$    & $\backslash$zeta    \\
\end{tabular}
\captionof{table}{The lowercase Greek letters in LaTeX\label{tab:Greek}}
}

\index{Greek letters} \index{LaTeX and Greek letters} When you put
labels and titles on your plots you can print Greek letters,
subscripts, and superscripts by using the LaTeX syntax. To print
Greek letters the string should be preceded with an 'r' and the math
should be enclosed in \texttt{\$}, like
this:
\begin{Verbatim}
pyplot.xlabel(r'$\theta$')
pyplot.ylabel(r'F($\theta)$')
\end{Verbatim}
And to put a title on you can do this:
\begin{Verbatim}
pyplot.title(r'F($\theta$)=$\sin$(5 $\theta$)')
\end{Verbatim}
You can use formatted strings to insert variables into your titles
\begin{Verbatim}
s=r'F($\theta$)=$\sin$({:d} $\theta$)'.format(5)
pyplot.title(s)
\end{Verbatim}
\index{LaTeX symbols in sprintf} \index{Sprintf, LaTeX symbols}
You can also print capital Greek letters, like this
$\backslash$Gamma, i.e., you just capitalize the first letter.

\index{Subscripts, superscripts} To put a subscript on a character
use the underscore character on the keyboard: $\theta_1$ is coded
by typing $\backslash${\tt theta\_1}. And if the subscript is more
than one character long do this: $\backslash${\tt theta\_\{12\}}
(makes $\theta_{12}$). Superscripts work the same way only using
the {\tt $\wedge$} character: use $\backslash${\tt
theta$\wedge$\{10\}} to print $\theta^{10}$.


\section{Multiple Plots}

\index{Multiple plots} \index{Figure windows} You may want to put
one graph in figure window 1, a second plot in figure window~2,
etc. To do so, put the Python command {\tt pyplot.figure()} before each plot
command, like this
\begin{Verbatim}
from numpy import arange, sin, cos
from matplotlib import pyplot

x=arange(0,20,0.01)
f1=sin(x)
f2=cos(x)/(1+x**2)
pyplot.figure()
pyplot.plot(x,f1)
pyplot.figure()
pyplot.plot(x,f2)
pyplot.show()
\end{Verbatim}

\subsection*{Overlaying Plots}
\index{Overlaid plots} \index{Hold on, off} Often you will want to
overlay two plots on the same set of axes.
Here's the first way---just ask for multiple plots on the
same plot line
\begin{Verbatim}
pyplot.plot(x,f1,'r-',x,f2,'b-')
pyplot.title('First way')
\end{Verbatim}
Here's the second way. After the first plot, tell Python
to hold the plot so you can put a second one with it
\begin{Verbatim}
pyplot.figure()
pyplot.plot(x,f1,'r-')
pyplot.plot(x,f2,'b-')
pyplot.title('Second way')
pyplot.show()
\end{Verbatim}
The second way is convenient if you have lots of plots to put on
the same figure.  

\subsection*{Subplots}

It is often helpful to put multiple plots in the same figure, but
on separate axes. The command to produce plots like this is {\tt
subplot}, and the syntax is \texttt{subplot(rows,columns,plot
number)} This command splits a single figure window into an array
of subwindows, the array having {\tt rows} rows and {\tt columns}
columns. The last argument tells Python which one of the windows
you are drawing in, numbered from plot number 1 in the upper left
corner to plot number {\tt rows*columns} in the lower right corner,
just as printed English is read. See online help for more
information on subplots.  For instance, to make two-row figure, do
this
\begin{Verbatim}
pyplot.subplot(2,1,1)
pyplot.plot(x,f1)
pyplot.subplot(2,1,2)
pyplot.plot(x,f2)
\end{Verbatim}



%\chapter{Plotting}
%\label{chap:Plotting}
%
%Creating plots is an important task in science and engineering.  The
%old addage ``A picture is worth a thousand words!'' is wrong.... it's
%worth way more than that if you do it right.  
%\section{Plotting Data}
%Python does not plot functions.  This is true of any computer
%software used to make plots.  Rather, the computer samples the
%function at a series of points in the domain and plots those points.
%In python, we can create a plot like this using a library called
%\texttt{matplotlib}.  A simple scatter plot can be generated like this
%\begin{Verbatim} 
%from matplotlib import pyplot  #Import necessary functions
%
%xData = [1.2,2.3,8.3,10.5]  # x-coordinates of the data points
%xData = [2.2,8.5,1.2,3.5]   # y-coordinates of the data points
%pyplot.scatter(xData,yData)  # make the plot
%pyplot.show()                # Display the plot to your screen.
%\end{Verbatim} 
%Here we have plotted some data and displayed them to the screen.
%\subsection*{Customizing Scatter Plots}
%You can easily modify the style,color, and size of the plot markers.
%The change the style of the plot marker, add the optional
%\texttt{marker} argrument to the scatter function, like this
%\begin{Verbatim} 
%pyplot.scatter(xData,yData,marker='+') # Use plus sign as the marker  
%\end{Verbatim} 
%A list of plot marker styles is given in table \ref{tab:styles}
%\marginpar{\footnotesize\captionsetup{type=table}
%  \vspace{0.5in}
%\begin{tabular}{ll}
%\texttt{'+'} & Plus \\
%\texttt{'.'} & Point \\
%\texttt{','} & Pixel \\
%\texttt{'v'} & Triangle down \\
%\texttt{'8'} & Octagon \\
%\texttt{'s'} & Square \\
%\texttt{'p'} & Pentagon \\
%\texttt{'x'} & X \\
%\end{tabular}
%\captionof{table}{Sampling of available plot markers.\label{tab:styles}}
%}
%
%
%\section{Plotting Functions}
%Plotting a function requires that you sample the function at some
%regular interval and then use those function samples to create a plot
%similar to the one we did above. You can then connect the plotted
%points using the \texttt{plot} function (also inside of
%\texttt{matplotlib}).
%\begin{Verbatim}
%from numpy import arange
%from math import sin
%from matplotlib import pyplot
%xData = arange(0,10,.01):
%yData = [sin(x) for x in xData] # Map the function onto my domain
%pyplot.plot(xData,yData)
%pyplot.show()
%\end{Verbatim}
%The \texttt{plot} function connects the data points being plotted.
%The \texttt{arange} function has been used to create a dense mesh of
%points to sample my function over.  Using the \texttt{range} function
%would probably not be a good choice here because it can't 
%Notice that \texttt{arange} takes three arguments: starting position,
%ending position, and step size.  You should take a look at the
%variable \texttt{xData} to see what you just created.  Also note the
%line that was used to generate the function values (\texttt{yData}).
%This is a compact way to combine a loop with a list and is worthwhile
%to learn.  Notice that I am evaluating \texttt{sin(x)} at each point
%in my list called \texttt{xData} and saving each number in a list
%(hence the brackets enclosing the whole things).  When
%I'm all done, I have a list of function values evaluated at the exact
%points in \texttt{xData}.\sidenote{Think about this until it makes
%  sense.  It will save you time.}
%
%\section{Histograms}
%Histograms are very useful plots when thinking about mean values and
%spread in data sets.  As with the other plots, the library
%\texttt{matplotlib} has the needed function to create this plot.
%Below is an example
%\begin{Verbatim}
%from matplotlib import pyplot
%measurements = [3.46,2.48,1.98,3.42,3.56,3.87,3.9,4.2,2.9]
%pyplot.hist(measurements)
%pyplot.show()
%\end{Verbatim}
%
%
%\section{Displaying multiple plots}
%Frequently you will want to display multiple plots on the same figure.
% To do this, simply execute multiple plot commands and wait until the
% end to use the \texttt{show} command, like this:
%\begin{Verbatim}
%from matplotlib import pyplot
%measurements = [3.46, 2.48, 1.98, 3.42, 3.56, 3.87, 3.9, 4.2, 2.9]
%data =         [2.44, 1.95, 2.64, 2.84, 1.83, 2.79, 3.5, 3.8, 1.9]
%x =            [1.24, 3.15, 4.62, 1.99, 2.01, 4.78, 2.1, 5.7, 3.2]
%
%pyplot.scatter(x,data)
%pyplot.scatter(x,measurements)
%pyplot.show()
%
%\end{Verbatim}
%Notice that the \texttt{pyplot.show()}  is only performed once after
%all of the plot objects have been created.
%
%If you want to display multiple plots on the same canvas, you have to
%use the \texttt{subplots} command, like this:
%\begin{Verbatim}
%f, (ax1, ax2) = pyplot.subplots( 2, sharex=True) 
%\end{Verbatim}
%
%This command creates a figure with 2 plots on it: one on top of the other.
%If you wanted the plots to be in a grid pattern, you could replace the
%\texttt{2} in the subplots command with \texttt{(2,2)} (for example).
%This would create a 2 x 2 grid of plots, a total of 4 plots.  Notice
%that the subplots command is returning 2 things: the figure, and the 2
%plots.  The figure is the canvas that the plots are being placed on.
%You can create the plot objects like this:
%\begin{Verbatim}
%from scipy.stats import norm  # Import norm to create some data to plot
%f, (ax1, ax2) = pyplot.subplots( 2, sharex=True) 
%data= norm.rvs(2.5,0.5,100)  # Create some data to plot
%ax1.hist(data,bins = 100)  # Create the first plot
%data= norm.rvs(2.5,0.5,1000)  # Create some more data to plot
%ax2.hist(data,bins = 100)  # Create the second plot
%
%pyplot.show()  #Show the whole thing.
%
%
%\end{Verbatim}


