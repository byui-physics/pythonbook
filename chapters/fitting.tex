\chapter{Fitting Functions to Data}
\label{chap:Fitting}

As a scientist(or engineer), you will frequently gather data.  Usually
when you gather data, the goal is to use that data to uncover some
relationship between the two variables.  In other words, you'd like to
find a function that best mimicks that data that you've gathered.  In
this chapter you'll see how to do that.

\section{Fitting to Linear Functions}
When you think of fitting a function to a data set you most likely
think of least-squares fitting.  Mathematically, this can be stated
as minimzing the following quantity:
\begin{equation}
\sum_{i=1}^N (y_i - f(x))^2
\end{equation}
Here, $y_i$ represents the data point collected at $x_i$ and $f(x)$ represents
the fitting function evaluated at $x_i$.  By varying the parameters in
the function until the above sum is minimized, you are finding the
function that best mimicks the data.

\section{Fitting to an Arbitrary Function}
The \texttt{scipy.optimizes} library contains a wealth of useful function, one
of which is called \texttt{curve\_fit}.  Sounds promising, right?
Let's use this function to fit to some made-up data.  Let's extract
some data from the function
\begin{equation}
y(x)  = a \sin(b x) \exp(c x)
\end{equation}
where $a = 3$, $b=2$ and $c= -1$.  We'll sample this function at the
following points
\begin{Verbatim}
[0.2,0.7,1,1.5,1.8,2,4,5,6,7,10,15]
\end{Verbatim}
and also add some random noise.  Here is the python code to do all of
that:
\begin{Verbatim}
from numpy import array,sin,exp  # Import needed math functions
from scipy.stats import norm     # Import normal distribution to add noise.

# Define the function
def func(x,a,b,c):
    value = a * sin(b * x) * exp(c * x)
    return value

#Define sample points
x = array([0.2,0.7,1,1.5,1.8,2,4,5,6,7,10,15])

a = 3
b = 2
c = -1
#         Sample the function............and add some random noise.
y = array([func(n,a,b,c)  for n in x]) + norm.rvs(0,.0069,len(x))
\end{Verbatim}
All of these things should be familiar to you if you have read through
previous chapters.  Notice the use of \texttt{norm.rvs} to sample a
normal distrubution.

Now we're ready to send our data into the function \texttt{curve\_fit}
perform the fitting. First, we need to import the library
\begin{Verbatim}
import scipy.optimize as opt

fit = opt.curve_fit(func,x,y)
\end{Verbatim}
The \texttt{curve\_fit} function requires three arguments: The function
that is being fitted, the independent variable values, and the
dependent variables values.  There are several optional arguments that
can be passed to this function.  For example, you can specify a guess
at the solution and that guess will serve as a starting
point.
\begin{Verbatim}
guess = [2.5,1.2,-3]  # Guess for a, b, and c.
fit = opt.curve_fit(func,x,y,p0=guess)
\end{Verbatim}
You can specify bounds on the search parameters:
\begin{Verbatim}
# Set bounds on a to (0,4)
# Set bounds on b to (0,3)
 # Set bounds on c to (-4,2)
paramBounds = [[0,0,-4],[4,3,2]] 
fit = opt.curve_fit(func,x,y,bounds=paramBounds)
\end{Verbatim}
and you can specify the uncertainty in the data points
\begin{Verbatim}
#Specify uncertainty on each data point
uncertainty = [0.01,0.02,0.1,0.2,0.05,0.2,0.5,0.01,0.02,0.01,0.1]  
fit = opt.curve_fit(func,x,y,sigma=uncertainty)
\end{Verbatim}

\section{Plotting the Fit}
The function \texttt{curve\_fit} will return several things.  The
first thing is the values of the fit parameters, which is the thing
you are most interested in.  Once you have those you can proceed to
plot the function just as we showed you in chapter \ref{chap:Plotting}.
Here I'll show you again (recall that the variable \texttt{fit}
contains the fit results)
\begin{Verbatim}
fitA = fit[0][0]  # Pull out value of a
fitB = fit[0][1]  # Pull out value of b
fitC = fit[0][2]  # Pull out value of c

xRange = arange(0,15,0.1)  # Define a grid of points on my domain

#Evaluate my fit function over the entire domain
#using my newly-found fit parameters
yVals = [func(n,fitA,fitB,fitC) for n in xRange] 
pyplot.scatter(x,y)  #Plot the data
pyplot.plot(xRange,yVals)  #Plot the fit function
pyplot.show()  #Show results.
\end{Verbatim}