\chapter{Variables and Data Types}
\label{chap:variablesDataTypes}


The simplest, and most fundamental object when programming a computer
is the variable, which is nothing more than a name which is given to a
piece of information. It allows the information to be saved (stored)
so that it can be recalled and used later.  \index{Data types} There
are different types of variables and each has its use and
limitations.  This chapter will focus on assigning,
manipulating, and using different variable types.


Python has many types of variables, but you will mostly
use the following types: integers, floats, lists, and arrays.  What
follows is a brief description of each variable type (we'll save our
discussion of arrays until later) with some examples to illustrate.

\section{Integer variables}
Some of you may be used to programming in C++\sidenote{C++ is a
  compiled language, whereas Python is an interpreted language}, where
the variables are declared before a value is assigned.  In Python,
variables are created and assigned in one statement using the
assignment operator (\code{=}).  The simplest type of data is an
integer, a decimal-less number.  An integer variable is created and
assigned a value like this:\index{Assigning values}
\begin{Verbatim}
a = 20
\end{Verbatim}
This statement creates the variable \code{a} and assigns\sidenote{Sometimes you may want your program to prompt the user to
enter a value and then save the value that the user inputs to a
varible.  This can be done like this:

\code{a = input("What is your age?")}

When you run your program, you will be prompted to enter your age.
When you do, that number will be saved to the variable \code{a}.} it a value of
$20$ (an \textbf{integer}).  The value of \code{a} can be modified in
any way you choose.  For instance, the statement
\begin{Verbatim}
a = a + 1
\end{Verbatim}
adds one to whatever value was previously stored in
\code{a}. \sidenote{An assign statement in programming is not a
  mathematical equation.  It makes perfect sense to write \code{a=a+1}
  as assign statement: it tells the computer to get the value of
  \code{a} and add 1 to it and then store the evaluated quantity back
  in the variable \code{a}.  However, the mathematical equation
  $a=a+1$ is clearly false.}  Note that {\it variable names in Python
  are case sensitive,} so watch your capitalization.  \index{Case
  sensitive}





You can define multiple variables by putting the assignements on
seperate lines:
\begin{Verbatim}
a = 2
b = 4
c = a * b
\end{Verbatim}
Here we have stored the value $2$ in \code{a}, the value $4$ in
\code{b} and then used those assignments to calculate a new value: $ 2
\times 4$ and assign that value to the variable \code{c}.
%\reminder{\lefthand}{If you are using version 3 or higher of Python,
%  printing a variable is done like this: \code{print(a)}.}
Other common mathematical operations can be performed on integer
variables.  For example:
\begin{Verbatim}
a = 20
b = 15
c = a + b  # add two numbers
d = a/b    # divide two numbers
r = a//b   # return only the quotient(an integer) of the division
r = a % b   # return only the remainder(an integer) of the division
e = a * b  # multiply two numbers together
f = c**4   # raise number to a power (use **, not ^)
\end{Verbatim}
Notice that performing an operation on two integers \ul{usually}
yields another integer.  This can pose a serious problem for division
if you think about it for a second. For example, what would be the
value of variable \code{d} above if the result has to be an integer?
Beginning in version 3 of Python, this problem was resolved so that
dividing two integers yields a float.

\section{Float variables}
Most meaningful calculations should be performed using floating point
numbers, or floats in Python.  Float variables are created and assigned
in
one of two ways.  The first way is to simply include a decimal place
in the number, like this
\begin{Verbatim}
a = 20.
\end{Verbatim}
You can also cast an integer variable to a float variable using the
\code{float} command
\begin{Verbatim}
a = 20
b = float(a)
\end{Verbatim}
You may wonder what would happen when a float and an integer are used
in a calculation, like this
\begin{Verbatim}
a = 0.1
b = 3 * a  #Integer multiplied by a float results in a float.
\end{Verbatim}
\marginpar{\footnotesize\captionsetup{type=table}
  \vspace{-2.5in}
\begin{tabular}{lp{1.05in}}
\code{abs(x)} & Find the absolute value of x\\ \\
\code{divmod(x,y)}        & Returns the quotient and remainder when
using long division.\\ \\
\code{x \% y}      & Return the remainder of $\frac{x}{y}$ \\ \\
\code{float(x)}      & convert \code{x} to a float. \\ \\
\code{int(x)}      & convert \code{x} to an integer. \\ \\
\code{round(x)}  & Round the number \code{x} using standard
rounding rules \\ \\
\end{tabular}
\captionof{table}{A sampling of built-in functions commonly used with
integers
  and floats.\label{tab:HousekeepingFloats}}
}
The result of such a calculation is always a float.  Only when
all of the numbers used in a calculation are integers is the result an
integer.
Floats can be entered using scientific notation like this
\begin{Verbatim}
1.23e15
\end{Verbatim}

\section{Boolean variables}
A boolean variable is one that stores one of two possible values: \code{True}
or \code{False}.  A boolean variable is created and assigned similar to the
other variables you've studied so far
\begin{Verbatim}
myVar = True
\end{Verbatim}
The purpose for using a boolean variable will become more clear as you
study loops and logical statements, so stay tuned.
\section{String Variables}
\index{Strings} String variables contain a sequence of characters, and
can be created and assigned using quotes, like this\sidenote{You may also enclose the characters in double quotes.}
\begin{Verbatim}
s='This is a string'
\end{Verbatim}

%If you need an apostrophe in your string, repeat a single quote,
%like this:
%\begin{Verbatim}
%t='Don''t worry'
%\end{Verbatim}
Some Python functions require options to be passed to them using
strings. Make sure you enclose them in quotes, as shown
above.  There are many useful functions that can be used with strings.
For example, strings can be concatenated (joined) together using the
\code{+} operator:
\begin{Verbatim}
a = 'Hello'
b = ', my name is B. Nelson'
c = a + b
\end{Verbatim}
The number of characters in a string can be calculated using the
\code{len} function.
\begin{Verbatim}
a = 'Hello'
b = ', my name is B. Nelson'
c = a + b
d = len(c)
\end{Verbatim}
Individual characters inside of a string may be accessed  by using
\code{[]} and placing the number of the character you want to access
inside of the brackets,like this:
\begin{Verbatim}
a = 'Hello, my name is B. Nelson'
a[0]
a[18]
\end{Verbatim}
Notice that strings are zero-indexed: the first character is the
zeroth element, the second is the 1st, etc.  If you want to extract a
substring, you can do it by placing multiple numbers in the brackets,
seperated by a\code{:}, like this:
\begin{Verbatim}
a = 'Hello, my name is B. Nelson'
a[2:10]
a[2:10:3]
\end{Verbatim}
The first statement here will extract the substring starting at the
third element (element 2 is the third character) and going up to the
tenth element\sidenote{Python does not include the last element.  \code{a[1:4]} includes \code{a[1]}, \code{a[2]}, and \code{a[3]} but not \code{a[4]}.}.  You can add a stepsize, like we've done in the second
statement.  In this case, the three means that it will extract every
third character starting at the third and ending at the tenth.
\marginpar{\footnotesize\captionsetup{type=table} \vspace{-5.5in}
\begin{tabular}{lp{1.05in}}
\code{a.join(b)}  & Join all elements in list \code{b} while
placing string \code{a} between each pair of elements.\\ \\
%\code{str.capitalize()}  & Capitalize all characters in string
%\code{str} \\ \\
\code{a.count(b)}  & Count the number of occurrences of
string \code{b} in string \code{a}\\ \\
\code{a.lower()}  & Convert upper case letters to lower case \\ \\
\code{a[x]}  &  Access element \code{x} in string \code{a} \\ \\
\code{a[x:y:z]}  &  Slice a string, starting at element
\code{x}, ending at element \code{y}, with a step size of \code{z}
\\ \\
\code{a + b}  &  Concatenate strings \code{a} and \code{b}.  \\ \\
%\code{a.replace('t','c')} & Replace every letter t in string \code{a} with the letter c.\\ \\
\end{tabular}
\captionof{table}{A sampling of ``housekeeping'' functions for
strings.\label{tab:HousekeepingStrings}}
}
% However, you'll have little luck modifying a single character in a
%string unless you first convert the string to a list, like this
%\begin{Verbatim}
%a = 'Hello, my name is B. Nelson'
%b = list(a)  # convert string a to a list
%b[18] = 'S'  # Modify the 18th element in the list.
%c = "".join(b)  # Join all the elements back together into one string.
%\end{Verbatim}
%The \code{join} function used here was probably unfamiliar to you.
%It is a built-in function for use with strings.  It joins together all
%of the elements in list \code{b} into one string, putting whatever
%is in "" in between
%each element.  There are many built-in functions that can be used with
%strings.
Some of the more commonly used functions for strings are shown in
Table \ref{tab:HousekeepingStrings}

\section{Lists}
It is very important that you fully grasp the concept of a list
variable.  They are heavily-used in mathematical and scientific
programming.  You can think of a list as a container that holds
multiple pieces of information. The list could hold integers, floats,
strings, or really just about anything.  Let's see how we might create
a list.

\subsection*{Creating Lists}
The easiest way to create  a list is by putting the list elements
inside of square brackets, like this:
\begin{Verbatim}
a = [5.6 , 2.1 , 3.4 , 2.9]
\end{Verbatim}
Note that square brackets (\code{[]}) must be used when creating the
list. If you accidentally use parenthesis (\code{()})\sidenote{Use
  parenthesis to create a tuple, which is just like a list but cannot
  be modified.}  or curly brackets (\code{\{\}})\sidenote{Use curly
  brackets to create a dictionary, which is like a list but can be
  indexed on any data type, not just integers} you'll end up creating
something other than a list.  If you want a list of integers evenly
spaced out over a given range, Python's \code{range} function can help:
\begin{Verbatim}
a = range(10,52,3)
\end{Verbatim}
This will construct a list of \ul{integers} starting at $10$, ending at
$51$, while stepping in increments of $3$. This function can also be
called with $1$ or $2$ arguments and default values will be assigned
to the missing ones.
\begin{Verbatim}
a = range(3,9)  # Creates a list starting at 3, ending a 8 stepping
                # in increments of 1 (default value)
b = range(10)   # Creates a list starting at 0(default), ending at 9
                # stepping
                # in increments of 1(default value)
\end{Verbatim}
\noindent You can make a list of anything.  For example, here is a
list of strings:
\begin{Verbatim}
a = ['Physics' , 'is' , 'so' , 'great']
\end{Verbatim}
The individual elements of any list need not be the same type of
data.  For instance, the following list is perfectly valid
\begin{Verbatim}
# Here is a list of strings and integers
a = ['Ben',90,'Chad',75,'Andrew',22]
\end{Verbatim}
You can even define a list of lists\sidenote{You should be careful about associating such a variable with a
mathematical matrix.  It's just a data container!}:
\begin{Verbatim}
a = [[4,3,2],[1,2.5,90],[4.2,2.9,10.5],[239.4,1.4],[2.27,98,234,16.2]]
\end{Verbatim}
\subsection*{Accessing and Slicing lists}
Accessing an element of a list (this will be done frequently so pay
attention!) can be accomplished using square brackets, like this
\begin{Verbatim}
a[0]  # Access the 1st element of array a
a[4]  # Access the 5th element of array a
a[-1] # Access the last element of array a
a[-2]  # Access the second to last element of array a
\end{Verbatim}
Take special note to the last two statements where a negative index is
used.  Using a negative index means that you are counting from the
back of the list forward.
Please note that Python lists are zero-indexed: the first element of
any list is 0, the second element is 1, etc.

Lists can be easily
modified by specifying which element you want to change and what you
want it changed to:
\begin{Verbatim}
a = ['Physics' , 'is' , 'so' , 'great']
a[3] = 'tough'  # Change the 4th element of a to "tough"
\end{Verbatim}
At times you may want to extract more than just a single element, but
not the entire list.  You can do this with the \code{:} operator, like
this
\begin{Verbatim}
aList = [4,5,10,1560,23,19]
#Try printing each of these to see what is in each slice
aList[1:]
aList[1:3]
aList[:3]
aList[1:4:2]
aList[5:2:-1]
\end{Verbatim}
There can be three numbers inside the brackets, each seperated by the
\code{:} symbol, like this \code{[x:y:z]}.  The section of the
list that is extracted starts at element \code{x}, ends at element
\code{y} (but does not include element \code{y}), while stepping in
increments of \code{z}.  Note that the last number is optional, and
omitting it will result in a default value of 1 for the step size.


When working with a list of list (also called a 2-dimensional list)
accessing an individual element requires two indices, like this
\begin{Verbatim}
a = [[4,3,2],[1,2.5,90],[4.2,2.9,10.5],[239.4,1.4],[2.27,98,234,16.2]]
c = a[3][1]
\end{Verbatim}
Here \code{[3][1]} means we are going to the 4th list and accessing the 2nd element. (remember: Python lists are zero-indexed)

\marginpar{\footnotesize\captionsetup{type=table}
  \vspace{-4.5in}
\begin{tabular}{lp{0.95in}}
\code{a[x]}        & Access element \code{x} in list \code{a}\\
\code{a[x:y:z]}        & Extract a slice of list \code{a}\\
\code{a.append(x)}        & Append \code{x} to list \code{a} \\
\code{a.pop()}      & Remove the last element of list \code{a}. \\
\code{len(a)}  & Find the number of elements in \code{a} \\
\code{range(x,y,z)}  & Create a list of integers, starting at x,
ending at y, and stepping in increments of z.\\
\code{a.insert(x,y)}    & Insert \code{y} at location
\code{x} in list \code{a} \\
\code{a.sort()}  & Sort list \code{a} from least to greatest. \\
\code{filter(f,x)}        & Filter list \code{x} using the
criteria function \code{f}(see lambda functions).\\
\code{a.reverse()}  & Reverse the order of list \code{a}. \\
\code{a.index(x)}  & Find the index where element x resides. \\
\code{a + b}  & Join list \code{a} to list \code{b} to form one
list. \\ \\
\code{max(a)}  & Find the largest element of \code{a} \\ \\
\code{min(a)}  & Find the smallest element of \code{a} \\ \\
\code{sum(a)}  & Returns the sum of the elements of \code{a} \\ \\
\end{tabular}
\captionof{table}{A sampling of ``housekeeping'' functions for
lists.\label{tab:HousekeepingList}}
}
\subsection*{Built-in Functions for Lists}
Python has many built-in functions that work on lists.  You've already
seem some of them.  Here are a few examples
\begin{Verbatim}
myList = range(5,25,2)  # Create list starting at 5, ending at 25,
                        # in increments of 2
len(myList)             # Find how many elements are in myList
myList.append(520)      # Add the number 520 to the end of myList
\end{Verbatim}
 Here, \code{range} will create a list \underline{of integers }
starting at $5$, ending at $25$ in increments of $2$.  The
\code{len} function will find the length of a list, and
\code{append} will add an element to the end of a list.  Table
\ref{tab:HousekeepingList} lists some of the more common built-in
functions/operations for lists.  Please note that this is not a
comprehensive list.
%A complete list of available function can be
%found in the appendix.


\section{Naming Variables}
So far in this book, we've named most of our variables using the first
few letters in the alphabet. (\code{a},\code{b},\code{c},...) In Python, you can assign a
variable any name you'd like, as long as you follow these two rules:
\begin{enumerate}
\item Variables must start with a letter or an underscore (\_).
\item The rest of your variable name can only consist of letters,
numbers, and underscores
\end{enumerate}
Here are a few examples of allowed names:
\begin{Verbatim}
susan = 72
susan2 = 21
This_is_allowed_but_you_would_never_want_a_name_this_long = 'Hello'
thisStyleIsCalledCamelCase = susan
\end{Verbatim}
Here are a few names that aren't allowed:
\begin{Verbatim}
2susan = 56
no spaces allowed = 'But you can put spaces in strings'
\end{Verbatim}


\section{Displaying Results}
It's great to calculate something useful, but not that helpful unless
you can see it.  Beginners often ask the question, ``I calculted XX
but Python didn't do anything.''  Actually, Python did exactly what
you told it:  It calculted XX and called it a day.  If you want to see
XX, {\em you have to print it} with the print statement:
\begin{Verbatim}
a = 5.3
b = [5,3,2.2]
c = 'Physics is fun'
print(a)
print(b)
print(c)
\end{Verbatim}
\marginpar{\footnotesize\captionsetup{type=table}
  \vspace{-2.5in}
\begin{tabular}{lp{1.05in}}
\code{\{\}} & Use the default format for the data type \\ \\
\code{\{:4d\}}  & Display integer with 4 spaces\\ \\
\code{\{:.4f\}}  & Display float with 4 numbers after the decimal\\ \\
\code{\{:8.4f\}}  & Display float with at least 8 total spaces and 4
numbers after the decimal\\ \\
\end{tabular}
\captionof{table}{Formatting strings available when
printing.\label{tab:HousekeepingFormat}}
}
 As you can see, you can \code{print} anything and Python will dump
it to screen as it pleases.  There are times when you may want to be
more careful about the formatting of your print statments.  For
example:
\begin{Verbatim}
a = 22
b = 3.5
print("Hi, I am Joe. I am {:d} years old and my GPA is:
{:5.2f}".format(a, b))

#This style also works
joe_string="Hi, I am Joe. My GPA is {:5.2f} and I am {:d} years old."
print(joe_string.format(b,a)

\end{Verbatim}
Notice the structure of this print statement: A string followed by the
\code{.} operator and the \code{format()} function. The variables to
be printed are provided as arguments to the \code{format} statement
and are inserted into the string sequentially wherever curly braces
(\code{\{\}}) are found.  The odd characters inside of the curly
braces are a format code: they indicate how you would like the
variable formatted when it is printed. The \code{:d}
indicates an integer variable and \code{:f} indicates a float.
Further specifications regarding spacing can also be made.  The
\code{5.2} in the float formatting indicates that I'd like the number
to be displayed with at least 5 digits and 2 numbers after the
decimal.  A selection of what available format statements is given in table
\ref{tab:HousekeepingFormat}


