\chapter{Running Python}
\label{chap:RunningPython}

\marginfig{Figures/pythonCW.jpg}{Canopy's environment for running
  Python code.}

Python is a computer programming language (don't freak out) with broad
applicability in science and engineering.  For those that are brand
new to computer programming, Python is simply a way to communicate a
set of instructions to your computer.  You'll quickly learn that your
computer is great at doing exactly what you tell it to do.  If you
find that your computer isn't doing what you think it should, it is
not because the machine is malfunctioning, rather you just probably
don't fully understand what you are telling it to do.

There are two ways to run python. One is at the command line and the
other is through an IDE (integrated development environment). If you
don't know what the command line is, I recommend you start out using
an IDE. A good IDE for python is called Canopy, which can
be downloaded \href{https://store.enthought.com/downloads/}{here}. If
you know what the command line is then you probably don't need any
help getting started. Feel free to use whatever editor you'd
like. Once you have downloaded and installed Canopy, launch the
editor. You should see a window that looks similar to the one
displayed in the margin. As you read this book, type the commands that
appear in \texttt{this font} in the editor window and run the code by
pushing the green arrow.

When writing a computer program, there are several big ideas, or main
pillars, that you should grasp to be successful.  Two of those pillars
are variables and functions. This chapter will focus on helping you
get comfortable with these ideas.  Later chapters will focus on other
main ideas and some more specific tasks that can be accomplished using
these foundational ideas.
\section{It's a Calculator}
Simple math can be performed pretty much just as you would
expect. Here are a few examples
\begin{Verbatim}
print(1+2)
print(5.0/6.0)
print(5**6)  # 5 raised to the power 6
print(678 * (3.5 + 2.8)**3.)
\end{Verbatim}
Note that typing the calculation alone, without the \texttt{print}
statement, will not produce any output to the screen.  Even though the
calculation has been performed, you will not see the result unless you
\texttt{print} it. Throughout this book, the \texttt{print}
statement will usually be omitted in code examples to maintain
brevity.  You should always print your result so you can see what you
have done.

%You may be wondering about more complex mathematical calculations like
%$sin(\frac{\pi}{2})$ or $e^{2.5}$.  Can Python evaluate them
%too?\sidenote{One would sincerely hope so, otherwise it would be of
%  little use as a scientific tool} We'll save that discussion until
%we explain functions and libraries.

\section{Variables}
The simplest, and most fundamental object when programming a computer
is the variable, which is nothing more than a name which is given to a
piece of information. It allows the information to be saved (stored)
so that it can be recalled and used later.  \index{Data types} There
are different types of variables and each has it's use and
limitations.  Python has many types of variables, but you will mostly
use the following types: integer, float, and list.  What follows is a
brief description of each variable type with some examples to
illustrate.

\subsection*{Integer variables}
Some of you may be used to programming in C++\sidenote{C++ is a
  compiled language, whereas Python is an interpreted language}, where
the variables are declared before a value is assigned.  In python,
variables are created and assigned in one statement using the
assignment operator (\texttt{=}).  The simplest type of data is an
integer, a decimal-less number.  An integer variable is created and
assigned a value like this:\index{Assigning values}
\begin{Verbatim}
a = 20
\end{Verbatim}
This statement creates the variable {\tt a} and assigns it a value of
$20$ (an \textbf{integer}).  The value of \texttt{a} can be modified in
any way you choose.  For instance, the statement
\begin{Verbatim}
a = a + 1
\end{Verbatim}
adds one to whatever value was previously stored in
\texttt{a}. \sidenote{An assign statement in programming is not a
  mathematical equation.  It makes perfect sense to write
  \texttt{a=a+1} as assign statement: it tells the computer to get the
  value of \texttt{a} and add 1 to it and then store the evaluated
  quantity back in the variable \texttt{a}.  However, the mathematical
  equation $a=a+1$ is clearly false.}  Note that {\it variable names
  in Python are case sensitive,} so watch your capitalization.
\index{Case sensitive} If you want to see the value of a variable, you
must use the \texttt{print} command, like this
\begin{Verbatim}
print a
\end{Verbatim}
\sidenote{If you are using version 3 or higher of python,
  printing a variable is done like this: \texttt{print(a)}.}
%\reminder{\lefthand}{If you are using version 3 or higher of python,
%  printing a variable is done like this: \texttt{print(a)}.}
Other common mathematical operations can be performed on integer
variables.\sidenote{For more complex mathematical calculation, wait
  until we discuss functions.}  For example:
\begin{Verbatim}
a = 20
b = 15
c = a + b  # add two numbers
d = a/b    # divide two numbers  
r = a//b   # return only the quotient(an integer) of the division
r = a % b   # return only the remainder(an integer) of the division
e = a * b  # multiply two numbers together
f = c**4   # raise number to a power (use **, not ^)
\end{Verbatim}
Notice that performing an operation on two integers \ul{usually}
yields another integer.  The one exception is division if you are
using Python version 2 or earlier.  For
example, what would be the value of \texttt{d} above if the result has
to be an integer? 

\subsection*{Float variables}
Most meaningful calculations should be performed using floating point
numbers, or floats in Python.  Float variables are created and assigned in
one of two ways.  The first way is to simply include a decimal place
in the number, like this
\begin{Verbatim}
a = 20.
\end{Verbatim}
You can also cast an integer variable to a float variable using the
\texttt{float} command
\begin{Verbatim}
a = 20
b = float(a)
\end{Verbatim}
\begin{enumerate}
\prob Check that the calculations that you performed above using integers
yield different answers when floats are used instead
\end{enumerate}
You may wonder what would happen when a float is multiplied by an
integer, like this
\begin{Verbatim}
a = 0.1
b = 3 * a  #Integer multiplied by a float results in a float.
\end{Verbatim}
\marginpar{\footnotesize\captionsetup{type=table}
  \vspace{-2.5in}
\begin{tabular}{lp{1.05in}}
\texttt{abs(x)}        & Find the absolute value of x\\ \\
\texttt{divmod(x,y)}        & Returns the quotient and remainder when
using long division.\\ \\
\texttt{x \% y}      & Return the remainder of $\frac{x}{y}$ \\ \\
\texttt{float(x)}      & convert \texttt{x} to a float. \\ \\
\texttt{int(x)}      & convert \texttt{x} to an integer. \\ \\
\texttt{round(x)}  & Round the number \texttt{x} using standard
rounding rules \\ \\
\end{tabular}
\captionof{table}{A sampling of built-in functions commonly used with integers
  and floats.\label{tab:HousekeepingFloats}}
} .  The result of such a calculation is always a float.  Only when
all of the numbers used in a calculation are integers is the result an
integer.  
Numbers can be entered using scientific notation like this
\begin{Verbatim}
1.23e15
\end{Verbatim}
Python floats are stored as a 53-digit, base-2 binary number(that's a
mouthful).  If you're interested in what that means, we can talk more.
If you're not that interested, just know that when you define a float
in python, the number that is stored in the computer is not
\textbf{exactly} the number that you think it is.  This can cause
problems when you are comparing two numbers that you think should be
equal but actually aren't equal in the computer. \sidenote{There is a
  library called \texttt{Decimal} that will fix a lot of these
  problems.}  Try the following to get a feel for this
\begin{Verbatim}
a = 0.1
b = 3 * a
print(b)
print(" {:.45f} ".format(b))  #Formatted print: you haven't learned
                              #about this yet
\end{Verbatim}
The first print statement displays the value of \texttt{b} to one
decimal place.  The second print statement is called a formatted print
and you'll learn about it a little later.  In this case I am forcing
Python to display the value of \texttt{b} out to 45 decimal places.
Notice that the true value of \texttt{b} is not \ul{exactly} equal to
0.3.  You'll want to keep this fact in mind when you are comparing two
numbers that you think are equal.  When we study logical statements
this should become more clear.

\subsection*{Boolean variables}
A boolean variable is one that stores one of two possible values: True
or False.  A boolean variable is created and assigned similar to the
other variables you've studied so far
\begin{Verbatim}
myVar = True
\end{Verbatim}
The purpose for using a boolean variable will become more clear as you
study loops and logical statements, so stay tuned.

\subsection*{Lists}
Lists are heavily-used data types in mathematical and
scientific programming.  You can think of a list as a
container that holds multiple pieces of information. The list could
hold integers, floats, strings, or really just about anything. A list
is created and assigned in a manner similar to the way floats and
integers are, like this:
\begin{Verbatim}
a = [5.6 , 2.1 , 3.4 , 2.9]
\end{Verbatim}
Note the use of brackets (\texttt{[]}) when creating the list. If you
accidentally use parenthesis (\texttt{()})\sidenote{Use parenthesis to
  create a tuple, which is just like a list but cannot be modified.}
or curly brackets (\texttt{\{\}})\sidenote{Use curly brackets to
  create a dictionary, which is like a list but can be indexed on any
  data type, not just integers} you'll end up creating something other
than a list.

\noindent Python has a built-in function called \texttt{range} that can be used
to construct a list of integers, like this
\begin{Verbatim}
a = range(10,52,3)  
\end{Verbatim}
This will construct a list of \ul{integers} starting at $10$, ending at
$52$, while stepping in increments of $3$. This function can also be
called with $1$ or $2$ arguments and default values will be assigned
to the missing ones.  
\begin{Verbatim}
a = range(3,9)  # Creates a list starting at 3, ending a 9 stepping
                # in increments of 1 (default value)
b = range(10)   # Creates a list starting at 0(default), ending at 10 stepping
                # in increments of 1(default value)
\end{Verbatim}
\noindent You can make a list of anything.  For example, here is a
list of strings:
\begin{Verbatim}
a = ['Physics' , 'is' , 'so' , 'great']
\end{Verbatim}
The individual elements of any list need not be the same type of
data.  For instance, the following list is perfectly valid
\begin{Verbatim}
# Here is a list of strings and integers
a = ['Ben',90,'Chad',75,'Andrew',22]  
\end{Verbatim}
You can even define a list of lists:
\begin{Verbatim}
a = [[4,3,2],[1,2.5,90],[4.2,2.9,10.5],[239.4,1.4],[2.27,98,234,16.2]]
\end{Verbatim}
though you should not think of this as a matrix and try to perform
matrix math on it. To do that sort of thing, we'll need to learn about
arrays.
\subsection*{Accessing and Slicing lists}
Accessing an element of a list (this will be done frequently so pay
attention!) can be accomplished using square brackets, like this
\begin{Verbatim}
a[0]  # Access the 1st element of array a
a[4]  # Access the 5th element of array a
a[-1] # Access the last element of array a
a[-2]  # Access the second to last element of array a
\end{Verbatim}
Take special note to the last two statements where a negative index is
used.  Using a negative index means that you are counting from the
back of the list forward.
Please note that python lists are zero-indexed: the first element of
any list is 0, the second element is 1, etc.  Lists can be easily
modified by specifying which element you want to change and what you
want it changed to:
\begin{Verbatim}
a = ['Physics' , 'is' , 'so' , 'great']
a[3] = 'tough'  # Change the 4th element of a to "tough"
\end{Verbatim}
At times you may want to access entire sections of list, though not
the entire list.  You can do this with the \texttt{:} operator, like
this
\begin{Verbatim}
aList = [4,5,10,1560,23,19]
aList[1:]  
aList[1:3] 
aList[:3]  
aList[1:4:2]
\end{Verbatim}
There can be three numbers inside the brackets, each seperated by the
\texttt{:} symbol, like this \texttt{[x:y:z]}.  The section of the
list that is extracted starts at element \texttt{x}, ends at element
\texttt{y} (but does not include element \texttt{y}), while stepping in
increments of \texttt{z}.  Note that the last number is optional, and
omitting it will result in a default value of 1 for the step size.
 \marginpar{\footnotesize\captionsetup{type=table}
  \vspace{-4.5in}
\begin{tabular}{lp{1.05in}}
\texttt{a[x]}        & Access element \texttt{x} in list \texttt{a}\\
\texttt{a[x:y:z]}        & Extract a slice of list \texttt{a}\\ 
\texttt{a.append(x)}        & Append \texttt{x} to list \texttt{a} \\
\texttt{a.pop()}      & Remove the last element of list \texttt{a}. \\
\texttt{len(a)}  & Find the number of elements in \texttt{a} \\
\texttt{range(x,y,z)}  & Create a list of integers, starting at x,
ending at y, and stepping in increments of z.\\
\texttt{a.insert(x,y)}    & Insert \texttt{y} at location
\texttt{x} in list \texttt{a} \\
\texttt{a.sort()}  & Sort list \texttt{a} from least to greatest. \\
\texttt{filter(f,x)}        & Filter list \texttt{x} using the
criteria function \texttt{f}(see lambda functions).\\
\texttt{a.reverse()}  & Reverse the order of list \texttt{a}. \\
\texttt{a.index(x)}  & Find the index where element x resides. \\
\texttt{a + b}  & Join list \texttt{a} to list \texttt{b} to form one list. \\ \\
\texttt{max(a)}  & Find the largest element of \texttt{a} \\ \\
\texttt{min(a)}  & Find the smallest element of \texttt{a} \\ \\
\texttt{sum(a)}  & Returns the sum of the elements of \texttt{a} \\ \\
\end{tabular}
\captionof{table}{A sampling of ``housekeeping'' functions for lists.\label{tab:HousekeepingList}}
}

When working with a list of list (also called a 2-dimensional list)
accessing an individual element requires two indices, like this
\begin{Verbatim}
a = [[4,3,2],[1,2.5,90],[4.2,2.9,10.5],[239.4,1.4],[2.27,98,234,16.2]]
c = a[3][1]  
\end{Verbatim}
Here \texttt{[3][1]} means we are accessing the 2nd element of the 4th
list. (remember: Python lists are zero-indexed) You may also think of
this as the 4th row and 2nd column of the matrix.  Warning: Don't
conceptualize 2-d lists as matrices and try to slice out a
sub-matrix.  For example, try this
\begin{Verbatim}
a = [[4,3,2],[1,2.5,90],[4.2,2.9,10.5],[239.4,1.4],[2.27,98,234,16.2]]
c = a[0:3][0:2]  
\end{Verbatim}
Slicing a 2-d list like this is rather redundant.  First you slice out
the first three elements of the main list, resulting in a list of 3
lists.  The second slice is performed on the result of the first
slice giving a list of 2 lists.  The same result could have been
extracted with the following slice:
\begin{Verbatim}
a = [[4,3,2],[1,2.5,90],[4.2,2.9,10.5],[239.4,1.4],[2.27,98,234,16.2]]
c = a[0:2]
\end{Verbatim}
The take home message here is to just not ever think of lists as
mathematical matrices.  Just think of them as data containers.
\subsection*{Built-in Functions for Lists}
Python has many built-in functions that work on lists.  You've already
seem some of them.  Here are a few examples
\begin{Verbatim}
myList = range(5,25,2)  # Create list starting at 5, ending at 25,
                        # in increments of 2  
len(myList)             # Find how many elements are in myList
myList.append(520)      # Add the number 520 to the end of myList
\end{Verbatim}
 Here, \texttt{range} will create a list \underline{of integers }
starting at $5$, ending at $25$ in increments of $2$.  The
\texttt{len} function will find the length of a list, and
\texttt{append} will add an element to the end of a list.  Table
\ref{tab:HousekeepingList} lists some of the more common built-in
functions/operations for lists.  Please note that this is not a
comprehensive list.  A complete list of available function can be
found in the appendix.


\subsection*{Vector and Matrix Math: A warning}
 You may feel tempted to associate a
list with a mathematical vector and try to perform vector math on
them. We've mentioned this warning before but just want to further emphasis
what list cannot be used to do. For instance, it may seem natural to try
\begin{Verbatim}
a = [5.1 , 3.2 , 6.8 , 9.2]
b = 5 * a
\end{Verbatim}
or 
\begin{Verbatim}
a = [5.1 , 3.2 , 6.8 , 9.2]
b = [2.7 , 1.9 , 3.2 , 9.9]
c = a + b
\end{Verbatim}
\begin{enumerate}
\prob  Try the above operations and explain the results.
\end{enumerate}
Doing vector or matrix math (any math that involves an operation on
lists of numbers, not just single numbers) \underline{cannot be done
  with a list}.  If you were allowed to do that, what would happen
when your lists were filled with non-numerical data.  This doesn't
mean that the numbers \ul{stored in lists} can't be extracted and used in
mathematical calculations, like this 
\begin{Verbatim}
a = [4.5,8,2.1,10.8,12]
c = a[0]**a[1]  # Take the first element of a and raise it to a power
                # equal to the second element of a
\end{Verbatim}
It just means that mathematical calculations involving entire arrays
of numbers (like vector and matrix math) cannot be done using the list
data type.  However, don't think for one second that Python is unable
to handle this kind of math.  Just keep reading and you'll learn how
this is done.  
\subsection*{String Variables}
\index{Strings} String variables contain a sequence of characters, and
can be created and assigned using quotes, like this
\begin{Verbatim}
s='This is a string'
\end{Verbatim}
\sidenote{You may also enclose the characters in double quotes. }
%If you need an apostrophe in your string, repeat a single quote,
%like this:
%\begin{Verbatim}
%t='Don''t worry'
%\end{Verbatim}
Some Python functions require options to be passed to them using
strings. Make sure you enclose them in quotes, as shown
above.  There are many useful function that can be used with strings.
For example, strings can be concatenated (joined) together using the
\texttt{+} operator:
\begin{Verbatim}
a = 'Hello'
b = ', my name is B. Nelson'
c = a + b
\end{Verbatim}
The number of characters in a string can be calculated using the
\texttt{len} function.
\begin{Verbatim}
a = 'Hello'
b = ', my name is B. Nelson'
c = a + b
d = len(c)
\end{Verbatim}
Individual characters inside of a string may be accessed  and sliced
in the same way that elements of a list are accessed and sliced.
\begin{Verbatim}
a = 'Hello, my name is B. Nelson'
a[18]
a[2:15]
\end{Verbatim}
\marginpar{\footnotesize\captionsetup{type=table}
  \vspace{-2.5in}
\begin{tabular}{lp{1.05in}}
\texttt{a.join(b)}  & Join all elements in list \texttt{b} while
placing string \texttt{a} between each pair of elements.\\ \\
%\texttt{str.capitalize()}  & Capitalize all characters in string
%\texttt{str} \\ \\
\texttt{a.count(b)}  & Count the number of occurrences of
string \texttt{b} in string \texttt{a}\\ \\
\texttt{a.lower()}  & Convert upper case letters to lower case \\ \\
\texttt{a[x]}  &  Access element \texttt{x} in string \texttt{a} \\ \\
\texttt{a[x:y:z]}  &  Slice a string, starting at element
\texttt{x}, ending at element \texttt{y}, with a step size of \texttt{z} \\ \\
\texttt{a + b}  &  Concatenate strings \texttt{a} and \texttt{b}.  \\ \\
\end{tabular}
\captionof{table}{A sampling of ``housekeeping'' functions for strings.\label{tab:HousekeepingStrings}}
}
 However, you'll have little luck modifying a single character in a
string unless you first convert the string to a list, like this
\begin{Verbatim}
a = 'Hello, my name is B. Nelson'
b = list(a)  # convert string \texttt{a} to a list
b[18] = 'S'  # Modify the 18th element in the list.
c = "".join(b)  # Join all the elements back together into one string.
\end{Verbatim}
The \texttt{join} function used here was probably unfamiliar to you.
It is a built-in function for use with strings.  It joins together all
of the elements in list \texttt{b} into one string, putting whatever
is in "" in between
each element.  There are many built-in functions that can be used with
strings.  Some of the more commonly used ones are shown in Table \ref{tab:HousekeepingStrings}
\section{Displaying Results}
It's great to calculate something useful, but not that helpful unless
you can see it.  Beginners often ask the question, ``I calculted XX
but Python didn't do anything.''  Actually, Python did exactly what
you told it:  It calculted XX and called it a day.  If you want to see
XX, you have to print it.   The fastest way to see something, as
you've already seen is using the print statement:
\begin{Verbatim}
a = 5.3
b = [5,3,2.2]
c = 'Physics is fun'
print(a)
print(b)
print(c)
\end{Verbatim}
As you can see, you can \texttt{print} anything and Python will dump
it to screen as it pleases.  There are times when you may want to be
more careful about the formatting of your print statments.  For
example:
\begin{Verbatim}
a = 22
b = 3.5
print("Hi, I am Joe. I am {:d} years old and my GPA is: {:5.2f}".format(a, b))
\end{Verbatim}
Notice the structure of this print statement: A string followed by the
\texttt{.} operator and the \texttt{format()} function. The variables to
be printed are provided as arguments to the \texttt{format} statement
and are inserted into the string sequentially wherever curly braces
(\texttt{\{\}}) are found.
 The odd characters inside of the curly braces indicate how you would
 like the variable formatted when it is printed. The \texttt{:d} is
 used to indicate an integer variable and \texttt{:f} is used for
 floats.  Further specifications regarding spacing can also be made.
 The \texttt{5.2} in the float formatting indicates that I'd like the
 number to be displayed with at least 5 digits and 2 numbers after the
 decimal.  A
 summary of what is available is given in table \ref{tab:HousekeepingFormat}

\marginpar{\footnotesize\captionsetup{type=table}
  \vspace{-2.5in}
\begin{tabular}{lp{1.05in}}
\texttt{\{:4d\}}  & Display integer with 4 spaces\\ \\
\texttt{\{:.4f\}}  & Display float with 4 numbers after the decimal\\ \\
\texttt{\{:8.4f\}}  & Display float with at leasat 8 total spaces and 4 numbers after the decimal\\ \\
\end{tabular}
\captionof{table}{Formatting strings available when printing.\label{tab:HousekeepingFormat}}
}

\section{Functions}
A function is simply a set of instructions that will be performed when
called upon.  The function can be as long and complex or short and
succinct as you wish.  You can think of a function as a black box. You
create the contents of the box, specify what information needs to
enter the box for it to be able to accomplish it's task, and what
information will exit the box.  Why a black box?  Well, one benefit of
functions is that the user of it doesn't need to know what's inside.
She only needs to know what information the box needs and what
information the box will give back to her.

\subsection*{User-defined functions}
At times it will be beneficial for you to write a function that you
have designed.  You (the programmer) can create your own function like
this
\begin{Verbatim}
def myFunction(a,b):
    c = a + b
    d = 3.0 * c
    f = 5.0 * d**4
    return f
\end{Verbatim}
This function performs several simple calculations and then uses the
\texttt{return} statement to pass the final result back out of the
function.(what exits the black box) Every user-defined function must
begin with the keyword \texttt{def} followed by the function name (you
can choose it). The variables \texttt{a} and \texttt{b} are the
arguments to the function. (the things that must enter the black box)
Python does not use an \texttt{end} statement or anything like it to
signal the end of a function.  Instead, it looks for indentation to
determine where the function ends.  

The function can be called like this
\begin{Verbatim}
r = 10
t = 15
result = myFunction(r,t)
\end{Verbatim}
In this case, when the function is called, \texttt{a} gets assigned
the value of $10$ and \texttt{b} gets assigned the value of $15$.  The
result of this calculation is stored in the variable \texttt{result},
which is outside the function.  
\subsection*{Lambda (unnamed) functions}
The user-defined function above had 4 lines in it.  Other functions
could be even longer.  Sometimes, however, the function you need to
define is quite simple (maybe even one line) and it'd be nice if you
could define it in one line.  Luckily, you can.  It's
called a lambda function and here is an example:
\begin{Verbatim}
f = lambda x: x**3  #Define the function
print f(5)   # Evaluate the function
g = lambda x,y: x + y # Function of two variables
pritn g(5,6)
\end{Verbatim}
Here we defined the function $f(x) = x^3$, a function of one variable,
and quickly evaluated it.  Your lambda functions can have as many
arguments as you'd like: just seperate the arguments with commas.  At
this point, you may be asking yourself why this is such a big deal:
Why would you ever really need to do this.  Well, there are some
python \underline{functions that take functions as their
  arguments}.\sidenote{Stop and process that for a minute.}(as opposed
to taking simple numbers or strings) An example of this is the
\texttt{filter} function which serves to extract elements of a list
according to some criteria.  The exact details of the criteria are
specified using a lambda function.  Here's an example:
\begin{Verbatim}
a = [1,3,4,6,9,2,3,8]
b = list( filter( lambda x: (x % 2 == 0) , a ) )
print b
\end{Verbatim}
Notice how the \texttt{filter} function is used.  The first argument
is the criteria function: the function the dictates which of the list
elements you want extracted. It's not an integer or a float or a
sting.  It's actually a function: something that takes an argument and
returns a result.  In this case, the filter function was being used to
extract the elements of \texttt{a} that were multiples of $2$ and the
lambda function serves to check each number in the list and return
either \texttt{True} or \texttt{False}.

\subsection*{Imported Functions and Libraries}
Thus far you have performed very simple mathematical
calculations. ($5/6, 8^4$, etc..)
However, you were probably left wondering about more complex
mathematical calculations, like $\sin(\frac{\pi}{2})$ or $e^{2.5}$.
 
Many times, the function that you need is available already. Somebody
else has already created the function and made it available to anyone
who wants it.  Groups of functions that perform similar tasks are usually
bundled together into libraries. These libaries can then be imported
and the functions that they contain can be used.  Just as with
user-defined functions, it is critical that you know what
information(variables) the function expects you to give it and what
useful information the function will hand back to you.  This
information can be found in the library's documentation.  Google will
be your friend in this regard.


If you are using Enthought's Canopy package, all of the
  libraries that you need are already installed.  If they aren't
  installed you'll have to download the libary and install it. Once
  the library is installed, it can be imported like this:
\begin{Verbatim}
import math
\end{Verbatim}
This imports a library called \texttt{math}.  Functions inside the
math library can be used like this
\begin{Verbatim}
math.sqrt(5.2)  # Take the square root of 5.2
math.pi         # Get the value of pi
math.sin(34)    # Find the sine of 34 radians
\end{Verbatim}
Using the functions inside a libary requires that you know what
functions are available.  This information is usually available in the
library's documentation.  Google will be a great resource here.

A library can be imported and then called by a different name like
this:
\begin{Verbatim}
import math as mt
\end{Verbatim}
Here, the short name \texttt{mt} was chosen for this library.
The desired functions can then be called like this
\begin{Verbatim}
mt.sqrt(5.2)  # Take the square root of 5.2
mt.pi         # Get the value of pi
mt.sin(34)    # Find the sine of 34 radians
\end{Verbatim}

Sometimes you may not want to import the entire library, just a few
functions. This can be done like this
\begin{Verbatim}
from math import sqrt,sin
\end{Verbatim}
and the \texttt{sqrt} and \texttt{sin} functions can then be used
without the library name before it, like this
\begin{Verbatim}
sqrt(5.5)
\end{Verbatim}
If you want to import every function inside of a libary, do this
\begin{Verbatim}
from math import *
\end{Verbatim}
Now, every function contained in \texttt{math} is available without
needing the \texttt{math.} prefix in front of it. \sidenote{For larger libraries, importing all of the
  functions can take a few seconds.  A better choice is to import only
  the functions that you need}

\subsection*{Native functions}
Python has many native functions: functions that are included with the
programming language.  You don't have to import native functions
because they came with the langauge.  You have already been using some
of them, like these
\begin{Verbatim}
len(mylist)  # Returns the length of a list.
float(5)     # Converts an integer to a float.
str(67.3)    # Converts a float to a string.
\end{Verbatim}
The functions: \texttt{len}, \texttt{float}, and \texttt{str} are all
built-in functions, and they each take a single argument.  Other
built-in function are found in the margin tables in the previous section.





Throughout this book, we will use a variety of libaries to accomplish
important tasks.  Instead of giving a complete description of each
library used and every function that it contains, we will simply
discuss the functions needed for each specific task.  The interested
reader is referred to the documentation of the various libaries for
further details.










