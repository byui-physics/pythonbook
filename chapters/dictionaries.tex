\chapter[Dictionaries and Classes]{Data Structure: Dictionaries and Classes}

The "Data Structure" of a variable tells you how information is stored in that variable.  So far in this book, we've worked mostly with three different types: single variables, lists, and Numpy arrays. This chapter introduces another very useful structure: the dictionary.

\section{Dictionaries}
Imagine that your program needed the name, age, and birthday of several of your classmates.  You could store that data as a list of lists, like this:
\begin{Verbatim}
classmates=[['Christine','Whitney',25, 'March 10'],
            ['Marvin','Flick', 21, 'April 17']]
\end{Verbatim}
Then you would have to remember that the first thing in the list is the first name, the second thing is the last name, then the age, then the birthday. You could get the last name of the second student with \code{classmates[1][1]}, and their birthday with \code{classmates[1][3]}.

Dictionaries make it much easier to store this sort of information.  Rather than using indices like a list, dictionaries store information in what are called key value pairs.  Here's our \code{classmates} list rewritten as a list of dictionaries:

\begin{Verbatim}
classmates=[{'First':'Christine','Last':'Whitney','Age':25,'Birthday':'March 10'},
            {'First':'Marvin','Last':'Flick','Age':21,'Birthday':'April 17'}]
\end{Verbatim}

In the first element of our list, the key \code{'First'} correspondes to the value \code{'Christine'}. This makes it much easier to keep strack of what kind of data is stored where.  Now, to get the first student's last name, you can use \code{classmates[0]['Last']}.

Dictionaries do not preserve order.  From Python's perspective, both of these dictionaries would be the same:
\begin{Verbatim}
dict_1={'First':'Christine','Last':'Whitney','Age':25,'Birthday':'March 10'}
dict_2 = {'Age':25,'Last':'Whitney','First':'Christine','Birthday':'March 10'}

\end{Verbatim}
Both \code{dict_1{'Age'}} and \code{dict_2{'Age'}} would have the same value, 25.

Here's a few operations you can do with dictionaries:
\begin{Verbatim}
student = {'First':'Marvin','Last':'Flick','Age':21,'Birthday':'April 17'}
student['Age']=22 #Change a value
student['Height']=1.77 #Add a key/value pair
student.pop('Last') #Remove a key/value pair

#Iterate over a dictionary's keys
for key in student:
    print(key)

#Iterate over key/value pairs
for key,value in student.items():
    print(key,value)

\end{Verbatim}


\section{Classes}
In Python, a class is a dictionary, with a few extra features. They are most commonly used when you have lots of different objects that will have the same data structure.  With the name, age, birthday example, you will have several different students\sidenote{The overall data structure is called a class, each individual student that you load with the class is referred to as an "object" or "instance".}, each with a name, age, and birthday.  By setting up a class, you can tell Python what to expect. Here's an example:
\begin{Verbatim}
class Student: #Declare the class name
    #This function runs whenever you make a new object with this class
    def __init__(self,First,Last,Age,Birthday):
        self.First=First  #self refers to the current object that you are making
        self.Last=Last
        self.Age=Age
        self.Birthday = Birthday

#Now Load the students:
firstStudent=Student('Christine','Whitney',25, 'March 10')
secondStudent = Student('Marvin','Flick', 21, 'April 17')

#Print the first student's last name
print(firstStudent.Last)
#You can also reference it like a dictionary
print(firstStudent['Last'])


#Make a list of students
student_list = [firstStudent, secondStudent]

\end{Verbatim}

In addition to saving information like a dictionary, you can also add functions to a class:

\begin{Verbatim}
class Student: #Declare the class name
    #This function runs whenever you make a new object with this class
    def __init__(self,First,Last,Age,Birthday):
        self.First=First  #self refers to the current object that you are making
        self.Last=Last
        self.Age=Age
        self.Birthday = Birthday

    def getOlder(self):  #By passing self to this function,
                         #The function has access to whatever is in the object
        self.Age+=1

#Now Load the students:
firstStudent=Student('Christine','Whitney',25, 'March 10')

#And use our function
firstStudent.getOlder()
print(firstStudent.Age)
\end{Verbatim}

\subsection*{A Physics Example}
Try to read through this next example and see if you can figure out how it works.  As a general outline, the program fills a two dimensional box with moving particles, lets the particles move around, and keeps the particles in the box by making them bounce off of the walls.  The program defines two classes: a box and a particle.

The particle class keeps track of where the particle is, and what its velocity is.  The particle class will also assign a random position and velocity (within a given range) whenever it creates a new particle.  Each particle has the ability to move itself, given an amount of time.

The box class sets up the box.  It keeps track of how big the box is, and what particles are in it.  It can move all of the particles in the box. It will also check if any particles have left the box, and makes them bounce off the walls instead.

Here's are the classes:

\begin{Verbatim}
#Define a particle class
class Particle:
    #Set up the initialization function
    def __init__(self,x_min,x_max,y_min,y_max,v_max):
        #Import a random number generator
        from random import random


        #Set the initial x position by randomly
        #choosing a value between x_min and x_max
        self.x=(x_max-x_min)*random()+x_min

        #Repeat for y
        self.y=(y_max-y_min)*random()+y_min

        #Choose a random speed between 0 and v_max
        v0=v_max*random()


        #This next part shows a fancy way to randomly generate an angle

        #First, choose a random sin(theta)
        sin_theta = random()

        #use sin^2+cos^2 = 1 to find cos(theta)
        cos_theta=(1-sin_theta**2)**0.5

        '''sin_theta and cos_theta are limited to positive values
        Randomly choose whether they should be positive or negative
        By multiplying by either 1 or negative 1.'''
        sin_theta*= 1.0 if random()>0.5 else -1.0
        cos_theta*= 1.0 if random()>0.5 else -1.0

        #Done generating the angle =================



        #Now give the particle its initial velocity
        self.vx=v0*cos_theta
        self.vy=v0*sin_theta


    #Set up a function that will move particles on a straight line path
    def move(self,dt): #dt is the time for which to move the particles
        self.x+=self.vx*dt
        self.y+=self.vy*dt



#Now build a class for the box
class Box:
    #Set up the initialization function
    '''
    The initialization function takes 4 inputs:
        x_size -> a list giving the size of the box in x -> [x_min, x_max]
        y_size -> a list giving the size of the box in y -> [y_min, y_max]
        num_parts -> the number of particles in the box
        v_max -> the maximum speed of a particle in the box '''

    def __init__(self,x_size,y_size,num_parts,v_max):
        #Load the box size
        self.x_size = x_size
        self.x_min = x_size[0]
        self.x_max = x_size[1]

        self.y_size = y_size
        self.y_min = y_size[0]
        self.y_max = y_size[1]

        #Fill the box with a list of particles
        self.particles=[
             Particle(self.x_min,self.x_max, #Use the particle class
                      self.y_min,self.y_max, v_max)
                      for _ in range(num_parts) ] #repeat num_parts times

    #Set up a function that keeps particles in the box by
    #bouncing them off the walls of the box.
    def check_walls(self):
        #Check each of the particles
        for particle in self.particles:
            #If a particle is outside of the box in the x direction,
            #Reflect it off of the walls until it it no longer outside
            #the box
            while particle.x > self.x_max or particle.x < self.x_min:
                #Find out whether the particle is above x_max or below x_min
                x_ref = self.x_max if particle.x > self.x_max else self.x_min
                #This part brings the particle back into the box
                #If it is one meter above x_max, this sets the particle
                #One meter below x_max, and will do the same for x_min
                particle.x=2*x_ref-particle.x

                #Flip the direction of its x-velocity
                particle.vx*=-1

            #Repeat for the y direction
            while particle.y > self.y_max or particle.y < self.y_min:
                y_ref = self.y_max if particle.y > self.y_max else self.y_min
                particle.y=2*y_ref-particle.y
                particle.vy*=-1

    #Set up a function that moves the particles
    def move_parts(self,dt):
        #First, move all of the particles
        [particle.move(dt) for particle in self.particles]

        #Bounce the particles off of the walls
        self.check_walls()

\end{Verbatim}

This program uses those classes to build a box with 2 particles, then plot the position of the particles as they are moved 50 times:

\begin{Verbatim}
#Box size
box_x = [-1,1]
box_y = [-2,2]

#build a box
my_box = Box(box_x,box_y,2,10)

import matplotlib.pyplot as plt

for i in range(50):
    #Plot the particles' positions
    plt.scatter(my_box.particles[0].x,my_box.particles[0].y,color='blue')
    plt.scatter(my_box.particles[1].x,my_box.particles[1].y,color='red')
    #Move the particles
    my_box.move_parts(.01)

plt.xlim(box_x)
plt.ylim(box_y)

plt.show()
\end{Verbatim}
