\chapter{Derivatives and Integrals}

\label{chap:Calculus}

\marginfig{Figures/centerdiff}{``Actual" shows the actual slope of the function at point b. ``b-c" shows the slope calculated using points b and c. ``a-c" shows the slope calculated using points a and c. Notice that ``a-c" is much closer to the actual slope than ``a-b".}

In numerical physics we represent functions like $f(x)$ as discrete points on
a grid.  If you are careful, you can use these discrete values to quickly
calculate numerical approximations to the derivative $f'(x)$ and the integral
$\int_a^b f(x) dx$.

\medskip

\section{Derivatives}
\index{Derivatives, numerical}

When working with numerical data, we usually don't know the exact function that generated the data.  Therefore, we cannot take a numerical derivative using the power law, or chain rule, or any other of the rules you learned in calculus.  However, a derivative is the slope of a line at a given point and you can find the slope of a line through any two points using:
\begin{equation}
\text{slope} = \frac{y_2-y_1}{x_2-x_1}
\end{equation}


In order to use the slope equation, you will need to choose two points.  When working with an evenly spaced grid, you will get the most accurate slope if you pick two points that are close to\sidenote{The further away the points are from each other, the less accurate the derivative is. To determine the exact uncertainty in your derivative requires Taylor expansions and some inventive algebra. However, here's a rough rule of thumb. If $f(x)$ changes
significantly over an interval in $x$ of about $L$, the first
derivative of $f(x)$ is fairly accurate as long as $\Delta x<=10^{-5}L$; for the second derivative  $\Delta x <=10^{-4}L$.} and centered around the point where you want the derivative (see Fig. \ref{Figures/centerdiff}). This technique is called a centered difference. Using index ($i$) notation, the derivative at a certain point ($f_i$) on a grid is
\begin{equation}
\frac{d f_i}{dx} \approx \frac{f_{i+1}-f_{i-1}}{2 \Delta x}
\end{equation}
which comes from finding the slope of the line through $f_i$'s nearest neighbors, $f_{i-i}$ and $f_{i+1}$.


As an example of what a good job centering does, try differentiating
$\sin{x}$ at $x=1$ this way:
\begin{Verbatim}
from numpy import sin
dfdx=(sin(1+1e-5)-sin(1-1e-5))/2e-5
\end{Verbatim}
Now take the ratio between the numerical derivative and the exact answer
$\cos(1)$ to see how well this does
\begin{Verbatim}
from numpy import cos
print(dfdx/cos(1))
\end{Verbatim}
You can also take the second derivative numerically using the formula
\begin{equation}
\frac{d^2 f_i }{ dx^2} \approx
\frac{f_{i+1}-2 f_i+f_{i-1} }{ \Delta x^2} .
\end{equation}
For example,
\begin{Verbatim}
from numpy import sin
d2fdx2=(sin(1+1e-4)-2*sin(1)+sin(1-1e-4))/1e-8
\end{Verbatim}
Again, we take the ratio between the numerical derivative and the exact
answer $-\sin(1)$ to see how well this does
\begin{Verbatim}
print( d2fdx2/(-sin(1)) )
\end{Verbatim}



\index{Derivative of an array} If you want to differentiate a function
defined by arrays $x$ and $f$, then the step size is already determined; you
just have to live with the accuracy obtained by using $h=\Delta x$, where
$\Delta x$ is the spacing between points in the $x$ array.  {\it Notice that
the data must be evenly spaced for the example we are going to give you to
work.}

We approximate the derivative at $x=x_j$ in the array by using the center differenced formula on inner points (1 to $N-1$), and linearly extrapolate to the end points:

\begin{codeexample}
\begin{VerbatimOut}{\listingFile}
from numpy import arange, sin, cos, zeros_like
from matplotlib import pyplot

dx=1./1000
x=arange(0,4,dx)
N=len(x)
f=sin(x)

# Do the derivative at the interior points all at once using
# the colon command

dfdx = zeros_like(f)  # Create array of zeros.
dfdx[1:N-1]=(f[2:N]-f[0:N-2])/(2*dx)  # Populate array with derivative values

# linearly extrapolate to the end points
dfdx[0]=2*dfdx[1]-dfdx[2]
dfdx[N-1]=2*dfdx[N-2]-dfdx[N-3]

# now plot both the approximate derivative and the exact
# derivative cos(x) to see how well we did
pyplot.plot(x,dfdx,'r-',x,cos(x),'b-')

# also plot the difference between the approximate and exact
pyplot.figure()
pyplot.plot(x,dfdx-cos(x),'b-')
pyplot.title('Difference between approximate and exact derivatives')
pyplot.show()
\end{VerbatimOut}
\end{codeexample}

Here is an example of a function that takes as inputs an array $y$
representing the function $y(x)$, and $dx$ the $x$-spacing between function
points in the array. It returns {\code yp} and {\code ypp}, numerical
approximations to the first and second derivatives. \index{Derivative
function}

\begin{codeexample}
\begin{VerbatimOut}{\listingFile}

# This function numerically differentiates the array y which
# represents the function y(x) for x-data points equally spaced
# dx apart.  The first and second derivatives are returned as
# the arrays yp and ypp which are the same length as the input
# array y.  Either linear or quadratic extrapolation is used
# to load the derivatives at the endpoints.  The user decides
# which to use by commenting out the undesired formula below.

from numpy import arange,cos,zeros_like
from matplotlib import pyplot

def derivs(y,dx):

    # load the first and second derivative arrays
    # at the interior points

    N=len(y)
    yp = zeros_like(y)  # Initialize array of zeros for first derivative
    ypp = zeros_like(y)  # Initialize array of zeros for second derivative

    yp[1:N-1]=(y[2:N]-y[0:N-2])/(2*dx)
    ypp[1:N-1]=(y[2:N]-2*y[1:N-1]+y[0:N-2])/(dx**2)

    # now use either linear or quadratic extrapolation to load the
    # derivatives at the endpoints


    # linear
    #yp[0]=2*yp[1]-yp[2]
    #yp[N-1]=2*yp[N-2]-yp[N-3]
    #ypp[0]=2*ypp[1]-ypp[2]
    #ypp[N-1]=2*ypp[N-2]-ypp[N-3]

    # quadratic
    yp[0]=3*yp[1]-3*yp[2]+yp[3]
    yp[N-1]=3*yp[N-2]-3*yp[N-3]+yp[N-4]
    ypp[0]=3*ypp[1]-3*ypp[2]+ypp[3]
    ypp[N-1]=3*ypp[N-2]-3*ypp[N-3]+ypp[N-4]

    return yp, ypp


# Build an array of function values
x=arange(0,10,.01)
y=cos(x)

# Then, since the function returns two arrays in the form
# yp,ypp, you would use it this way:
fp,fpp=derivs(y,.01)

# plot the approximate derivatives
pyplot.plot(x,fp,'r-',x,fpp,'b-')
pyplot.title('Approximate first and second derivatives')
pyplot.show()
\end{VerbatimOut}
\end{codeexample}

\section{Integrals}

\subsection*{Definite Integrals}
\index{Integrals, numerical} The simplest way to do numerical integrals is with the midpoint rule. It is simple to code and usually\sidenote{The trapezoid rule and Simpson's rule are both more accurate, but are also more difficult to code. As a general rule, use the easiest method that gives the level of accuracy that you need.} accurate enough. The midpoint rule approximates the integral $\int_a^b f(x) dx$ by subdividing the interval
$[a,b]$ into $N$ equal-width rectangles\sidenote{Calculate the width of the triangles with $\Delta x = (b-a)/N$.}. The height of each rectangle is equal to the value $f(x)$ at the center of the rectangle (see Fig.~\ref{Figures/figMidpoint}).


\marginfig{Figures/figMidpoint}{The midpoint rule works OK if the function is
    nearly straight across each interval like in the second rectangle. A smaller $\Delta x$ would make the integral much more accurate.}

To see how accurate this method is, try running the example below with $N=1000$ points, then
2000, then 4000, etc., and watch which decimal points are changing as you go
to higher accuracy.

\begin{codeexample}
\begin{VerbatimOut}{\listingFile}
from numpy import arange, cos,sin

N=1000
a=0
b=5.
dx=(b-a)/N

x=arange(.5*dx,b - 0.5 * dx, dx)  # build an x array of centered values
f=cos(x)  # load the function

# do the approximate integral
s=sum(f)*dx

# compare with the exact answer, which is sin(5)
err=s-sin(5)
print(err)
\end{VerbatimOut}
\end{codeexample}



%\subsection*{Definite Integral Function \code{defint.m}}
%
%\index{Definite integral function} This function uses the midpoint
%rule to integrate a function over a chosen interval using a chosen
%number of integration points. The function to be integrated must be
%coded in the sub function contained at the end of the function file
%\code{defint.m} (Note: Python's \code{quad} and especially \code{quadl}
%do the same thing, only better. This is just a simple example to
%show you how to program.)
%
%\index{Subfunctions}
%
%\begin{codeexample}[defint.m]
%\begin{VerbatimOut}{\listingFile}
%function s=defint(a,b,N)
%
%% this function uses the midpoint rule on N subintervals
%% to calculate the integral from a to b of the function
%% defined in the sub function at the bottom of this
%% function
%
%% load dx and build the midpoint rule x-values
%dx=(b-a)/N;
%x=a+.5*dx:dx:b-.5*dx;
%
%% use the function f(x) defined in the sub function below
%% to obtain the midpoint approximation to the integral and
%% assign the result to s
%
%s=sum(f(x))*dx;
%
%% here's the sub function
%
%function y=f(x)
%
%% define the function f(x) and assign it to y
%y=cos(x);
%\end{VerbatimOut}
%\end{codeexample}
%
%To use it, first edit the file \code{defint.m} so that the sub
%function at the bottom of the file contains the function you want to
%integrate. Then give \code{defint.m} the integration limits (say
%$a=0$ and $b=1$) and the number of points $N=1000$ to use in the
%midpoint rule like this
%\begin{Verbatim}
%defint(0,1,1000)
%\end{Verbatim}
%or
%\begin{Verbatim}
%s=defint(0,1,1000);
%\end{Verbatim}
%In the first case the approximate integral prints on the screen; in
%the second it doesn't print but is assigned to \code{s}.

\subsection*{Indefinite integrals}

What if you need to find the indefinite integral? For example,
maybe you need to calculate the integral of
\begin{equation}
f(x) = e^{-x^2}
\end{equation}
which has no analytical solutions. You can integrate this function
numerically in exactly the same way that you did the definite
integral with one difference: instead of saving just the total value of the integral, keep a cumulative total at each point.  Here is
one way that you could do it with Python:

\begin{Verbatim}
from numpy import arange,exp
from matplotlib import pyplot
N=1000
a=0
b=5.
dx=(b-a)/N

x=arange(.5*dx,b - 0.5 * dx, dx)  # build an x array of centered values
f=exp(-x**2)  # load the function

# do the approximate integral
s=[sum(f[:n]) * dx for n in range(len(f))]  #<- pay attention to this line.
print(s)
pyplot.plot(x,s)
pyplot.show()
\end{Verbatim}
You'll want to study the line starting with \verb!s=! until it makes
sense\sidenote{This notation is called a list comprehension. You can find out more about them in the List Comprehensions section of Chapter \ref{chap:advpython}}.  If this line doesn't make sense to you, we can make it more
explicit (but less elegant programatically) by replacing the
integration line with:
\begin{Verbatim}
s = []
thesum = 0
for i in f:
    thesum += i * dx
    s.append(thesum)
\end{Verbatim}

Here we have implemented the midpoint rule for integrating the function.
More complicated integration methods (trapezoidal, Simpson's rule,
etc.) can be implemented by changing the sum inside the loop.

\section{Python Integrators}
The \code{numpy} and \code{scipy} libraries have several functions
designed to perform numerical integration. For example, \code{numpy}
has a function called \code{cumsum} (sounds helpful here right?) that
integrates using the rectangle rule, just as we did above. Here is how
it works:

\begin{Verbatim}
from numpy import arange,exp,cumsum
from matplotlib import pyplot
N=1000
a=0
b=5.
dx=(b-a)/N

x=arange(.5*dx,b - 0.5 * dx, dx)  # build an x array of centered values
f=exp(-x**2)  # load the function

# do the approximate integral
s=cumsum(f*dx)
pyplot.plot(x,s)
pyplot.show()
\end{Verbatim}

When integrating, there
are two types of situations that are most likely:
\begin{enumerate}

\item You don't know the
function you are integrating but you have samples (data points) from it.
\item You do know the
function that you want to integrate.
\end{enumerate}




The Scipy libary has a sublibrary called \code{integrate} which
contains a host of useful functions for integrating when you only have samples.
\index{Indefinite integral function} The \code{cumtrapz} function from this library takes an array of
values in \code{y} and an $x$-spacing \code{dx} and returns an approximate
indefinite integral function $g(x)=\int_a^x y(x') dx'$ using the trapezoid
rule\sidenote{Because \code{cumtrapz} uses the trapezoid rule instead of the midpoint
rule, the array of function values must start at $x=a$ and be defined at the
edges of the intervals rather than at the centers.}.

Once you have your function values stored in an array, say \code{f}, you calculate
the indefinite integral like this:\index{Integration, Python's Cumtrapz}
\begin{Verbatim}
from scipy.integrate import cumtrapz
from numpy import arange,abs,cos,pi
from matplotlib import pyplot
dx = 0.01
x = arange(0,2 * pi,dx)
f = cos(x)
g=cumtrapz(f,x,initial = 0)
pyplot.plot(x,g,'ro',x,f,'b-')
pyplot.show()
\end{Verbatim}
where \code{dx} is the point spacing.



\index{Integrate, Python Integrator} \index{Integrate2, Python 2-D
Integrator} For simple integrals of functions sampled at discrete points
(e.g. measured data) it is usually\sidenote{Performing integrals on raw measured data is usually a
bad idea unless your data is very clean. If your data is noisy, you'll
probably want to consider smoothing it or fitting it to a curve before trying
to take an integral.} fine to use \code{trapz}\index{Integration, Python's Trapz} to calculate a
definite integral, or \code{cumtrapz} to calculate an indefinite integral.
Both functions treat the function as little line segments connecting the data
points in your array.

If you want to calculate the integral of a function rather than data stored
in an array, you have more options because you can calculate function values
at arbitrary points rather than having to interpolate between array elements.
For this type of problem, the functions
\code{quad}\sidenote{\code{dblquad} and \code{tplquad} are also
  available for double and triple integrations respectively.  Look for
online help for more details.},
\code{romberg}, and \code{quadrature} are good choices.  Here is an
example of how to use \code{romberg}:
\begin{Verbatim}
from numpy import sin, exp
from scipy.integrate import romberg

# First define the function you are wanting to integrate
def func(x):
   return sin(2 * x**2) * exp(-x)

integral = romberg(func,0,5)
print integral
\end{Verbatim}
The functions \code{quad} and \code{quadruature} can be used in the
same way.


% use the command \code{integral}.  This function takes
%references to a Python function in its arguments (like \code{fzero},
%\code{fsolve}, and \code{fminsearch} did) and adjust the grid step size
%to optimize accuracy.

%The \code{integral} function adaptively approximates the integration function with
%parabolas\sidenote{This method is called Simpson's Rule.} (as shown in Fig.~\ref{Figures/figSimpson}) or other functions instead of
%the rectangles of the midpoint method. As you can see from Fig.~\ref{Figures/figSimpson}, parabolas do a
%much better job, making \code{quad} a standard Python choice for integration.
%To use the \code{integral} command, you need to define your function using an
%anonymous function.  For example, to
%define an anonymous function to represent the function $f(x) = \cos(x)
%e^{-x}$, you use this syntax:
%\begin{Verbatim}
%f=@(x) exp(-x).*cos(x)|
%\end{Verbatim}
%After you have defined \code{f} this way, you can use it just like built-in
%Python functions like \code{sin} and \code{cos}.  For instance, try these
%commands after defining \code{f} as above:
%\begin{Verbatim}
%f(1)
%f(1:10)
%\end{Verbatim}
%Using the anonymous function syntax, we can use \code{integral} to integrate
%$\cos(x) e^{-x}$ from 0 to 2 like this:
%
%\marginfig{Figures/figSimpson}{Fitting parabolas (Simpson's Rule) works better.}
%
%
%\begin{codeexample}
%\begin{VerbatimOut}{\listingFile}
%clear; close all;
%
%f=@(x) exp(-x).*cos(x)
%
%integral(f,0,2,'AbsTol',1e-8)
%\end{VerbatimOut}
%\end{codeexample}
%
%Python also has a command \code{integrate2} that does double integrals. Here's
%how to use it to integrate the function $f(x,y) = \cos (xy)$.
%
%\begin{codeexample}[f2int.py]
%\begin{VerbatimOut}{\listingFile}
%clear; close all;
%
%f = @(x,y) cos(x.*y);
%
%integral2(f,0,2,0,2,'AbsTol',1e-10)
%\end{VerbatimOut}
%\end{codeexample}
