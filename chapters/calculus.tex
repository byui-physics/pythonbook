\chapter{Derivatives and Integrals}

\label{chap:Calculus}

In numerical physics we represent functions like $f(x)$ as discrete points on
a grid.  If you are careful, you can use these discrete values to quickly
calculate numerical approximations to the derivative $f'(x)$ and the integral
$\int_a^b f(x) dx$.

\medskip

\section{Derivatives}
\index{Derivatives, numerical}

The instructor of your first calculus class probably introduced the
derivative with a formula like this:
\begin{equation}
    \frac{df}{dx} =
    \lim_{ h \rightarrow 0} \frac{f(x+h)-f(x) }{ h} .
\end{equation}
To do a derivative numerically on a grid, we use the following slightly
different, but numerically more accurate, form:
\begin{equation}
\frac{df }{ dx} = \lim_{ h \rightarrow 0} \frac{f(x+h)-f(x-h) }{ 2 h} .
\end{equation}
It's more accurate because it's centered about the value of $x$ where we want
the derivative to be evaluated.

\marginfig{Figures/figDerivatives}{The centered derivative approximation works
    best.}

To see the importance of centering, consider Fig.~\ref{figDerivatives}. In
this figure we are trying to find the slope of the tangent line at $x=0.4$.
The usual calculus-book formula uses the data points at $x=0.4$ and $x=0.5$,
giving tangent line $a$. It should be obvious that using the ``centered''
pair of points $x=0.3$ and $x=0.5$ to obtain tangent line $b$ is a much
better approximation.

As an example of what a good job centering does, try differentiating
$\sin{x}$ at $x=1$ this way:
\begin{Verbatim}
from numpy import sin
dfdx=(sin(1+1e-5)-sin(1-1e-5))/2e-5
\end{Verbatim}
Now take the ratio between the numerical derivative and the exact answer
$\cos(1)$ to see how well this does
\begin{Verbatim}
from numpy import cos
print(dfdx/cos(1))
\end{Verbatim}
You can also take the second derivative numerically using the formula
\begin{equation}
\frac{d^2 f }{ dx^2} = \lim_{ h \rightarrow 0}
\frac{f(x+h)-2 f(x)+f(x-h) }{ h^2} .
\end{equation}
For example,
\begin{Verbatim}
from numpy import sin
d2fdx2=(sin(1+1e-4)-2*sin(1)+sin(1-1e-4))/1e-8
\end{Verbatim}
Again, we take the ratio between the numerical derivative and the exact
answer $-\sin(1)$ to see how well this does
\begin{Verbatim}
print( d2fdx2/(-sin(1)) )
\end{Verbatim}

You may be wondering how to choose the step size $h$. This is a little
complicated; take a course on numerical analysis and you can see how it's
done. But until you do, here's a rough rule of thumb. If $f(x)$ changes
significantly over an interval in $x$ of about $L$, approximate the first
derivative of $f(x)$ using $h=10^{-5}L$; to approximate the second derivative
use $h=10^{-4}L$.

\index{Derivative of an array} If you want to differentiate a function
defined by arrays $x$ and $f$, then the step size is already determined; you
just have to live with the accuracy obtained by using $h=\Delta x$, where
$\Delta x$ is the spacing between points in the $x$ array.  {\it Notice that
the data must be evenly spaced for the example we are going to give you to
work.}

The idea is to approximate the derivative at $x=x_j$ in the array by using
the function values $f_{j+1}$ and $f_{j-1}$ like this
\begin{equation}
f'(x_j) \approx \frac{f_{j+1}-f_{j-1} }{ 2 h} .
\end{equation}
This works fine for an $N$ element array at all points from $x_2$ to
$x_{N-1}$, but it doesn't work at the endpoints because you can't reach
beyond the ends of the array to find the needed values of $f$. So we use this
formula for $x_2$ through $x_{N-1}$, then use linear extrapolation to find
the derivatives at the endpoints, like this
\begin{codeexample}
\begin{VerbatimOut}{\listingFile}
from numpy import arange, sin, cos, zeros_like
from matplotlib import pyplot

dx=1./1000
x=arange(0,4,dx)
N=len(x)
f=sin(x)

# Do the derivative at the interior points all at once using
# the colon command

dfdx = zeros_like(f)  # Create array of zeros.
dfdx[1:N-1]=(f[2:N]-f[0:N-2])/(2*dx)  # Populate array with derivative values

# linearly extrapolate to the end points
dfdx[0]=2*dfdx[1]-dfdx[2]
dfdx[N-1]=2*dfdx[N-2]-dfdx[N-3]

# now plot both the approximate derivative and the exact
# derivative cos(x) to see how well we did
pyplot.plot(x,dfdx,'r-',x,cos(x),'b-')

# also plot the difference between the approximate and exact
pyplot.figure()
pyplot.plot(x,dfdx-cos(x),'b-') 
pyplot.title('Difference between approximate and exact derivatives')
pyplot.show()
\end{VerbatimOut}
\end{codeexample}

Here is an example of a function that takes as inputs an array $y$
representing the function $y(x)$, and $dx$ the $x$-spacing between function
points in the array. It returns {\tt yp} and {\tt ypp}, numerical
approximations to the first and second derivatives. \index{Derivative
function}

\begin{codeexample}[derivs.py]
\begin{VerbatimOut}{\listingFile}

# This function numerically differentiates the array y which
# represents the function y(x) for x-data points equally spaced
# dx apart.  The first and second derivatives are returned as
# the arrays yp and ypp which are the same length as the input
# array y.  Either linear or quadratic extrapolation is used
# to load the derivatives at the endpoints.  The user decides
# which to use by commenting out the undesired formula below.

from numpy import arange,cos,zeros_like
from matplotlib import pyplot

def derivs(y,dx):

    # load the first and second derivative arrays
    # at the interior points

    N=len(y)
    yp = zeros_like(y)  # Initialize array of zeros for first derivative
    ypp = zeros_like(y)  # Initialize array of zeros for second derivative

    yp[1:N-1]=(y[2:N]-y[0:N-2])/(2*dx)
    ypp[1:N-1]=(y[2:N]-2*y[1:N-1]+y[0:N-2])/(dx**2)
    
    # now use either linear or quadratic extrapolation to load the
    # derivatives at the endpoints


    # linear
    #yp[0]=2*yp[1]-yp[2]
    #yp[N-1]=2*yp[N-2]-yp[N-3]
    #ypp[0]=2*ypp[1]-ypp[2]
    #ypp[N-1]=2*ypp[N-2]-ypp[N-3]

    # quadratic
    yp[0]=3*yp[1]-3*yp[2]+yp[3]
    yp[N-1]=3*yp[N-2]-3*yp[N-3]+yp[N-4]
    ypp[0]=3*ypp[1]-3*ypp[2]+ypp[3]
    ypp[N-1]=3*ypp[N-2]-3*ypp[N-3]+ypp[N-4]

    return yp, ypp


# Build an array of function values
x=arange(0,10,.01)
y=cos(x)

# Then, since the function returns two arrays in the form
# yp,ypp, you would use it this way:
fp,fpp=derivs(y,.01)

# plot the approximate derivatives
pyplot.plot(x,fp,'r-',x,fpp,'b-') 
pyplot.title('Approximate first and second derivatives')
pyplot.show()
\end{VerbatimOut}
\end{codeexample}

\section{Integrals}

\subsection*{Definite Integrals}
\index{Integrals, numerical} There are many ways to do definite integrals
numerically, and the more accurate these methods are the more complicated
they become. But for everyday use the midpoint method usually works just
fine, and it's very easy to code. The idea of the midpoint method is to
approximate the integral $\int_a^b f(x) dx$ by subdividing the interval
$[a,b]$ into $N$ subintervals of width $h = (b-a)/N$ and then evaluating
$f(x)$ at the center of each subinterval. We replace $f(x) dx$ by $f(x_j) h$
and sum over all the subintervals to obtain an approximate integral. This
method is shown in Fig.~\ref{figMidpoint}. Notice that this method should
work pretty well over subintervals like [1.0,1.5] where $f(x)$ is nearly
straight, but probably is lousy over subintervals like [0.5,1.0] where the
function curves.


\marginfig{Figures/figMidpoint}{The midpoint rule works OK if the function is
    nearly straight across each interval. }


\begin{codeexample}
\begin{VerbatimOut}{\listingFile}
from numpy import arange, cos,sin

N=1000
a=0
b=5.
dx=(b-a)/N

x=arange(.5*dx,b - 0.5 * dx, dx)  # build an x array of centered values
f=cos(x)  # load the function

# do the approximate integral
s=sum(f)*dx

# compare with the exact answer, which is sin(5)
err=s-sin(5)
print(err)
\end{VerbatimOut}
\end{codeexample}

If you need to do a definite integral in Python, this is an easy way to do
it. And to see how accurate your answer is, do it with 1000 points, then
2000, then 4000, etc., and watch which decimal points are changing as you go
to higher accuracy.

%\subsection*{Definite Integral Function {\tt defint.m}}
%
%\index{Definite integral function} This function uses the midpoint
%rule to integrate a function over a chosen interval using a chosen
%number of integration points. The function to be integrated must be
%coded in the sub function contained at the end of the function file
%{\tt defint.m} (Note: Python's {\tt quad} and especially {\tt quadl}
%do the same thing, only better. This is just a simple example to
%show you how to program.)
%
%\index{Subfunctions}
%
%\begin{codeexample}[defint.m]
%\begin{VerbatimOut}{\listingFile}
%function s=defint(a,b,N)
%
%% this function uses the midpoint rule on N subintervals
%% to calculate the integral from a to b of the function
%% defined in the sub function at the bottom of this
%% function
%
%% load dx and build the midpoint rule x-values
%dx=(b-a)/N;
%x=a+.5*dx:dx:b-.5*dx;
%
%% use the function f(x) defined in the sub function below
%% to obtain the midpoint approximation to the integral and
%% assign the result to s
%
%s=sum(f(x))*dx;
%
%% here's the sub function
%
%function y=f(x)
%
%% define the function f(x) and assign it to y
%y=cos(x);
%\end{VerbatimOut}
%\end{codeexample}
%
%To use it, first edit the file {\tt defint.m} so that the sub
%function at the bottom of the file contains the function you want to
%integrate. Then give {\tt defint.m} the integration limits (say
%$a=0$ and $b=1$) and the number of points $N=1000$ to use in the
%midpoint rule like this
%\begin{Verbatim}
%defint(0,1,1000)
%\end{Verbatim}
%or
%\begin{Verbatim}
%s=defint(0,1,1000);
%\end{Verbatim}
%In the first case the approximate integral prints on the screen; in
%the second it doesn't print but is assigned to {\tt s}.

\subsection*{Indefinite integrals}

And what if you need to find the indefinite integral? For example,
maybe you need to calculate the integral of 
\begin{equation}
f(x) = e^{-x^2}
\end{equation}
which has no analytical solutions. You can integrate this function
numerically in exactly the same way that you did the definite
integral.  The only difference is that instead of summing up all of
the contributions into one number, you keep a cumulative sum.  Here is
one way that you could do it with Python:

\begin{Verbatim}
from numpy import arange,exp
from matplotlib import pyplot
N=1000
a=0
b=5.
dx=(b-a)/N

x=arange(.5*dx,b - 0.5 * dx, dx)  # build an x array of centered values
f=exp(-x**2)  # load the function

# do the approximate integral
s=[sum(f[:n]) * dx for n in range(len(f))]  #<- pay attention to this line.
print(s)
pyplot.plot(x,s)
pyplot.show()
\end{Verbatim}
You'll want to study the line starting with \verb!s=! until it makes
sense.  If this line doesn't make sense to you, we can make it more
explicit (but less elegant programatically) by replacing the
integration line with:
\begin{Verbatim}
thesum = 0
for i in f:
    thesum += i * dx

\end{Verbatim}

Here we have implemented the rectangle rule for integrating the function.
More complicated integration methods (trapezoidal, Simpson's rule,
etc.) can be implemented by changing the thing that is being added in
the loop.

\section{Python Integrators}
The \code{numpy} and \code{scipy} libraries have several functions
designed to perform numerical integration.  For example, \code{numpy}
has a function called \code{cumsum} (sounds helpful here right?) that
integrates using the rectangle rule, just as we did above. Here is how
it works:

\begin{Verbatim}
from numpy import arange,exp,cumsum
from matplotlib import pyplot
N=1000
a=0
b=5.
dx=(b-a)/N

x=arange(.5*dx,b - 0.5 * dx, dx)  # build an x array of centered values
f=exp(-x**2)  # load the function

# do the approximate integral
s=cumsum(f*dx)
pyplot.plot(x,s)
pyplot.show()
\end{Verbatim}

The Scipy libary has a sublibrary called {\tt integrate} which
contains a host of useful functions for integrating.  function will do a similar thing except that it
will use.
\index{Indefinite integral function} This function takes an array of function
values in {\tt y} and an $x$-spacing {\tt dx} and returns an approximate
indefinite integral function $g(x)=\int_a^x y(x') dx'$ using the trapezoid
rule. This rule says to use as the height of the rectangle on the interval of
width $h$ the average of the function values on the edges of the interval:
\begin{equation}
\int_x^{x+h} y(x') dx' \approx \frac{y(x)+y(x+h) }{ 2} h
\end{equation}
Because \texttt{cumtrapz} uses the trapezoid rule instead of the midpoint
rule, the array of function values must start at $x=a$ and be defined at the
edges of the subintervals of size $h$ rather than at the centers.  Once you
have your function values stored in an array, say \texttt{f}, you calculate
the indefinite integral like this:
\begin{Verbatim}
from scipy.integrate import cumtrapz
from numpy import arange,abs,cos,pi
from matplotlib import pyplot
dx = 0.01
x = arange(0,2 * pi,dx)
f = cos(x)
g=cumtrapz(f,x,initial = 0)
pyplot.plot(x,g,'ro',x,f,'b-')
pyplot.show()
\end{Verbatim}
where \texttt{dx} is the point spacing.

\index{Integration, Python's Cumtrapz} \index{Integration, Python's Trapz}
Here is a function that does essentially the same calculation as
\texttt{cumtrapz}, but is coded to be readable. This function takes
\texttt{dx} as an argument and multiplies it at each step of the calculation,
whereas \texttt{cumtrapz} assumes a step size of one. This code is more
intuitive, but less efficient. Study this code to see how it works, but use
\texttt{cumtrapz} and multiply by \texttt{dx} as above when you actually do
calculations.
\begin{codeexample}[indefint.py]
\begin{VerbatimOut}{\listingFile}
function g=indefint(y,dx)

% returns the indefinite integral of the function
% represented by the array y.  y(1) is assumed to
% be y(a), the function value at the lower limit of the
% integration.  The function values are assumed to be
% values at the edges of the subintervals rather than
% the midpoint values.  Hence, we have to use the
% trapezoid rule instead of the midpoint rule:
%
% integral(y(x)) from x to x+dx is (y(x)+y(x+dx))/2*dx

% The answer is returned as an array of values defined
% at the same points as y

% the first value of g(x) is zero because at this first value
% x is at the lower limit so the integral is zero

g(1)=0;

N=length(y);

% step across each subinterval and use the trapezoid area
% rule to find each successive addition to the indefinite
% integral

for n=2:N

   % Trapezoid rule
   g(n)=g(n-1)+(y(n-1)+y(n))*.5*dx;

end
\end{VerbatimOut}
\end{codeexample}



\index{Integrate, Python Integrator} \index{Integrate2, Python 2-D
Integrator} For simple integrals of functions sampled at discrete points
(e.g. measured data) it is usually fine to use {\tt trapz} to calculate a
definite integral, or {\tt cumtrapz} to calculate an indefinite integral.
Both functions treat the function as little line segments connecting the data
points in your array.  Performing integrals on raw measured data is usually a
bad idea unless your data is very clean. If your data is noisy, you'll
probably want to consider smoothing it or fitting it to a curve before trying
to take an integral.

If you want to calculate the integral of a function rather than data stored
in an array, you have more options because you can calculate function values
at arbitrary points rather than having to interpolate between array elements.
For this type of problem, use the command {\tt integral} (or {\tt integral2},
or {\tt integrate3} for multiple dimension integrals).  These functions take
references to a Python function in their arguments (like {\tt fzero},
\texttt{fsolve}, and \texttt{fminsearch} did) and adjust the grid step size
to optimize accuracy.

The Python routine {\tt integral} adaptively approximates the function with
parabolas (as shown in Fig.~\ref{figSimpson}) or other functions instead of
the rectangles of the midpoint method. The parabolic method is called
Simpson's Rule. As you can see from Fig.~\ref{figSimpson}, parabolas do a
much better job, making {\tt quad} a standard Python choice for integration.
To use the {\tt integral} command, you need to define your function using an
anonymous function.  For example, to
define an anonymous function to represent the function $f(x) = \cos(x)
e^{-x}$, you use this syntax:
\begin{Verbatim}
f=@(x) exp(-x).*cos(x)|
\end{Verbatim}
After you have defined {\tt f} this way, you can use it just like built-in
Python functions like {\tt sin} and {\tt cos}.  For instance, try these
commands after defining {\tt f} as above:
\begin{Verbatim}
f(1)
f(1:10)
\end{Verbatim}
Using the anonymous function syntax, we can use {\tt integral} to integrate
$\cos(x) e^{-x}$ from 0 to 2 like this:

\marginfig{Figures/figSimpson}{Fitting parabolas (Simpson's Rule) works better.}


\begin{codeexample}
\begin{VerbatimOut}{\listingFile}
clear; close all;

f=@(x) exp(-x).*cos(x)

integral(f,0,2,'AbsTol',1e-8)
\end{VerbatimOut}
\end{codeexample}

Python also has a command {\tt integrate2} that does double integrals. Here's
how to use it to integrate the function $f(x,y) = \cos (xy)$.

\begin{codeexample}[f2int.py]
\begin{VerbatimOut}{\listingFile}
clear; close all;

f = @(x,y) cos(x.*y);

integral2(f,0,2,0,2,'AbsTol',1e-10)
\end{VerbatimOut}
\end{codeexample}
