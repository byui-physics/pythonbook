\chapter{Advanced Python Techniques}
There are several very useful Python techniques where full walkthroughs and descriptions don't into the scope of this book.  However, many of them you may not come across through the course of a normal science/engineering education.

This chapter contains many of those, with brief descriptions and a few examples so that you as a reader can see first of all that they exist, second, when they might be useful, and third, what they are called so you can further research them on your own.

\section{Iterating with the enumerate function}
You've seen how to iterate over a list, extracting each element,
one-by-one.  There are times when, in addition to extracting the
elements of the thing you are iterating over, you'd like to also
extract it's location in the list.  You can do that with the
\code{enumerate} function.  Here's an example where every fourth
element is added up:  
\begin{Verbatim}
a = [4,5,2,8,99,65,78,32,39,67,92,120,567,23]
thesum = 0
for index,val in enumerate(a):
    if index % 4 == 0:
        thesum +=val
\end{Verbatim}
The function \code{enumerate} returns the list element \textbf{and}
it's location in the list.
\section{Lambda (unnamed) functions}
Sometimes, a function that you need to
define is quite simple (maybe even one line) and it'd be nice if you
could define it in one line.  Luckily, you can.  It's
called a lambda function and here is an example:
\begin{Verbatim}
f = lambda x: x**3  #Define the function
print f(5)   # Evaluate the function
g = lambda x,y: x + y # Function of two variables
print g(5,6)
\end{Verbatim}
Here we defined the function $f(x) = x^3$, a function of one variable,
and quickly evaluated it.  Your lambda functions can have as many
arguments as you'd like: just seperate the arguments with commas.  At
this point, you may be asking yourself why this is such a big deal:
Why would you ever really need to do this.  Well, there are some
Python \underline{functions that take functions as their
  arguments}.\sidenote{Stop and process that for a minute.}(as opposed
to taking simple numbers or strings) An example of this is the
\code{filter} function which serves to extract elements of a list
according to some criteria.  The exact details of the criteria are
specified using a lambda function.  Here's an example:
\begin{Verbatim}
a = [1,3,4,6,9,2,3,8]
b = list( filter( lambda x: (x % 2 == 0) , a ) )
print(b)
\end{Verbatim}
Notice how the \code{filter} function is used.  The first argument
is the criteria function: the function that dictates which of the list
elements you want extracted. It's not an integer or a float or a
sting.  It's actually a function: something that takes an argument and
returns a result.  In this case, the filter function was being used to
extract the elements of \code{a} that were multiples of $2$ and the
lambda function serves to check each number in the list and return
either \code{True} or \code{False}.

\section{Inline If Statements}
Many times, you will need to set up this sort of if statement:
\begin{Verbatim}
if x>3:
    y=10
else:
    y=-10
\end{Verbatim}
This line of code does exactly the same thing:
\begin{Verbatim}
y= 10 if x>3 else -10
\end{Verbatim}
This format is called a "ternary operator" or a "conditional expression". They can make your code a little easier to read (sometimes), but they are exceptionally helpful when you perform operations that will not let you write a multiline if/else statement, like when setting values in a list comprehension.

All conditional expressions follow the format: <value if true> if <conditional> else <value if false>.  Here's another example:
\begin{Verbatim}
mood='Bad'
print('Life is {}.'.format('terrible' if mood=='Bad' else 'wonderful'))
\end{Verbatim}


\section{List Comprehensions}\label{sec:comprehensions}
A list comprehension is a more compact way to write a for loop.  Also, because of how Python handles memory, a list comprehension will be much faster than a for loop.  As an example, here is a for loop that makes a list of the numbers from 0 to 10, squared:
\begin{Verbatim}
n_squared=[]
for n in range(10):
    n_squared.append(n**2)

\end{Verbatim}
This code will produce the exact same result, but faster:
\begin{Verbatim}
n_squared = [n**2 for n in range(10)].
\end{Verbatim}
This structure is referred to as a list comprehension. Here is the general format:\sidenote{Putting <> around a phrase is a stand in for "put this thing here". You cannot copy this exact code and expect it to run in Python.}
\begin{Verbatim}
[<Thing_in_new_list> for <item_in_old_list> in <old_list>]
\end{Verbatim}
Notice that you can only do list comprehensions on existing lists.  They do not work well for things like Euler's method, where the next value in a list depends on a previous value.

Here are a few more examples of useful list comprehensions:
\begin{Verbatim}
#Get only odd numbers from 0 to 20
a = [n for n in range(20) if n%2==1]

#Use a conditional expression to halve even numbers and square odds
b = [n/2 if n%2==0 else n**2 for n in range(10)]

#You don't just have to use numbers
c = ['Physics','is','the','best']
d = [word.upper() for word in c]
\end{Verbatim}

You can also put multiple \code{for} statements in one comprehension.  Execute this block of code and try to wrap your head around what each comprehension does.
\begin{Verbatim}
list_1 = [5,3,7,8]
list_2 = [2,7,10]

#Make a list of every possible pairing
combinations = [[x,y] for x in list_1 for y in list_2]
print(combinations)

#Remove any pairs where both numbers are the same
remove_doubles = [x for x in combinations if x[1]!=x[0]]
print(remove_doubles)

#Put all the numbers into one long list
long_list = [x for pair in remove_doubles for x in pair]
print(long_list)
\end{Verbatim}
Notice in the \code{long_list} example that \code{pair} must be set with the first \code{for} statement, then iterated over in the second. Python reads left to right.  This format will create an error since Python runs into the variable \code{pair} before it knows what it means:
\begin{Verbatim}
long_list = [ x for x in pair for pair in remove_doubles]
\end{Verbatim}


You can also use comprehensions with tuples and dictionaries\sidenote{For more information on tuples and dictionaries, see section \ref{sec:datastruct}.}:
\begin{Verbatim}
#Tuple comprehension  -> use ()
my_tuple = ((-1)**n/n for n in range(10))

#Dictionary comprehension
my_dict = {x:x**2 for x in range(30)}
\end{Verbatim}




\section{Data Structures}\label{sec:datastruct}
The "Data Structure" of a variable tells you how information is stored in that variable.  So far in this book, we've worked mostly with three different types: single variables, lists, and Numpy arrays. This section introduces three other very useful structures: the tuple, the dictionary, and the class.

\subsection*{Tuples}
This book has mentioned tuples a few different times.  They are very similar to lists, but with two key differences:
\begin{itemize}
\item Tuples cannot change size or shape\sidenote{This property is referred to as being {\bf immutable}}.  So \code{append}, \code{insert}, \code{pop}, and other similar methods do not work on Tuples.
\item Tuples are faster than lists, since they cannot change size or shape.
\end{itemize}
Creating a tuple is ver similar to creating a list. Just replace the \code{[]} with \code{()}.
\begin{Verbatim}
my_list = [5,'Hello',[72, 63], -1.431]
my_tuple = (5,'Hello',[72, 63], -1.431)

print(my_list[1])
print(my_tuple[1]) #Notice that tuples still use [] for indexing
\end{Verbatim}

Since tuples cannot be modified , they are great when you don't want
to accidentally modify your data set.
\subsection*{Dictionaries}
Imagine that your program needed the name, age, and birthday of several of your classmates.  You could store that data as a list of lists, like this:
\begin{Verbatim}
classmates=[['Christine','Whitney',25, 'March 10'],
            ['Marvin','Flick', 21, 'April 17']]
\end{Verbatim}
Then you would have to remember that the first thing in the list is the first name, the second thing is the last name, then the age, then the birthday. You could get the last name of the second student with \code{classmates[1][1]}, and their birthday with \code{classmates[1][3]}.

Dictionaries make it much easier to store this sort of information.  Rather than using indices like a list, dictionaries store information in what are called key value pairs.  Here's our \code{classmates} list rewritten as a list of dictionaries:

\begin{Verbatim}
classmates=[{'First':'Christine','Last':'Whitney','Age':25,'Birthday':'March 10'},
            {'First':'Marvin','Last':'Flick','Age':21,'Birthday':'April 17'}]
\end{Verbatim}

In the first element of our list, the key \code{'First'} correspondes to the value \code{'Christine'}. This makes it much easier to keep strack of what kind of data is stored where.  Now, to get the first student's last name, you can use \code{classmates[0]['Last']}.

Dictionaries do not preserve order.  From Python's perspective, both of these dictionaries would be the same:
\begin{Verbatim}
dict_1={'First':'Christine','Last':'Whitney','Age':25,'Birthday':'March 10'}
dict_2 = {'Age':25,'Last':'Whitney','First':'Christine','Birthday':'March 10'}

\end{Verbatim}
Both \code{dict_1{'Age'}} and \code{dict_2{'Age'}} would have the same value, 25.

Here's a few operations you can do with dictionaries:
\begin{Verbatim}
student = {'First':'Marvin','Last':'Flick','Age':21,'Birthday':'April 17'}
student['Age']=22 #Change a value
student['Height']=1.77 #Add a key/value pair
student.pop('Last') #Remove a key/value pair

#Iterate over a dictionary's keys
for key in student:
    print(key)

#Iterate over key/value pairs
for key,value in student.items():
    print(key,value)

\end{Verbatim}


\subsection*{Classes}
In Python, a class is a dictionary, with a few extra features. They are most commonly used when you have lots of different objects that will have the same data structure.  With the name, age, birthday example, you will have several different students\sidenote{The overall data structure is called a class, each individual student that you load with the class is referred to as an "object" or "instance".}, each with a name, age, and birthday.  By setting up a class, you can tell Python what to expect. Here's an example:
\begin{Verbatim}
class Student: #Declare the class name
    #This function runs whenever you make a new object with this class
    def __init__(self,First,Last,Age,Birthday):
        self.First=First  #self refers to the current object that you are making
        self.Last=Last
        self.Age=Age
        self.Birthday = Birthday

#Now Load the students:
firstStudent=Student('Christine','Whitney',25, 'March 10')
secondStudent = Student('Marvin','Flick', 21, 'April 17')

#Print the first student's last name
print(firstStudent.Last)
#You can also reference it like a dictionary
print(firstStudent['Last'])


#Make a list of students
student_list = [firstStudent, secondStudent]

\end{Verbatim}

In addition to saving information like a dictionary, you can also add functions to a class:

\begin{Verbatim}
class Student: #Declare the class name
    #This function runs whenever you make a new object with this class
    def __init__(self,First,Last,Age,Birthday):
        self.First=First  #self refers to the current object that you are making
        self.Last=Last
        self.Age=Age
        self.Birthday = Birthday

    def getOlder(self):  #By passing self to this function,
                         #The function has access to whatever is in the object
        self.Age+=1

#Now Load the students:
firstStudent=Student('Christine','Whitney',25, 'March 10')

#And use our function
firstStudent.getOlder()
print(firstStudent.Age)
\end{Verbatim}

\subsection*{A Physics Example}
Try to read through this next example and see if you can figure out how it works.  As a general outline, the program fills a two dimensional box with moving particles, lets the particles move around, and keeps the particles in the box by making them bounce off of the walls.  The program defines two classes: a box and a particle.

The particle class keeps track of where the particle is, and what its velocity is.  The particle class will also assign a random position and velocity (within a given range) whenever it creates a new particle.  Each particle has the ability to move itself, given an amount of time.

The box class sets up the box.  It keeps track of how big the box is, and what particles are in it.  It can move all of the particles in the box. It will also check if any particles have left the box, and makes them bounce off the walls instead.

Once the classes are built, this program uses those classes to build a box with 2 particles, then plot the position of the particles as they are moved 50 times. Here is the example code:

\begin{codeexample}
\begin{VerbatimOut}{\listingFile}
#Define a particle class
class Particle:
    #Set up the initialization function
    def __init__(self,x_min,x_max,y_min,y_max,v_max):
        #Import a random number generator
        from random import random


        #Set the initial x position by randomly
        #choosing a value between x_min and x_max
        self.x=(x_max-x_min)*random()+x_min

        #Repeat for y
        self.y=(y_max-y_min)*random()+y_min

        #Choose a random speed between 0 and v_max
        v0=v_max*random()


        #This next part shows a fancy way to randomly generate an angle

        #First, choose a random sin(theta)
        sin_theta = random()

        #use sin^2+cos^2 = 1 to find cos(theta)
        cos_theta=(1-sin_theta**2)**0.5

        '''sin_theta and cos_theta are limited to positive values
        Randomly choose whether they should be positive or negative
        By multiplying by either 1 or negative 1.'''
        sin_theta*= 1.0 if random()>0.5 else -1.0
        cos_theta*= 1.0 if random()>0.5 else -1.0

        #Done generating the angle =================



        #Now give the particle its initial velocity
        self.vx=v0*cos_theta
        self.vy=v0*sin_theta


    #Set up a function that will move particles on a straight line path
    def move(self,dt): #dt is the time for which to move the particles
        self.x+=self.vx*dt
        self.y+=self.vy*dt



#Now build a class for the box
class Box:
    #Set up the initialization function
    '''
    The initialization function takes 4 inputs:
        x_size -> a list giving the size of the box in x -> [x_min, x_max]
        y_size -> a list giving the size of the box in y -> [y_min, y_max]
        num_parts -> the number of particles in the box
        v_max -> the maximum speed of a particle in the box '''

    def __init__(self,x_size,y_size,num_parts,v_max):
        #Load the box size
        self.x_size = x_size
        self.x_min = x_size[0]
        self.x_max = x_size[1]

        self.y_size = y_size
        self.y_min = y_size[0]
        self.y_max = y_size[1]

        #Fill the box with a list of particles
        self.particles=[
             Particle(self.x_min,self.x_max, #Use the particle class
                      self.y_min,self.y_max, v_max)
                      for _ in range(num_parts) ] #repeat num_parts times

    #Set up a function that keeps particles in the box by
    #bouncing them off the walls of the box.
    def check_walls(self):
        #Check each of the particles
        for particle in self.particles:
            #If a particle is outside of the box in the x direction,
            #Reflect it off of the walls until it it no longer outside
            #the box
            while particle.x > self.x_max or particle.x < self.x_min:
                #Find out whether the particle is above x_max or below x_min
                x_ref = self.x_max if particle.x > self.x_max else self.x_min
                #This part brings the particle back into the box
                #If it is one meter above x_max, this sets the particle
                #One meter below x_max, and will do the same for x_min
                particle.x=2*x_ref-particle.x

                #Flip the direction of its x-velocity
                particle.vx*=-1

            #Repeat for the y direction
            while particle.y > self.y_max or particle.y < self.y_min:
                y_ref = self.y_max if particle.y > self.y_max else self.y_min
                particle.y=2*y_ref-particle.y
                particle.vy*=-1

    #Set up a function that moves the particles
    def move_parts(self,dt):
        #First, move all of the particles
        [particle.move(dt) for particle in self.particles]

        #Bounce the particles off of the walls
        self.check_walls()

#End of class definitions, the rest of this program uses them ==========================


#Box size
box_x = [-1,1]
box_y = [-2,2]

#build a box
my_box = Box(box_x,box_y,2,10)

import matplotlib.pyplot as plt

for i in range(50):
    #Plot the particles' positions
    plt.scatter(my_box.particles[0].x,my_box.particles[0].y,color='blue')
    plt.scatter(my_box.particles[1].x,my_box.particles[1].y,color='red')
    #Move the particles
    my_box.move_parts(.01)

plt.xlim(box_x)
plt.ylim(box_y)

plt.show()
\end{VerbatimOut}
\end{codeexample}

\section{Pandas}
Pandas is a Python package built to work with spreadsheet-like data, and it is very good at its job.

Pandas stores data in something called a "dataframe".  A dataframe is simply data stored in rows and columns. As an example, here is some sample data taken by an accelerometer sitting on an elevator floor:
\begin{Verbatim}
time,gFx,gFy,gFz
0.007,-0.0056,-0.0046,1.012
0.008,0.0007,0.0024,1.0022
0.008,0,0.0059,1.0039
0.009,0.0054,-0.0022,1.0032
0.009,-0.0015,-0.0056,1.0042
0.009,0.0037,-0.002,0.9951
0.01,-0.002,-0.002,1.002
0.014,0.009,-0.0024,1.0159
0.015,0.0012,-0.0037,1.01
0.017,-0.0115,-0.002,1.0012
0.019,-0.0022,-0.0015,1.001
0.021,0.0024,-0.0022,1.0166
\end{Verbatim}

Here is that same data loaded into a dataframe:
\begin{Verbatim}
    time     gFx     gFy     gFz
0  0.007 -0.0056 -0.0046  1.0120
1  0.008  0.0007  0.0024  1.0022
2  0.008  0.0000  0.0059  1.0039
3  0.009  0.0054 -0.0022  1.0032
4  0.009 -0.0015 -0.0056  1.0042
5  0.009  0.0037 -0.0020  0.9951
6  0.010 -0.0020 -0.0020  1.0020
7  0.014  0.0090 -0.0024  1.0159
8  0.015  0.0012 -0.0037  1.0100
9  0.017 -0.0115 -0.0020  1.0012

\end{Verbatim}

It has 10 rows (0-9) and 4 columns (time, gFx, gFy, gFz).  Here's an example of the few things you can do with Pandas.  This example assumes that the elevator data is saved in the same folder as the program and is titled \code{elevator.csv}\sidenote{"csv" stands for comma separated values.  It is a very common datafile type, where each row is another set of data, and each value is separated by a comma.  Two other common data file types that are useful to look into are "JSON" and "xml".}.

\begin{codeexample}
\begin{VerbatimOut}{\listingFile}
import pandas as pd
import numpy as np


#Load the datafile into a dataframe
elevator_data=pd.read_csv('elevator.csv')

#Print the first five lines of the dataframe
print(elevator_data.head(5))

#Load column names into a list

column_names = list(elevator_data.columns)

#Re-Zero the gFz column using the first five data points
elevator_data['gFz']-=elevator_data['gFz'][0:5].mean()

#Print the first 3 times
print(elevator_data['time'][0:3])

#convert accelerations from units of g to m/s^2
accel_columns=['gFx','gFy','gFz']
elevator_data[accel_columns]*=9.8

#Create a new column with the magnitude of the acceleration
elevator_data['a_mag']=np.sqrt((elevator_data[accel_columns]**2).sum(axis=1))

#Find the maximum accleration
max_accel = elevator_data['a_mag'].max()

#Zero out any accelerations that are less than 10% of the maximum.
zero_spots=elevator_data['a_mag']<0.1*max_accel
#loc is useful when you need to set values on only part of a dataframe
elevator_data.loc[zero_spots,accel_columns+['a_mag']]*=0
#Usually, you can do the same thing with
#elevator_data[zero_spots][accel_columns+['a_mag']]*=0
#But there are a few exceptions, so pandas will give a warning
#if you don't use .loc[]

#plot t vs a_mag
import matplotlib.pyplot as plt
plt.plot(elevator_data['time'],elevator_data['a_mag'])
plt.show()

#Save the dataframe to a file
elevator_data.to_csv('elevator_processed.csv')

\end{VerbatimOut}
\end{codeexample}
\section{Regular Expressions}
Python by itself is good at finding regular strings inside of others.  For example,
\begin{Verbatim}
my_string = "The quick brown fox"
print(my_string.find('fox'))
\end{Verbatim}
will print \code{16} since the word fox starts at \code{my_string[16]}. However, \code{find} will not help you get all of the dates out of this sort of data:
\begin{Verbatim}
date, temperature
05/08/2016, 69
05/09/2016, 59
05/10/2016, 51
05/11/2016, 56
05/12/2016, 70
05/13/2016, 78
05/14/2016, 78
\end{Verbatim}
You could search individually for every single date string, but there is a better way.

Regular expressions, or regex for short, are a way to match patterns in strings.  Here's an example that will match any date in the mm/dd/20yy format:
\begin{Verbatim}
'[0-1][0-9]/[0-9]{2}/20[0-9]{2}'
\end{Verbatim}
Let's break down that example piece by piece.  Square brackets \code{[]} will match whatever is listed inside the brackets. So, \code{[0-9]} will match any digit from zero to nine while \code{[0-1]} only matches zero or one.  Therefore, you can use \code{[0-1][0-9]} to match \code{00}, \code{01}, \code{02}, all the way up to \code{19}.  Putting curly braces \code{\{\}} around a number tells the computer how many times the thing before is repeated. Therefore, \code{21\{5\}} tells the computer to look for \code{211111}, and \code{[0-9]\{2\}} matches any two digit number. There is a list of common regular expressions in Tables \ref{tab:regexchar} and \ref{tab:regexop}.

In Python, you can use regex with the \code{re} library.  However, regex is widely used in many programming languages, not only in Python. Knowing regex is an incredibly useful skill for anyone who needs to analyze data with a computer.


\vspace{7mm}

\begin{tabular}{cp{0.4\textwidth}p{0.4\textwidth}}
Regex & \multicolumn{1}{c}{Description} & \multicolumn{1}{c}{Example} \\
\hline
\code{\\w} & Any word character (letter, number, or subscript) &\code{an\\w} matches \code{any} and \code{and}, but not \code{an,} \\
\code{\\W} & Anything that isn't a word character &\code{friend\\W} matches \code{friend.} and \code{friend,} but not \code{friends}\\
\code{\\s} & Any white space (spaces, tabs, etc) &\\
\code{\\S} & Anything but white space &\\
\code{\\n} & Newline (line break) &\\
\code{.}  & Any character (wildcard) &\\
\code{\\} & Escape character. Let's you use special characters as the actual character.& \code{.} is a wildcard. \code{\\.} is a period. \\
\code{[ ]} & Anything listed in the brackets. & \code{[ABC]} matches A, B, or C. \\
\code{[^ ]} & Anything not listed in the brackets & \code{[^0-9]} matches anything but the digits 0 through 9\\
\code{^} & beginning of a line & \code{^A} only matches A when it is the first character in a line of text. \\
\code{$} & end of a line & \code{A$} only matches A when it is the last character in a line of text. \\

\end{tabular}
\captionof{table}{Common regular expressions that represent characters.\label{tab:regexchar}}


\vspace{7mm}
\begin{tabular}{cp{0.4\textwidth}p{0.4\textwidth}}
Regex & \multicolumn{1}{c}{Description} & \multicolumn{1}{c}{Example} \\
\hline
\code{+} & one or more occurrences & \code{AB+} matches \code{AB}, \code{ABB}, and \code{ABBBBBB}\\
\code{*} & zero or more occurrences & \code{AB+} matches \code{A}, \code{AB}, and \code{ABBBBBB}\\
\code{?} & zero or one occurrence & \code{Hell?o} matches \code{Hello} and \code{Helo}\\
\code{\{n\}} & $n$ occurrences & \code{[0-9]{3}} matches any three digits\\
\code{\{n,\}} & $n$ or more occurrences & \code{[0-9]{3,}} matches \code{121} and \code{17572}\\
\code{\{n,m\}} & $n$ to $m$ occurrences & \code{[0-9]{3,4}} matches \code{121} and \code{1984} but not \code{17572}\\
\code{()} & Marks a pattern group & \code{A(BC){3}} matches \code{ABCBCBC}\\
\code{|} & "or". Must be used inside \code{()} & \code{A(BC|DE)} matches \code{ABC} and \code{ADE}

\end{tabular}
\captionof{table}{Common regular expression quantifiers. These quantifiers apply to the proceeding character or group. \label{tab:regexop}}






