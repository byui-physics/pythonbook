\chapter{Interpolation and Extrapolation}

\label{chap:interp}\index{Extrapolation} \index{Interpolation}

In computational physics we usually represent functions as arrays of values at
discrete points in time and space. But we often want to be able to find
function values at points not in the arrays. Finding function values between
data points in the array is called {\it interpolation}; finding function
values beyond the endpoints of the array is called {\it extrapolation}. A
common way to do both is to use nearby function values to define a polynomial
approximation to the function that is pretty good over a small region.  Both
linear and quadratic function approximations will be discussed here.

%% because Matlab's ode solvers don't use equally spaced
%% time steps, and because you might want equal spacing,
%% here's how you convert from Matlab's unequally-spaced (t,x,v)
%% to equally spaced data (te,xe,ve)
%
%N=length(t);
%taue=(tfinal-tstart)/(N-1);
%te=tstart + (0:taue:(N-1)*taue) ;
%te=te';  % convert te to a column vector, to match t
%xe=interp1(t,x,te,'spline');
%ve=interp1(t,v,te,'spline');
%
%% Note that you could have obtained equally-spaced points by
%% telling ode45 to give you the solutions at times you specify.
%% For instance, suppose you wanted 1024 points between t=0 and
%% t=200. You could build them like this (the code is commented)
%
%% N=1024;
%% taue=(tfinal-tstart)/(N-1);
%% te=tstart + (0:taue:(N-1)*taue) ;
%% [t,u]=ode45(@rhs,te,u,options);
%% xe=u(:,1);ve=u(:,2);


\medskip

\marginfig{Figures/figLineInterp}{Linear interpolation only works well over
intervals where the function is straight.}

\section{Manual Interpolation and Extrapolation}

\subsection*{Linear approximation}
A linear approximation can be obtained with just two data points,
say $(x_1,y_1)$ and $(x_2,y_2)$. You learned a long time ago that
two points define a line, and the two-point formula for a line
is
\begin{equation}
y = y_1+ \frac{(y_2-y_1) }{ (x_2-x_1)} (x-x_1)
\end{equation}
This formula can be used between any two data points to linearly
interpolate. For example, if $x$ in this formula is half-way between
$x_1$ and $x_2$ at $x=(x_1+x_2)/2$ then it is easy to show that
linear interpolation gives the obvious result $y=(y_1+y_2)/2$.

But you must be careful when using this method that your points are
close enough together to give good values. In
Fig.~\ref{figLineInterp}, for instance, the linear approximation to
the curved function represented by the dashed line ``a'' is pretty
poor because the points $x=0$ and $x=1$ on which this line is based
are just too far apart. Adding a point in between at $x=0.5$ gets
us the two-segment approximation ``c'' which is quite a bit better.
Notice also that line ``b'' is a pretty good approximation because
the function doesn't curve much.

This linear formula can also be used to extrapolate. A common way
extrapolation is often used is to find just one more function value
beyond the end of a set of function pairs equally spaced in $x$. If
the last two function values in the array are $f_{N-1}$ and $f_N$,
it is easy to show that the formula above gives the simple rule
\begin{equation}\label{eq:linExtrap}
f_{N+1} = 2 f_N-f_{N-1}
\end{equation}

You must be careful here as well: segment ``d'' in Fig.~\ref{figLineInterp}
is the linear extrapolation of segment ``b'', but because the
function starts to curve again ``d'' is a lousy approximation unless
$x$ is quite close to $x=2$.



\subsection*{Quadratic approximation}
Quadratic interpolation and extrapolation are more accurate than
linear because the quadratic polynomial $a x^2 + bx + c$ can more
easily fit curved functions than the linear polynomial $ax + b$.
Consider Fig.~\ref{figQuadInterp}. It shows two quadratic fits to the curved
function. The one marked ``a'' just uses the points $x=0,1,2$ and is
not very accurate because these points are too far apart. But the
approximation using $x=0,0.5,1$, marked ``b'', is really quite good,
much better than a two-segment linear fit using the same three
points would be.

\marginfig{Figures/figQuadInterp}{Quadratic interpolation follows the curves
    better if the curvature doesn't change sign. }

To derive the quadratic interpolation and extrapolation function,
we assume that we have three known points, $(x_1,y_1)$,
$(x_2,y_2)$, and $(x_3,y_3)$.  If our parabola $y=a x^2 + bx + c$ is
to pass through these three points, then the following set of
equations must hold
\begin{equation}
    \begin{aligned}
    y_1 &= a x_1^2 + bx_1 + c \\
    y_2 &= a x_2^2 + bx_2 + c \\
    y_3 &= a x_3^2 + bx_3 + c \\
    \end{aligned}
\end{equation}
Unfortunately, when you solve this set of equations for $a$, $b$,
and $c$, the formulas are ugly. If we simplify to the case where
the three points are part of an equally spaced grid, things are
prettier.  Let's assume equally spaced points spaced by $h$, so
that $x_1=x_2-h$ and $x_3=x_2+h$. In this case, the solutions
are\footnote{It is common in numerical analysis to derive this
result using \index{Taylor's theorem} Taylor's theorem, which
approximates the function $y(x)$ near the point $x=a$ as
\[
y(x) \approx y(a) + y'(a)(x-a) + \frac{1}{2} y''(a) (x-a)^2 + \cdots
\]
If we ignore all terms beyond the quadratic term in $(x-a)$) near a
point $(x_n,y_n)$, use an array of equally spaced $x$ values, and
employ numerical derivatives as discussed in
Chapter~\ref{chap:Calculus}, the Taylor's series becomes
\[
y(x) \approx y_n + \frac{y_{n+1}-y_{n-1} }{ 2 h} (x-x_n)
+ \frac{y_{n-1}-2 y_n + y_{n+1} }{ 2 h^2} (x-x_n)^2 ~.
\]
This can be solved to find $a$, $b$, and $c$.
}
\begin{equation}
    \begin{aligned}
    a &= \frac{y_1-2y_2+y_3}{2h^2} \\
    b &= \frac{y_3-y_1}{2h} -2x_2 a \\
    c &= y_2 + x_2 \frac{y_1-y_3}{2h} + x_2^2 a
    \end{aligned}
\end{equation}

With these coefficients, we can quickly find approximate $y$ values
near our three points using $y=a x^2 + bx + c$. This formula is
very useful for getting function values that aren't in the array.
For instance, we can use this formula to obtain the interpolation
approximation for a point half way between two known points, i.e.
$y_{n+1/2} \equiv y(x_n+h/2)$
\begin{equation}
y_{n+1/2} = -\frac{1}{8} y_{n-1} + \frac{3}{4} y_n + \frac{3}{8}
y_{n+1}
\end{equation}
and also to find the quadratic extrapolation rule for a data point
one grid spacing beyond the last point, i.e. $y_{N+1} \equiv y(x_N
+ h)$
\begin{equation}
y_{N+1} = 3 y_N - 3 y_{N-1} + y_{N-2} ~.
\end{equation}



\section{Matlab interpolaters}

\subsection*{Interp1}
\index{Interp1} Matlab has its own interpolation routine {\tt
interp1} which does the things discussed in the previous two sections
automatically. Suppose you have a set of data points $\{x,y\}$ and
you have a different set of x-values $\{x_i\}$ for which you want to
find the corresponding $\{y_i\}$ values by interpolating in the
$\{x,y\}$ data set. You simply use any one of these three forms of
the {\tt interp1} command:
\begin{Verbatim}
yi=interp1(x,y,xi,'linear')
yi=interp1(x,y,xi,'pchip')
yi=interp1(x,y,xi,'spline')
\end{Verbatim}
We haven't talked about spline interpolation yet.  It is a
piece-wise polynomial fit that typically does an excellent job of
matching smooth functions.

Here is an example of how each of these three types of interpolation
works on a crude data set representing the sine function.

\begin{codeexample}
\begin{VerbatimOut}{\listingFile}
clear; close all;

% make the crude data set with dx too big for
% good accuracy
dx=pi/5;
x=0:dx:2*pi;
y=sin(x);

% make a fine x-grid
xi=0:dx/20:2*pi;

% interpolate on the coarse grid to
% obtain the fine yi values

% linear interpolation
yi=interp1(x,y,xi,'linear');

% plot the data and the interpolation
plot(x,y,'b*',xi,yi,'r-')
title('Linear Interpolation')

% cubic interpolation
yi=interp1(x,y,xi,'pchip');

% plot the data and the interpolation
figure
plot(x,y,'b*',xi,yi,'r-')
title('Cubic Interpolation')

% spline interpolation
yi=interp1(x,y,xi,'spline');

% plot the data and the interpolation
figure
plot(x,y,'b*',xi,yi,'r-')
title('Spline Interpolation')
\end{VerbatimOut}
\end{codeexample}



\section{Two-dimensional interpolation}

\index{Interp2} \index{Interpolation, 2-dimensions} Matlab also knows how to
do 2-dimensional interpolation on a data set of $\{x,y,z\}$ to find
approximate values of $z(x,y)$ at points $\{x_i,y_i\}$ which don't lie on
the data points $\{x,y\}$.  In the completely general situation where your
data points $\{x,y,z\}$ don't fall on a regular grid, you can use the
command {\tt TriScatteredInterp} to interpolate your function onto an arbitrary
new set of points $\{x_i,y_i\}$, such as an evenly
spaced 2-dimensional grid for plotting. Examine the code below to see how
{\tt TriScatteredInterp} works, and play with the value of {\tt N} and see
how the interpolation quality depends on the number of points.
\begin{codeexample}
\begin{VerbatimOut}{\listingFile}
clear; close all;

% Make some "data" at random points x,y points
N=200;
x = (rand(N,1)-0.5)*6;
y = (rand(N,1)-0.5)*6;
z = cos((x.^2+y.^2)/2);

% Create an interpolating function named F
F = TriScatteredInterp(x,y,z,'natural');

% Create an evenly spaced grid to interpolate onto
xe = -3:.1:3;
ye = xe;
[XE,YE] = ndgrid(xe,ye);

% Evaluate the interpolation function on the even grid
ZE = F(XE,YE);

% plot the interpolated surface
surf(XE,YE,ZE);

% overlay the "data" as dots
hold on;
plot3(x,y,z,'.');
axis equal
\end{VerbatimOut}
\end{codeexample}

The {\tt TriScatteredInterp} command is very powerful in the sense
that you can ask it to estimate $z(x,y)$ for arbitrary $x$ and $y$ (within
your data range).  However, for large data sets it can be slow.  In the
case that your data set is already on a regular grid, its much faster to
use the {\tt interpn} command, like this:
\begin{codeexample}
\begin{VerbatimOut}{\listingFile}
clear; close all;

x=-3:.4:3; y=x;

% build the full 2-d grid for the crude x and y data
% and make a surface plot
[X,Y]=ndgrid(x,y);
Z=cos((X.^2+Y.^2)/2);
surf(X,Y,Z);
title('Crude Data')

% now make a finer 2-d grid, interpolate linearly to
% build a finer z(x,y) and surface plot it.

% Because the interpolation is linear the mesh is finer
% but the crude corners are still there
xf=-3:.1:3;
yf=xf;
[XF,YF]=ndgrid(xf,yf);
ZF=interpn(X,Y,Z,XF,YF,'linear');
figure
surf(XF,YF,ZF);
title('Linear Interpolation')

% Now use cubic interpolation to round the corners.  Note that
% there is still trouble near the edge because these points
% only have data on one side to work with, so interpolation
% doesn't work very well

ZF=interpn(X,Y,Z,XF,YF,'cubic');
figure
surf(XF,YF,ZF);
title('Cubic Interpolation')

% Now use spline interpolation to also round the corners and
% see how it is different from cubic.  You should notice that
% it looks better, especially near the edges.  Spline
% interpolation is often the best.

ZF=interpn(X,Y,Z,XF,YF,'spline');
figure
surf(XF,YF,ZF);
title('Spline Interpolation')
\end{VerbatimOut}
\end{codeexample}

In this example our grids were created using {\tt ndgrid}.  If you choose
to use the {\tt meshgrid} command to create your grids, you'll need
to use the command {\tt interp2} instead of {\tt interpn}.
