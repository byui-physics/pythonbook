\chapter{Interpolation and Extrapolation}

\label{chap:interp}\index{Extrapolation} \index{Interpolation}

In computational physics we usually represent functions as arrays of values at
discrete points in time and space. But we often want to be able to find
function values at points not in the arrays. Finding function values between
data points in the array is called {\it interpolation}; finding function
values beyond the endpoints of the array is called {\it extrapolation}.

You can do both either using sophisticated functions built into the Scipy library or using quick, manually coded methods.  This chapter discusses both approaches.

%% because Matlab's ode solvers don't use equally spaced
%% time steps, and because you might want equal spacing,
%% here's how you convert from Matlab's unequally-spaced (t,x,v)
%% to equally spaced data (te,xe,ve)
%
%N=length(t);
%taue=(tfinal-tstart)/(N-1);
%te=tstart + (0:taue:(N-1)*taue) ;
%te=te';  % convert te to a column vector, to match t
%xe=interp1(t,x,te,'spline');
%ve=interp1(t,v,te,'spline');
%
%% Note that you could have obtained equally-spaced points by
%% telling ode45 to give you the solutions at times you specify.
%% For instance, suppose you wanted 1024 points between t=0 and
%% t=200. You could build them like this (the code is commented)
%
%% N=1024;
%% taue=(tfinal-tstart)/(N-1);
%% te=tstart + (0:taue:(N-1)*taue) ;
%% [t,u]=ode45(@rhs,te,u,options);
%% xe=u(:,1);ve=u(:,2);


\medskip

\marginfig{Figures/figLineInterp}{Linear interpolation only works well over
intervals where the function is straight, or when the interval is small enough that the function is approximately straight.}

\section{Manual Interpolation and Extrapolation}
If you are going to Interpolate or Extrapolate large datasets, you are usually better off using the Scipy library.  However, you will often have a need to quickly calculate a boundary condition, or grab a single point between two others.  In these situations, knowing how to manually interpolate and extrapolate will significantly improve your program's performance. This section discusses linear and quadratic interpolation and extrapolation.
\subsection*{Linear approximation}
Two points define a line.  The equation of the line through those two points
(say $(x_1,y_1)$ and $(x_2,y_2)$) is
\begin{equation}\label{eq:pointline}
y = y_1+ \frac{(y_2-y_1) }{ (x_2-x_1)} (x-x_1)
\end{equation}
where $x$ is where you want to interpolate, and $y$ is the value given by the interpolation.\sidenote{If $x$ in eq. \ref{eq:pointline} is half-way between
$x_1$ and $x_2$ (at $x=(x_1+x_2)/2$), then the linear interpolation formula simplifies to $y=(y_1+y_2)/2$.}

When using linear interpolation, you must be careful  that your points are
close enough together that a line gives a good approximation to your data. In
Fig.~\ref{Figures/figLineInterp}, for instance, the linear approximation to
the curved function represented by the dashed line ``a'' is pretty
poor because the points $x=0$ and $x=1$ on which this line is based
are just too far apart. Adding a point in between at $x=0.5$ gets
us the two-segment approximation ``c'' which is quite a bit better.
Notice also that line ``b'' is a pretty good approximation because
the function doesn't curve much.

You can also use eq. \ref{eq:pointline} to extrapolate. Often,
you will need to find just one more point at the end of your data set. As long as your data is evenly spaced in $x$, eq. \ref{eq:pointline} simplifies to
\begin{equation}\label{eq:linExtrap}
y_{N+1} = 2 y_N-y_{N-1}
\end{equation}
where $y_N$ is the last data point, and $y_{N-1}$ is the second to last data point.

Just like with interpolation, you must be sure that the data point you are extrapolating to is roughly linear with the two points you are using to extrapolate: segment ``d'' in Fig.~\ref{Figures/figLineInterp}
is the linear extrapolation of segment ``b'', but because the
function starts to curve ``d'' is a lousy approximation once $x>2$.



\subsection*{Quadratic approximation}
Quadratic interpolation and extrapolation is more accurate than
linear with curved functions because the quadratic polynomial, $a x^2 + bx + c$, fits curve segments better than the linear polynomial, $ax + b$.
Though once again, you must be sure that the data you are fitting to is roughly parabolic. As an example, look at Fig.~\ref{Figures/figQuadInterp}. It shows two quadratic fits to the curved
function. The one marked ``a'' just uses the points $x=0,1,2$ and is
not very accurate because these points are too far apart. But the
approximation using $x=0,0.5,1$, marked ``b'', is really quite good,
and is much better than a two-segment linear fit\sidenote{See the fit labeled ``c'' in Fig.~\ref{Figures/figLineInterp}.} to the same three points.

\marginfig{Figures/figQuadInterp}{Quadratic interpolation follows the curves
    better if the curvature doesn't change sign. }

To derive the quadratic interpolation and extrapolation function,
we need three known points, $(x_1,y_1)$,
$(x_2,y_2)$, and $(x_3,y_3)$.  If our parabola $y=a x^2 + bx + c$ passes through all three points, then these equations will be true:
\begin{equation}
    \begin{aligned}
    y_1 &= a x_1^2 + bx_1 + c \\
    y_2 &= a x_2^2 + bx_2 + c \\
    y_3 &= a x_3^2 + bx_3 + c \\
    \end{aligned}
\end{equation}
Unfortunately, when you solve this set of equations for $a$, $b$,
and $c$, the formulas are ugly and hard to work with. Unless your data is evenly spaced, you are better off using a more robust fitter from a library.  However, if the data is evenly spaced, the result is much simpler.  If we assume that $x_1,x_2,$ and $x_3$ are separated by a distance $h$ (so
that $x_1=x_2-h$ and $x_3=x_2+h$) the solutions
are\footnote{It is common in numerical analysis to derive this
result using \index{Taylor's theorem} Taylor's theorem, which
approximates the function $y(x)$ near the point $x=a$ as
\[
y(x) \approx y(a) + y'(a)(x-a) + \frac{1}{2} y''(a) (x-a)^2 + \cdots
\]
If we ignore all terms beyond the quadratic term in $(x-a)$) near a
point $(x_n,y_n)$, use an array of equally spaced $x$ values, and
employ numerical derivatives as discussed in
Chapter~\ref{chap:Calculus}, the Taylor's series becomes
\[
y(x) \approx y_n + \frac{y_{n+1}-y_{n-1} }{ 2 h} (x-x_n)
+ \frac{y_{n-1}-2 y_n + y_{n+1} }{ 2 h^2} (x-x_n)^2 ~.
\]
which is a quadratic function of $(x-x_n)$, and you can just read off the coefficients $a$, $b$, and $c$.
}
\begin{equation}
    \begin{aligned}
    a &= \frac{y_1-2y_2+y_3}{2h^2} \\
    b &= \frac{y_3-y_1}{2h} -2x_2 a \\
    c &= y_2 + x_2 \frac{y_1-y_3}{2h} + x_2^2 a
    \end{aligned}
\end{equation}

With these coefficients, we can quickly find approximate $y$ values
near our three points using $y=a x^2 + bx + c$. This formula is
very useful for getting function values that aren't in the array.
For instance, we can use this formula to obtain the interpolation
approximation for a point half way between two known points, i.e.
$y_{n+1/2} \equiv y(x_n+h/2)$
\begin{equation}
y_{n+1/2} = -\frac{1}{8} y_{n-1} + \frac{3}{4} y_n + \frac{3}{8}
y_{n+1}
\end{equation}
and also to find the quadratic extrapolation rule for a data point
one grid spacing beyond the last point, i.e. $y_{N+1} \equiv y(x_N
+ h)$
\begin{equation}
y_{N+1} = 3 y_N - 3 y_{N-1} + y_{N-2} ~.
\end{equation}



\section{Python interpolaters}

\subsection*{Interp1}
\index{Interp1} The Scipy library has an interpolation routine called \code{
interp1d} which does the things discussed in the previous two sections
automatically. Suppose you have a set of data points $\{x,y\}$ and
you have a different set of x-values $\{x_i\}$ for which you want to
find the corresponding $\{y_i\}$ values by interpolating in the
$\{x,y\}$ data set. You simply use any one of these three forms of
the \code{interp1} command (assuming \code{x} and \code{y} are already
loaded with the data points):\sidenote{Spline interpolation breaks the data into chunks, then fits a polynomial to each chunk. We refer to fitting to different chunks of data as a piece-wise fit.  Splines tend to do an excellent job fitting to smooth functions.}
\begin{Verbatim}
from scipy.interpolate import interp1d
yi=interp1d(x,y)  # Linear interpolation
yi=interp1d(x,y,kind = 'slinear')  # Spline interpolation to 2nd order
yi=interp1d(x,y,kind = 'quadratic')
\end{Verbatim}

Here is an example of how each of these three types of interpolation
works on a crude data set representing the sine function.

\begin{codeexample}
\begin{VerbatimOut}{\listingFile}
from scipy.interpolate import interp1d
from numpy import pi,linspace, sin,random
from matplotlib import pyplot

# make the crude data set with dx too big for
# good accuracy
dx=pi/5
x=linspace(0,2 * pi,10)
y=sin(x)

# make a fine x-grid
xi=linspace(0,2 * pi,200)
# interpolate on the coarse grid to
# obtain the fine yi values

# linear interpolation
yi=interp1d(x,y,kind = 'linear')
# plot the data and the interpolation
pyplot.plot(x,y,'b*',xi,yi(xi),'r-')
pyplot.title('Linear Interpolation')

# quadratic spline interpolation
yi=interp1d(x,y,kind='quadratic');

# plot the data and the interpolation
pyplot.figure()
pyplot.plot(x,y,'b*',xi,yi(xi),'r-')
pyplot.title('Quadratic Spline Interpolation')

# cubic spline interpolation
yi=interp1d(x,y,kind = 'cubic');

# plot the data and the interpolation
pyplot.figure()
pyplot.plot(x,y,'b*',xi,yi(xi),'r-')
pyplot.title('Cubic Spline Interpolation')
pyplot.show()
\end{VerbatimOut}
\end{codeexample}
The data used to interpolate need not be spaced out regularly over the domain of
interest, but the quality of the interpolation will not be good when
there are big gaps between data points.  Note also that
\code{interp1d} is an {\it interpolator} not an {\it extrapolator}.
Hence if you try to ``interpolate'' at a location that is beyond the
domain of the sampled values (that's called extrapolation), you will
get an error.


\section{Two-dimensional interpolation}

\index{Interp2} \index{Interpolation, 2-dimensions} The Scipy library
also has function that will allow you to do 2-dimensional interpolation on a
data set of $\{x,y,z\}$ to find approximate values of $z(x,y)$ at
points $\{x_i,y_i\}$ which don't lie on the data points $\{x,y\}$.  In
the completely general situation where your data points $\{x,y,z\}$
don't fall on a regular grid, you can use the command \code{griddata}
to interpolate your function onto an arbitrary new set of points
$\{x_i,y_i\}$, such as an evenly spaced 2-dimensional grid for
plotting. Examine the code below to see how \code{griddata}
works, and play with the value of \code{N} and see how the
interpolation quality depends on the number of points.
\begin{codeexample}
\begin{VerbatimOut}{\listingFile}
from scipy.interpolate import griddata
from numpy import mgrid,cos,sin,arange,random,pi
from matplotlib import pyplot
from mpl_toolkits.mplot3d import Axes3D


# Make some "data" at random points x,y points
N=200
points = (random.rand(N,2) - 0.5) * 6
#z = points[:,0]*(1-points[:,0])*cos(4*pi*points[:,0]) *
sin(4*pi*points[:,1]**2)**2
z=cos((points[:,0]**2+points[:,1]**2)/2)

# Define grid of points to interpolate at
grid_x, grid_y = mgrid[-3:3:100j, -3:3:200j]

# Interpolate to points located at grid_x,grid_y and save to F
F = griddata(points,z,(grid_x,grid_y),method='cubic')

#Plot a contour map
pyplot.imshow(F.T, extent=(0,1,0,1), origin='lower')

#Plot a surface plot
fig = pyplot.figure()
ax = fig.gca(projection='3d')
ax.plot_surface(grid_x,grid_y,F)

# overlay the "data" as dots
pyplot.plot(points[:,0],points[:,1],z,'r.')
pyplot.axis('equal')
pyplot.show()
\end{VerbatimOut}
\end{codeexample}

The \code{griddata} command is very powerful in the sense
that you can ask it to estimate $z(x,y)$ for arbitrary\sidenote{\code{griddata} will work with data of any shape and interpolate it onto a grid, which can be very useful, but is a very computationally expensive process.} $x$ and $y$ (within
your data range).  However, \code{griddata} is slow, especially for large data sets.  In the case that your data set is already on a regular grid, it's much faster to
use the \code{interpn}\sidenote{\code{interpn} is built to interpolate in multiple dimensions. If your data is strictly two-dimensional, consider looking into one of these other interpolating functions in the \code{scipy} library: \code{interp2d} or \code{RectBivariateSpline}} command, like this:
\begin{codeexample}
\begin{VerbatimOut}{\listingFile}
from scipy.interpolate import interpn
from numpy import mgrid,cos,sin,arange,random,pi, vstack
from matplotlib import pyplot
from mpl_toolkits.mplot3d import Axes3D

# Generate the rough x and y grids
x_grid_rough, y_grid_rough = mgrid[-3:3.1:0.4,-3:3.1:0.4]
# calculate a function usings the rough x and y, then build a surface plot
Z=cos((x_grid_rough**2+y_grid_rough**2)/2)
fig = pyplot.figure()
ax = fig.gca(projection='3d')
ax.plot_surface(x_grid_rough,y_grid_rough,Z)
pyplot.title('Original')
pyplot.show()

#Build the finer grids to interpolate onto
#Note: these grids do not have to have uniform spacing, but they do need
#to be meshable
x_grid_fine, y_grid_fine = mgrid[-3:3.1:0.1,-3:3.1:0.1]

#interpn fits to tuples of arrays, so here we set up those tuples

#These are the original data points.  x_grid_rough[:,0] is a 1-d array
#that contains all of the x positions, y_grid_rough[0,:] holds the y positions
rough_points = (x_grid_rough[:,0],y_grid_rough[0,:])
#Make a tuple out of the grids that will be interpolated onto.
fine_grid= (x_grid_fine,y_grid_fine)

#interpn uses three different methods: nearest,linear, and splinef2d.
#Try each of them


#nearest - sets the value of the interploated point equal to the nearest data point
Z_fine = interpn(rough_points,Z,fine_grid,method='nearest')
fig = pyplot.figure()
ax = fig.gca(projection='3d')
ax.plot_surface(x_grid_fine,y_grid_fine,Z_fine)
pyplot.title('Nearest')
pyplot.show()

'''
linear interpolation
Notice that when you plot it, the mesh is finer, and it is much
smoother than nearest.
However, since the interpolation is linear, the sharp corners of the
rough grid are still there.
'''

Z_fine = interpn(rough_points,Z,fine_grid,method='linear')
fig = pyplot.figure()
ax = fig.gca(projection='3d')
ax.plot_surface(x_grid_fine,y_grid_fine,Z_fine)
pyplot.title('Linear')
pyplot.show()

'''
splinef2d
splinef2d does a two dimensional piecewise polynomial fit,
which can fit curves. Notice the increased smoothness
'''

Z_fine = interpn((x_grid_rough[:,0],y_grid_rough[0,:]),Z,(x_grid_fine,y_grid_fine),method='splinef2d')
fig = pyplot.figure()
ax = fig.gca(projection='3d')
ax.plot_surface(x_grid_fine,y_grid_fine,Z_fine)
pyplot.title('2d Spline')
pyplot.show()
\end{VerbatimOut}
\end{codeexample}


