\chapter*{Preface}

This is a tutorial to help you get started in Python. Examples of
Python code in this book are in font \code{like
this}. As you read through the text, type and execute in Python all
of the examples. Longer sections of code
are set off and named.

There are three major parts to this book: {\em Getting Started}, {\em The Basics}, and {\em Advanced Topics}.

{\em Getting Started} walks you through how to get Python installed and running, and then gives tips on what to do when your programs don't work.

We designed {\em The Basics} section to teach you Python, piece by piece.  Each chapter builds on the last, and teaches you something new. You should work through each chapter, one by one, in order.

{\em Advanced Topics} covers things that are very useful in Python, but it serves more as a reference guide.  Go through the chapters there either when you need to use what is in them, or if you just want to learn something new.  Each chapter works as a stand alone tutorial, assuming that you've already covered {\em The Basics}.

In addition to teaching you Python, this booklet can also be used as a reference manual because it is
short, it has lots of examples, and it has a table of contents and
an index.

 Please tell us about mistakes and make suggestions to improve the text
(nelsonla@byui.edu).

