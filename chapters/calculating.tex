\chapter{Calculating}
\label{chap:Calculating}
In a scientific setting, much of what you will ask Python to do will
involve math.  You've already seen how to do very simple math. Here we
will give you all the tools you will need to do any mathematical
calculation you could want.


\section{Mathematical functions}
\marginpar{\footnotesize\captionsetup{type=table}
  \vspace{0.5in}
\begin{tabular}{l}
\texttt{sin(x)} \\
\texttt{cos(x)} \\
\texttt{tan(x)} \\
\texttt{arcsin(x)} \\
\texttt{arccos(x)} \\
\texttt{arctan(x)} \\
\texttt{sinh(x)} \\
\texttt{cosh(x)} \\
\texttt{tanh(x)} \\
\texttt{sign(x)} \\
\texttt{exp(x)} \\
\texttt{sqrt(x)} \\
\texttt{log(x)} \\
\texttt{log10(x)} \\
\texttt{log2(x)} \\
\end{tabular}
\captionof{table}{A very small sampling of functions belonging to the
  \texttt{numpy} library.\label{tab:Numpy}}
}
 Common (and not-so common) mathematical functions like \texttt{exp}
and \texttt{sqrt} are available via the libraries \texttt{numpy},
\texttt{scipy}, and \texttt{math}.  There are some good reasons to not
use the \texttt{math} library, which we will discuss shortly.  Some
commonly-used mathemtical functions from these libraries can be found
in the tables.

\section{Numpy Arrays}
Often you'll find that you want to perform math on an entire set of
data.  For example, let's say you had a large data set
\begin{Verbatim}
x = [2.42762254  2.53691271  3.15932278  1.7128872   2.54105921  2.54094893
  2.55284336  2.36430906  2.37972415  2.70342833  2.2846214   2.37636944
  2.74236195  3.06429336  2.29889954  1.99944808  2.46066766  1.86346638
  2.69619554  1.81298331  2.96144256  3.020208    2.71914935  2.59783385
  2.41512769  2.84674515  2.92394769  3.15879826  2.25886137  3.04074924
  3.14635756  2.60488105  2.79643916  2.67695452  2.77874282  1.94903284
  2.60399377  1.88255081  2.38624122  3.43726289  2.46514806  2.74985076
  2.33684695  2.58710514  2.10996793  3.19191947  3.93418676  2.90987071
  2.52449511  1.71514896  2.42465365  2.24485334  2.88390193  2.97911184
  2.86770773  2.97543667  2.00454583  2.56522443  2.99691011  2.79259592
  2.01617544  1.66098216  2.59230004  2.31295971  3.49570792  2.37890997
  2.14965171  2.40578128  2.44831872  2.0519382   2.41011389  3.07252157
  2.50662296  2.49878442  1.97225157  2.00764702  2.67472532  3.02465629
  2.45257132  2.9325564   2.69301075  2.81356219  2.49886432  1.97998459
  2.86166356  3.24091275  2.83846089  2.58103089  2.23525104  2.85815534
  3.33391592  2.6850452   2.3267767   3.27800198  2.17433118  2.17612604
  2.80002452  2.48975877  3.01856681  2.34280246]
\end{Verbatim}
and you wanted to calculate the summation
\begin{equation}
\sum_{i=1}^N (x_i - D)^3
\end{equation}
where $x_i$ are the data given above and $D = 5$.  You could
calculate, one-by-one, each contribution in the sum and then add them
up.  But there has to be an easier way.  The easier way involves a
library called \texttt{numpy} (pronounced num-pie, short for numerical
python).  The main object used in this library is called an
\texttt{array}, which is very similar to a list except that an array
is intended for mathematical use.  Let's explore arrays a little more.

\subsection*{Array Creation}
There are several ways to create an array.  If you already have a list
of numbers and you just want to convert it to an array, you can do it
with \texttt{numpy}'s \texttt{array} function.:
\begin{Verbatim}
from numpy import array
xList = [2 , 3 , 5.2 , 2 , 6.7]
xArray = array(xList)
\end{Verbatim}
If you are looking for a function to create an array from scratch,
there are plenty of options.  The function \texttt{arange} is very similar to
the native \texttt{range} function that you have already seen.  The
difference is that \texttt{arange} creates an array object instead of
a list object and \texttt{arange} allows the stepsize to be less than
one.  Here is an example:
\begin{Verbatim}
from numpy import arange
myArray = arange(0,10,.1)
\end{Verbatim}
This will create an array that looks like this:
\begin{Verbatim}
[0,.1,.2,.3,.4,.5,.6.... 9.5,9.6,9.7,9.8,9.9,10]
\end{Verbatim}
\marginpar{\footnotesize\captionsetup{type=table} \vspace{-2.5in}
\begin{tabular}{lp{1.05in}}
\texttt{logspace}        & Returns numbers evenly spaced on a
log scale.  Same arguments as \texttt{linspace}\\ \\
\texttt{empty}        & Returns an empty array with the
specified shape\\ \\
\texttt{zeros}        & Returns an array of zeros with the
specified shape\\ \\
\texttt{ones}        & Returns an array of ones with the
specified shape.\\ \\
\texttt{zeros\_like}        & Returns an array of zeros with the
same shape as the provided array. \\ \\
\texttt{fromfile}        & Read in a file and create an array from the
data.\\ \\
\texttt{copy}        & Make a copy of another array.\\ \\
\texttt{mgrid}        & Create coordinate matrices from coordinate vectors.\\ \\
\texttt{meshgrid}        & Create coordinate matrices from coordinate vectors.\\ \\
\end{tabular}
\captionof{table}{A sampling of array-building functions in
  numpy. The arguments to the functions has been omitted to maintain
  brevity.  See online documentation for further details.\label{tab:Arraybuilding}}
}
 Another very useful function for array-creation is \texttt{linspace},
which creates an array by specifying the starting value, ending value,
and the number of elements that the array should contain.  For example:
\begin{Verbatim}
from numpy import linspace
myArray = linspace(0,10,10)
\end{Verbatim}
This will create an array that looks like this:
\begin{Verbatim}
[0,1.11111,2.222222,3.333333,4.4444444,5.55555555,6.666666,7.7777777,8.8888888,10]
\end{Verbatim}
Many other useful function for creating arrays are available.  Online
documentation is freely available.  Table \ref{tab:Arraybuilding} gives some of the more
heavily-used ones:

\subsection*{Simple Math with Arrays}
Once the array object is created, a whole host of mathematical
operations become available.  For example, you can square the array
and python knows that you want to square each element, or you can add
two arrays together and python knows that you want to add the
individual elements of the arrays.  You can add a constant value to
every element of an array, or even multiply two arrays together and
the elements of the first array are multiplied by the corresponding
element in the second.  Here's a sampling of examples.
\begin{Verbatim}
from numpy import array
xList = [2 , 3 , 5.2 , 2 , 6.7]
xArray = array(xList)    # Create first array
yArray = array([4,8,9.8,2.1,8.2,4.5])  # Create second array

c = xArray**2   # Square the elements of the first array
d = xArray + 3  # Add 3 to every element of the first array
e = xArray * 5  # Multipy every element of the first array by 5
f = xArray + yArray  # Add the elements of array one to the elements of
                     # array two
g = xArray * yArray  # Multiply the elements of array one to the elements of
                     # array two
\end{Verbatim}
In short, you can do all of the math that you were hoping you could do
when you first learned about Python lists.
\subsection*{Evaluating Functions on Arrays}
What if you wanted to evaluate the \texttt{sin} function over an
entire data set.  You surely don't want to loop over every single
value in the data set and evaluate the sine function on each number.
It turns out that the \texttt{math} library and the \texttt{numpy}
library both contain a function called \texttt{sin}.  The one from the
\texttt{numpy} library is designed to work on arrays but the one from
the \texttt{math} library is not. Here is an example
\begin{Verbatim}
import numpy as np
import math 

xList = [2 , 3 , 5.2 , 2 , 6.7]
xArray = array(xList)

c = np.sin(xArray)   # Works just fine, returning array of numbers.
d = math.sin(xArray) # Returns an error.
\end{Verbatim}
Notice that \texttt{math}'s version of \texttt{sin} does not
know what to do when you give it an array of numbers.  It only works
for single numbers.  \texttt{Numpy}'s \texttt{sin} function, on the
other hand, does know what to do with an array of numbers: it
calculates the sin of all the numbers.

\subsection*{Summing the elements}
Suppose you have a list(or array) of numbers and you'd like to add up
all of the elements.  Python has a built-in \texttt{sum} function and
there is also a \texttt{sum} function inside of \texttt{numpy} that
will do this. They both do the same thing for one-dimensional lists.
\begin{Verbatim}
a = [1.5,2.2,9.8,4.6]
b = sum(a)   #Use built-in sum function
from numpy import sum
c = sum(a)   # Use numpy's sum function
\end{Verbatim}
If you are summing up the elements of a two-dimensional list, the
built-in version of the function will not work and you will have to
use \texttt{numpy}'s version.
\begin{Verbatim}
a = [[1.5,2.2],[9.8,4.6]]  # Define a 2-d list, a list of lists
b = sum(a)   #Use built-in sum function, notice the error
from numpy import sum
c = sum(a)   # Use numpy's sum function, no error.
d = sum(a,axis = 1) 
\end{Verbatim}
Notice the extra argument to the \texttt{sum} function in the last
line.  The \texttt{axis=1} indicates that you want to sum up the
elements in each individual list and return a list of sums.  If you
had a higher-dimensional list, you could use \texttt{axis=2} or
\texttt{axis = 3} as well.
\subsection*{Accessing Multiple Elements}
There is additional functionality for accessing and slicing arrays
than for lists when the arrays become higher dimensional.  As an
example, here is a two-dimensional array
\begin{Verbatim}
from numpy import array
a = array([[1,2,3],[4,5,6],[7,8,9]])
\end{Verbatim}
Single elements of this array can easily be extracted, just as with
lists.
\begin{Verbatim}
from numpy import array
a = array([[1,2,3],[4,5,6],[7,8,9]])
b = a[0][2]
\end{Verbatim}
This will extract the number 3, which is the third element in the
first list.  In other words, it's element 0,2 in array \texttt{a}.
What if you wanted to extract multiple elements in one shot.  Could
you do that?  In fact you can and here is how:
\begin{Verbatim}
from numpy import array
a = array([[1,2,3],[4,5,6],[7,8,9]])
b = a[[0,2],[1,1]]
\end{Verbatim}
Pay special attention to the last line.  It will extract the numbers
2 and 8.  The first list (\texttt{[0,2]}) indicate the rows (first
dimension of the array) where the element is to be extracted and the
second list (\texttt{[1,1]}) indicate the columns that correspond to
the rows provided.  So elements \texttt{[0,1]} and \texttt{[2,1]} from
list \texttt{a} will be extracted.  The lists of indices can be as
long as you want them to be and all of the corresponding elements will
be extracted.
\subsection*{Boolean Slicing}
The accessing and slicing of Numpy arrays can be done in exactly the
same way that it is done for lists.  However, there is some
additionaly functionality available specifically for arrays that you
will at times find very helpful.  Most of these features involve
two-dimensional (or higher) arrays.  An array can be sliced according
to a provided criteria very easily by replacing the index with the
selection criteria as a boolean statement.
\begin{Verbatim}
from numpy import array
a = array([1,2,3,4,5,6])
b = a[a>2]
\end{Verbatim}
Here, only those elements that are greater than 2 will be extracted
from the array.
\subsection*{Multi-dimensional Slicing}
We've already shown you how to slice a list using the \texttt{:}
operator.  The same can be done with arrays.  However, for 2D (and
higher) arrays the slicing is more powerful (intuitive).  It can be
helpful to visualize an array as a matrix, even if it is not being
treated that way Mathematically.  For example, let's say that you
define the following array:
\begin{Verbatim}
from numpy import array
a = array([[1,2,3],[4,5,6],[7,8,9]])
\end{Verbatim}
which could be interpreted as this matrix:
\begin{equation}
\left( \begin{tabular}{ccc}
1 & 2 & 3\\
4 & 5 & 6\\
7 & 8 & 9\\
\end{tabular}
\right)
\end{equation}
If you wanted to slice out the following 2 x 2 sub-matrix:
\begin{equation}
\left(\begin{tabular}{cc}
5 & 6 \\
8 & 8 \\
\end{tabular}
\right)
\end{equation}
you could do it like this:
\begin{Verbatim}
b[1:3,1:3]  # Slice out a sub-array (Exactly what you wanted!)
\end{Verbatim}
This can't be done with lists, but using an array it's very simple.
Also note that you must use the \texttt{[x1:y1,x2:y2]} notation rather
than the \texttt{[x1:y1][x2:y2]} notation.  Use of the latter will not
fail, but it will not produce the sub-matrix desired.
\section{Matrices}
Matrix math is different from normal math.  That will become more
clear after you take a linear algebra class.  If you want to do matrix
math or linear algebra, \texttt{numpy}'s \texttt{matrix} object is what you
want to use.  A \texttt{matrix} object can be created in a few ways.  Here are
a few examples
\begin{Verbatim}
from numpy import matrix

a = matrix('1 2; 3 4')  # Create a 2 x 2 matrix from string
b = matrix([[1,2],[3,4]])  # Create 2 x 2 matrix from list
c = matrix('1;2;3;4')  #Create column vector: a 4 x 1 matrix
\end{Verbatim}
The first definition is a nice way to create a matrix from a string.
The \texttt{;} indicates the end of the rows.  You can also convert a
list or array into a matrix.  Note that if you print a matrix object
to screen, it will probably look the same as an array or list (or
similar).  The differences are the things you can do with a matrix
object as compared to an array object, or list object.

Once the matrix is defined, lots of cool and useful math becomes
available to you.  Here are a few examples:

\begin{Verbatim}
from numpy import matrix

a = matrix('1 2; 3 4')  # Create 2 x 2 matrix from string
b = matrix('5 6; 8 9')  # Create 2 x 2 matrix from string
col = matrix('3;4')  # Create 2 x 1 column vector

c = a.T   # Transpose the matrix
d = a.I   # Find inverse of matrix
e = a.H   # Find conjugate transpose of matrix
f = a * b # Matrix multiplication
g = b * col # Multiply matrix b to column vector
h = a**2   # Square the matrix. Not the same as squaring an array.
\end{Verbatim}
 

\section{Statistics}
\texttt{Numpy} has a statistical package that includes many useful
functions.  Below is an example that calculates the mean and standard
deviation of a data set
\begin{Verbatim}
import numpy as np

data = [1.3,7.8,4.5,9.83,2.23,3.67]  # Define the data set
dataMean = np.mean(data)    # Calculate the arithmetic mean
dataSD = np.std(data)       # Calculate the standard deviation
\end{Verbatim}

\subsection*{Sampling from a Distribution}