\chapter{File Input and Output}

\label{chap:fileio}

\section{Reading Files}
Often you will find yourself wanting to read the contents of a file
into Python so that you can perform a calculation or make a plot.
There are two possible types of files you may want to read: (i)
highly-structured files with only numerical data and ii)
randomly-formatted files with non-numerical data mixed in with
numerical data.  By highly structured I mean files where i) every row of your
file has the same number of entries, ii) all of your entries are
seperated by the same character (maybe just a space but it could be
something else too), and iii) no non-numerical entries.  The following
data file is highly structured:
\begin{Verbatim}
1.2 4 5 2
3.1 2.9 4.9 1.9
4.2 1 9 2
3.1 2.9 4.9 1.9
1.2 4 5 2
3.1 2.9 4.9 1.9
\end{Verbatim}
It contains only numbers, each row has the same number of entries and
the numbers are all seperated by white space.  In contrast, here is an
example of a data file that is not highly structured:
\begin{Verbatim}
x y z t r(optional)
1.2 4 5 2 6
3.1 2.9&& 4.9 1.9
4.2 ,1 9 2
3.1 2.9, 4.9, 1.9& 5
1.2 4 5 2
3.1 2.9 4.9 1.9
\end{Verbatim}
Notice that: i) there are non-numerical entries, ii) rows have a
different number of entries, iii) entries are seperated by different
delimiters\sidenote{A {\bf delimiter} is a character that separates data points.} (sometimes \&\&, sometimes a comma is used, and sometimes
white space is used).

If your file is the first type(highly structured, numerical data
only), the Numpy library has functions that will make your life very
simple.  If your file falls into the latter category, you will have to
use native Python functions to read and parse the file contents manually based
on the formatting of the file.

\subsection*{Structured data files with numerical data only}
Let's start with highly-structured data files with only numerical entries.
In this case, Numpy's \code{loadtxt} is your friend.  It will load the
file and save the numerical data as an array.  Here is an example:
\begin{Verbatim}
from numpy import loadtxt
data = loadtxt('dataFile.txt')
\end{Verbatim}
If your entries were seperated by something other than white space
(commas for example), you could add the optional argument
\code{delimiter =','} to \code{loadtxt} to specify the seperator.
It's worth mentioning that \code{loadtxt} can handle the presence of
non-numerical entries if you need it to.  You'll want to use the
optional argument \code{dtypes} to help you specify what type of
data \code{loadtxt} should expect to find.  Thank you Numpy!

\subsection*{Randomly-formatted files with non-numerical data}
If your data file contains characters and does not have a well-defined
structure, you'll have to read the file using Python's native
functions and then parse the contents manually.  To read in a file you
first need to open the file using the \code{open} command.  Next, you
can read the contents using one of the following: \code{read},
\code{readlines}, or \code{readline}.  Finally, you should always
close the file using the \code{close} command.  Let's say you created
a file called \code{dataFile.txt} that had the following contents:
\begin{Verbatim}
x y z t r(optional)
1.2 4 5 2 6
3.1 2.9&& 4.9 1.9
4.2 ,1 9 2
3.1 2.9, 4.9, 1.9& 5
1.2 4 5 2
3.1 2.9 4.9 1.9
\end{Verbatim}
You could read this file like this
\begin{Verbatim}
f = open('dataFile.txt','r')  #Open the file
data=f.readlines()   # Read the entire file and save it as a list of strings
f.close()
\end{Verbatim}
The \code{'r'} in the \code{open} command indicates that you are
opening the file for the purpose of reading it.  Other options are
\code{'w'} for writing to files (more to come later on that),
\code{'a'} for appending, and \code{'r+'} for reading {\bf and}
writing.  The \code{readlines} command will read the entire contents
of the file (it's your problem if the size of the file exceeds your
RAM storage capability) and saves it as a {\bf list of strings}: one
string for each line in your file.  If you were to use \code{read}
instead of \code{readlines}, the entire file would be read and saved
to a {\bf single string}.  The \code{readline} command will read a
single line in the file.  An alternate way to read data that
automatically ensures that the file closes after the read completes is
done like this:
\begin{Verbatim}
with open('dataFile.txt','r') as file:
    data = file.readlines()
\end{Verbatim}
This will do the exact same thing that the above code does.  Python will automatically close the file once the program is no longer indented.

Notice that the data inside of a file is saved as a string (or list of
strings). That's most likely not the most convenient form if you are
dealing with numerical data.  For the above example, using
\code{readlines} to read the example file from above would save the
following list in Python:\sidenote{The string, \code{'\n'} is called either "new line" or "line break" and is the equivalent of hitting enter on a keyboard in a text editor.}
\begin{Verbatim}
[' x y z t r(optional)\n','1.2 4 5 2 6\n', ' 3.1 2.9&& 4.9 1.9\n',
' 4.2 ,1 9 2\n',' 3.1 2.9, 4.9, 1.9& 5\n','1.2 4 5 2\n','3.1 2.9 4.9 1.9\n']
\end{Verbatim}
How do you get at the data that you need?  This is where your ability
to work with and modify lists and strings will come in handy.  The
exact details of what you should do will depend on your specific data
file.  For this particular file, the commands \code{del},
\code{split}, and \code{float} would probably come in handy.
%\begin{Verbatim}
%del data[0]
%\end{Verbatim}
%This will delete the first element of the list,leaving only the
%numerical data. \reminder{\lefthand}{Note that using \code{del} modifies the
%  variable containing the list. So performing two \code{del} commands,
%one right after another, would delete the first two elements.}

\section{Writing to Files}
Writing Python data to file is as simple as is reading a file.  Just
like when you are reading a file, determining which function to use
will be determined by the type of data that you are writing.  If your
data is strictly numerical information stored in an array, Numpy has
the function you want to use.  On the other hand, if your data is rife
with inconsistencies, non-numerical data, etc, you'll have to use
something else.

\subsection*{Writing well-structured, numerical data}
If you have an {\bf array} of data and you want to save that data to a
file, you can do it with \code{Numpy}'s \code{savetxt} command, like
this:
\begin{Verbatim}
from numpy import savetxt
savetxt('filename.txt',myArray)
\end{Verbatim}
Maybe you have multiple one-dimensional arrays and you want them saved
to a file as columns.  You can do that like this:
\begin{Verbatim}
savetxt('filename.txt',(x,y,z))
\end{Verbatim}
where \code{x},\code{y}, and \code{z} are the one-dimensional
arrays and all have the same length. You can specify a delimiter (seperator) to include in between
data entries using the \code{delimiter=' '} option and putting the
delimiter in quotes.  See online documentation for other optional
arguments.

\subsection*{Non-numerical and randomly-structured file writes}
If you need to write a non-data file, you can write it line by using the \code{write} command.  Just like the \code{readlines()} command, first you need to open the file. Then, use the \code{write}\sidenote{\code{write} follows the same formatting rules as \code{print}} command to write in your open file:\sidenote{Note the use of \code{'\n'} to tell Python to start a new line in the file.}
\begin{Verbatim}
with open('myFile.txt','w') as f:
   f.write('The first line of the file\n')

   for i in range(2,10):
      f.write('Line Number {}\n'.format(i))
   f.write('The last line of the file')


\end{Verbatim}
Python closes the file once the program is no longer indented.



\marginpar{\footnotesize\captionsetup{type=table}
  \vspace{-4.5in}
\begin{tabular}{lp{1.05in}}
\code{sum(a)}  & Returns the sum of the elements of \code{a} \\ \\
\end{tabular}
\captionof{table}{A sampling of ``housekeeping'' functions for
lists.\label{tab:HousekeepingList}}
}


