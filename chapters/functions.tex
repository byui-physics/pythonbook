\chapter{Functions and Libraries}
\label{chap:Functions}

One of the most fundamental constructs for any programming language is
the function. A function is nothing more than a set of instructions
packaged up and given a name.  The function can be as long and complex
or short and succinct as you wish.

The main purpose of a function is to take in information, do some calculations or other things, then produce a result.  Here's a function you should already be comfortable with:
\begin{Verbatim}
print('Print is a function.')
\end{Verbatim}
The \code{print} function takes a string, and tells the computer to write that string to the console.  You do not need to know all of the details of how a function works in order to use it.  You just need to know what to give it, and what the result is.

All Python functions have the same general format: first you type the name of the function (\code{print}), then you put round parentheses \code{()} around whatever information you are giving to the function.  If a function needs more than one piece of information, you separate them with a comma:
\begin{Verbatim}
x=[1,2,3]
y=[4,5,6]
zipped=zip(x,y) #Join two lists into one, item by item
print(zipped)
\end{Verbatim}

You can think of a function as a
black box. Someone created the contents of the box, specified what
information needs to enter the box for it to be able to accomplish
its task, and what information will exit the box.  Why a black box?
Well, one benefit of functions is that the user doesn't need to
know what's inside.  They only need to know what information the box
need and what information the box will give back to her.

Python functions generally fall into three groups: functions that come standard with Python (called native functions), functions that you can import into Python, and functions that you write yourself.

\section{Native functions}
There are a few functions that are always ready to go whenever you run Python. They are included with the
programming language.  We call these functions native
functions.  You have already been using some
of them, like these
\begin{Verbatim}
len(mylist)  # Returns the length of a list.
float(5)     # Converts an integer to a float.
str(67.3)    # Converts a float to a string.
\end{Verbatim}
The functions: \code{len}, \code{float}, and \code{str} are all
built-in functions, and they each take a single argument.  Other
built-in function are found in the margin tables in the previous section.






\section{Imported Functions and Libraries}
Many times, you will need to go beyond what Python can do by itself\sidenote{For example, Python doesn't include \code{sin()} and \code{cos()} as Native functions.}. However, that doesn't mean you have to create everything you need to do from scratch.  Most likely, the function that you need has already been coded. Somebody
else created the function and made it available to anyone
who wants it.  Groups of functions that perform similar tasks are
typically bundled together into libraries. These libaries can be
imported and the functions that they contain can be used.

It is critical that you know what information(variables) the function expects you to give it and what
useful information the function will give back to you.  This
information can be found in the library's documentation. Most libraries have great documentation with lists of the included functions, what the functions do, what the functions expect, and examples on how to use the most common ones.  You can usually find the library documentation by searching the internet for the library's name, plus "Python documentation".

Providing a complete list of all available libraries and function is well beyond
the scope of this book. Instead, we'll illustrate how to import
functions and use them.  As you use Python more and more you
should get in the habit of searching out the appropriate library to
accomplish the task at hand. When faced with a task to accomplish,
your first thought should be, `` I'll bet somebody has already done that.
I'm going to try to find that library.''



Let's see how to import libraries and use their functions\sidenote{Using the functions inside a libary requires that you know what
functions are available.  This information is usually available in the
library's documentation.  Google will be a great resource here.}. You've
already seen how to perform very simple mathematical calculations. ($5/6, 8^4$,
etc..)  For more complex
mathematical calculations, like $\sin(\frac{\pi}{2})$ or $e^{2.5}$,
you'll need to import these functions from a library.
\begin{Verbatim}
import math
\end{Verbatim}
This imports a library called \code{math}.  This command is like telling to go get the \code{math} book off of the shelf.  Functions inside the
math library can be used like this
\begin{Verbatim}
math.sqrt(5.2)  # Take the square root of 5.2
math.pi         # Get the value of pi
math.sin(34)    # Find the sine of 34 radians
\end{Verbatim}
The \code{math.} before each function is equivalent to telling Python "Use the \code{sqrt()} function that you find in the \code{math} book I told you to grab." If you just type
\begin{Verbatim}
sqrt(5.2)
\end{Verbatim}
Python won't know where to find the \code{sqrt} function and will give you an error.



A library can be imported and then called by a different name like
this:
\begin{Verbatim}
import math as mt
\end{Verbatim}
Here, the short name \code{mt} was chosen for this library.  This tells Python "I'm going to call the \code{math} book \code{mt}."
The desired functions can then be called like this
\begin{Verbatim}
mt.sqrt(5.2)  # Take the square root of 5.2
mt.pi         # Get the value of pi
mt.sin(34)    # Find the sine of 34 radians
\end{Verbatim}

Sometimes you may not want to import the entire library, just a few
functions. This can be done like this
\begin{Verbatim}
from math import sqrt,sin,pi
\end{Verbatim}
This code tells Python "Go grab \code{sqrt}, \code{sin}, and \code{pi} from the \code{math} book.  Then, you can use the \code{sqrt} and \code{sin} functions
without the library name before it, like this
\begin{Verbatim}
sqrt(5.5)
sin(pi)
\end{Verbatim}
but, you will only have access to the functions you imported, not all of the functions in the \code{math} library.

If you want to import every function inside of a library and not have to use a prefix, do this\sidenote{For larger libraries, importing all of the
  functions this way can take a long time.  A better choice is to import only
  the functions that you need.  You will also get some unexpected results if you use this method to import two different libraries that have functions with the same name.}
\begin{Verbatim}
from math import *
\end{Verbatim}
Now, every function contained in \code{math} is available without
needing the \code{math.} prefix in front of it.

\section{User-defined functions}
Sometimes, you will need to do something over and over again that you can't find in a library.  You (the programmer) will need to write your own function. You do it like this:
\begin{Verbatim}
def myFunction(a,b):
    c = a + b
    d = 3.0 * c
    f = 5.0 * d**4
    return f
\end{Verbatim}
This function performs several simple calculations and then uses the
\code{return} statement to pass the final result back out of the
function.(what exits the black box) Every user-defined function must
begin with the keyword \code{def} followed by the function name (you
can choose it). The variables \code{a} and \code{b} are the
arguments to the function. (the things that must enter the black box)
Python does not use an \code{end} statement or anything like it to
signal the end of a function.  Instead, it looks for indentation to
determine where the function ends.

The function can be called like this
\begin{Verbatim}
def myFunction(a,b):
    #Everything that is part of the function
    #needs to be indented.
    c = a + b
    d = 3.0 * c
    f = 5.0 * d**4
    return f
#The rest of this code is not part of this function.
r = 10
t = 15
result = myFunction(r,t)
\end{Verbatim}
In this case, when the function is called, \code{a} gets assigned
the value of $10$ and \code{b} gets assigned the value of $15$.  The
result of this calculation is stored in the variable \code{result},
which is outside the function.  The rest of your program only sees \code{r}, \code{t}, and \code{result}.  If you try this:
\begin{Verbatim}
result = myFunction(r,t)
print(c)
\end{Verbatim}
you will get an error, since Python does not remember that inside the function \code{c=a+b}.

Let's look at one more example.  Here's (roughly) what Python does every time you use \code{math.sin(x)}:
\begin{Verbatim}
def sin(x):
    from math import factorial
    result=0
    for k in range(20): #This Starts a "for loop"
        sign=(-1)**k
        denom=factorial(2*k+1)
        result+=sign*x**(2*k+1)/denom
    return result
\end{Verbatim}
This function takes in a number (\code{x}) and returns the sine of \code{x}.  When you use \code{sin()}, the rest of your program has no idea that the variables \code{result}, \code{k}, \code{sign}, and \code{denom} were assigned along the way.  The main program only knows that \code{sin()} took \code{x} and returned a number that has the value of $\sin(x)$.


\section{(Advanced topic) Lambda (unnamed) functions}
The user-defined function \code{myFunction} above had 4 lines in it.  Other functions
could be even longer.  Sometimes, however, the function you need to
define is quite simple (maybe even one line) and it'd be nice if you
could define it in one line.  Luckily, you can.  It's
called a lambda function and here is an example:
\begin{Verbatim}
f = lambda x: x**3  #Define the function
print f(5)   # Evaluate the function
g = lambda x,y: x + y # Function of two variables
print g(5,6)
\end{Verbatim}
Here we defined the function $f(x) = x^3$, a function of one variable,
and quickly evaluated it.  Your lambda functions can have as many
arguments as you'd like: just seperate the arguments with commas.  At
this point, you may be asking yourself why this is such a big deal:
Why would you ever really need to do this.  Well, there are some
python \underline{functions that take functions as their
  arguments}.\sidenote{Stop and process that for a minute.}(as opposed
to taking simple numbers or strings) An example of this is the
\code{filter} function which serves to extract elements of a list
according to some criteria.  The exact details of the criteria are
specified using a lambda function.  Here's an example:
\begin{Verbatim}
a = [1,3,4,6,9,2,3,8]
b = list( filter( lambda x: (x % 2 == 0) , a ) )
print(b)
\end{Verbatim}
Notice how the \code{filter} function is used.  The first argument
is the criteria function: the function that dictates which of the list
elements you want extracted. It's not an integer or a float or a
sting.  It's actually a function: something that takes an argument and
returns a result.  In this case, the filter function was being used to
extract the elements of \code{a} that were multiples of $2$ and the
lambda function serves to check each number in the list and return
either \code{True} or \code{False}.

%In the previous chapter I showed you how to perform very simple
%mathematical calculations using variables.  You were also briefly
%introduced to some simple functions that are available in the Python
%language.  However, you were probably left wondering about more complex
%mathematical calculations, like $sin(\frac{\pi}{2})$ or $e^{2.5}$.
%The native Python language does not include functions for these
%mathematical operations and they must be imported from libaries.  In
%this chapter we will discussion functions and libraries.
%
%\section{User-defined functions}
%A function is simply a set of instructions to be performed when called
%upon.  You (the user) can create your own function like this
%\begin{Verbatim}
%def myFunction(a,b):
%    c = a + b
%    d = 3.0 * c
%    f = 5.0 * d**4
%    return f
%\end{Verbatim}
%This function performs several simple calculations and then uses the
%\code{return} statement to pass the final result back out of the
%function.  Every user-defined function must begin with the keyword
%\code{def} followed by the function name (you can choose it). The
%variables \code{a} and \code{b} are the arguments to the function.
%This means that when you want to call this function, you must provide
%these numbers.  This function can be called like this
%\begin{Verbatim}
%r = 10
%t = 15
%result = myFunction(r,t)
%\end{Verbatim}
%In this case, when the function is called, \code{a} gets assigned
%the value of $10$ and \code{b} gets assigned the value of $15$.  The
%result of this calculation is stored in the variable \code{result},
%which is outside the function.  You can think of a function as a black
%box. You create the contents of the box and specify what information
%must be passed into the black box.  The user of the black box need not
%know, or fully understand, everything that the designer put inside
%it.  He only needs to know what to give the box and what the box will
%give back to him.
%
%\section{Native functions}
%Python has many native functions: functions that are included with the
%programming language.  You have already been using some of them, like these
%\begin{Verbatim}
%len(mylist)  % Returns the length of a list.
%float(5)     % Converts an integer to a float.
%str(67.3)    % Converts a float to a string.
%\end{Verbatim}
%The functions: \code{len}, \code{float}, and \code{str} are all
%built-in functions, and they each take a single argument.  Other
%built-in function are found in the margin tables in the previous chapter.
%
%\section{Imported Functions and Libraries}
%If a function isn't included in the standard Python distribution and
%you don't want to write your own function, you may be able to find the
%function that you need elsehwere.  For the vast majority of the things
%you will want to do, there will be an existing libary that has the
%function you need. Many third-party developers have
%written very useful functions and bundled them into libraries.  To
%gain access to these great tools requires that you first install
%them.\sidenote{If you are using Enthought's Canopy package, all of the
%  libraries that you need are already installed} Once they are
%installed, you can import the library of functions like this:
%\begin{Verbatim}
%import math
%\end{Verbatim}
%This imports a library called \code{math}.  You can use a function
%inside of this libary like this
%\begin{Verbatim}
%math.sqrt(5.2)  % Take the square root of 5.2
%math.pi         % Get the value of pi
%math.sin(34)    % Find the sine of 34 radians
%\end{Verbatim}
%Using the functions inside a libary requires that you know what
%functions are available.  This information is usually available in the
%library's documentation.  Google will be a great resource here.
%
%A library can be imported and then called by a different name like
%this:
%\begin{Verbatim}
%import math as mt
%\end{Verbatim}
%Here, the short name \code{mt} was chosen for this library.
%The desired functions can then be called like this
%\begin{Verbatim}
%mt.sqrt(5.2)  % Take the square root of 5.2
%mt.pi         % Get the value of pi
%mt.sin(34)    % Find the sine of 34 radians
%\end{Verbatim}
%
%Sometimes you may not want to import the entire library, just a few
%functions. This can be done like this
%\begin{Verbatim}
%from math import sqrt
%\end{Verbatim}
%and the \code{sqrt} function can then be used without the library
%name before it, like this
%\begin{Verbatim}
%sqrt(5.5)
%\end{Verbatim}
%If you want to import every function inside of a libary, do this
%\begin{Verbatim}
%from math import *
%\end{Verbatim}
%Now, every function contained in \code{math} is available without
%needing the \code{math.} prefix in front of it.
%
%Throughout this book, we will use a variety of libaries to accomplish
%important tasks.  Instead of giving a complete description of each
%library used and every function that it contains, we will simply
%discuss the functions needed for each specific task.  The interested
%reader is referred to the documentation of the various libaries for
%further details.
%
%\section{The numpy (numerical python) library}
%Previously, we learned about lists.  Recall that lists were not
%designed to do vector/matrix math.  For all of your math needs in
%python, you should use a library called Numpy\sidenote{Short for
%  numerical python} (prounounced num-pie).  Covering every function
%avaialble in the numpy library would take way too long.  For now,
%we'll just discuss some common tasks.
%
%\subsection*{Arrays}
%Arrays are numpy objects and are the main object used by this
%library.  You can create an array using numpy's array function, like
%this
%\begin{Verbatim}
%from numpy import array
%
%a = [1,5,6,7]
%b = array(a)
%\end{Verbatim}
%The \code{array} function converts a python list to a numpy array.
%There are many other ways to create numpy arrays.  Here are a few
%examples
%\begin{Verbatim}
%from numpy import zeros, eye
%
%a = zeros(5)
%b = zeros([5,5])
%b = eye(5)
%\end{Verbatim}
%
%\code{zeros} will create an array of zeros, with the size of the
%array specified as an argument.



